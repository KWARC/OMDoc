\documentclass[12pt]{article}
\usepackage{a4wide,url}
\usepackage[hyper]{acronyms}
\usepackage{lstomdoc,xmlindex,myref}
\usepackage[show]{ed}
\lstset{language=[1.6]OMDoc,basicstyle=\scriptsize}
\def\llquote#1{\ensuremath{\langle\kern-.25em\langle\hbox{\sl{#1}}\rangle\kern-.25em\rangle}}

\title{Content Markup for Persons in OMDoc 1.6 Metadata} 
\author{Michael Kohlhase}

\begin{document}
\maketitle
\begin{abstract}
  We propose a semantic representation scheme for Persons in OMDoc 1.6\ednote{continue}
\end{abstract}

\section{Introduction}

In a content-oriented document format like {\omdoc}~\cite{Kohlhase:omdoc1.2} format, we
also need a content-oriented representation for metadata and in particular persons, who
come in as authors, contributors, or maintainers of documents. With persons, we face many
of the same problems we face with mathematical objects. We want to identify them as
individuals, we want to reference them in various communicative situations by salient
properties --- hoping that the interlocutor will be able to identify them from these, and
we want to have a way of finding out more about them. And we want to achieve all of this
without overburdening the author of documents. 

Content Markup for Bibliographic referencs is probably best organized as
BibTeX\ednote{cite it} does it. We have a database for bibliography entries, and have
citations just reference that. An obvious idea for our problem with persons is to do
something similar with persons, i.e. keep a database of general person descriptions like
the one in {\mysecref{example}} and just identify the person by a URI pointing to a
{\element{person}} element in such a database. Then document presentation applications
could look up the respective information needed by their document class and style it into
the result document. 

We propose to integrate this approach into {\omdocv{1.6}}, see {\mysecref{rnc}} for a
grammar and {\mysecref{example}} for an example.

\section{Related Work and Possible Extensions}\label{sec:relwork}

Another approach would be to use an ontology like FOAF (Friend of a
Friend~\cite{FOAF:webpage}) and rely on information on the web for describing knowledge
about persons. This approach is clearly well-suited to find up-to-date information about
people on the web, but allows the document author very little editorial control.\ednote{continue}

Note that our proposal does not answer the requirement to be able to uniquely identify
persons. Here an interface to a service like OpenID~\cite{OpenID:webpage} might
help\ednote{continue}. 

Finally, we could use {\omdoc} content dictionaries themselves to represent persons. We
would essentially turn people into ``mathematical objects'' and desribe them using
``mathematical formulae''. While this sounds like an abuse of the format on the first
glance, it would allow us to use the {\omdoc} technology for multiple ontological views on
persons as well.\ednote{make this clearer, and maybe generalize to position {\omdoc} as an
  ontology language.}

Certainly we should take care that the {\omdoc} ontology for persons is compatible with
external ontologies like FOAF,\ednote{contine; how can we integrate all of these?}

\section{An Example}\label{sec:example}
\lstinputlisting[language=RNC,nolol]{examples/persons.omdoc}

\section{The RelaxNG Schema Module for Persons}\label{sec:rnc}
\lstinputlisting[language=RNC,nolol]{rnc/omdocPER.rnc}

\section{Support}
This setup is supported by {\stex}~\cite{Kohlhase:ulsmf08,sTex:webpage} package for
bibliographic metadata~\cite{Kohlhase:dcm08}.  

\bibliographystyle{alphahurl}
\bibliography{kwarc}
\end{document}

% LocalWords:  ns attr xml byctx cd MMT cdbase mmt arith MMTWS mmtget saxon
% LocalWords:  Sacerdoti Coen Zacchiroli kwarc
