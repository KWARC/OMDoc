\documentclass[12pt]{article}
\usepackage{a4wide,url}
\usepackage[hyper]{acronyms}
\usepackage{lstomdoc,xmlindex}
\usepackage[show]{ed}
\usepackage{hyperref}
\lstset{language=[1.6]OMDoc,basicstyle=\scriptsize}
\def\llquote#1{\ensuremath{\langle\kern-.25em\langle\hbox{\sl{#1}}\rangle\kern-.25em\rangle}}

\title{Tables for OMDoc 1.6}
\author{Michael Kohlhase}

\begin{document}
\maketitle
\begin{abstract}
  We propose a semantic representation scheme for OMDoc 1.6\ednote{continue}
\end{abstract}

\section{Introduction}

Tables are an important means for presenting data as well as functional dependencies in
documents. They are used in the form of ledger sheets in accounting, to report the the
results of experiments in scientific documents, and

\section{Analysis}

\section{Representation}

The simplest representation for the grids underlying tables it to consider them as
(partial) functions from cell names to values. This representation is 

\section{RelaxNG Schema}
  \lstinputlisting[language=RNC,nolol]{rnc/omdocTAB.rnc}

\section{Caveats and Text Roadmap}

This note is still in a very early stage, and is intended mainly as a vehicle or
discussion between {\omdoc} developers. In particular, the element names provisional and
will probably evolve over time. At a later and more mature stage, part of the text might
go into the {\omdoc} specification.

\section{Acknowledgements}
This proposal has been greatly influenced by discussions with Andrea Kohlhase in the
{\tt{SACHS}} project at DFKI Bremen and is based on insights of our
paper~\cite{KohKoh:csbs08}

\bibliographystyle{alphahurl} 
\bibliography{kwarc}
\end{document}

% LocalWords:  ns attr xml byctx cd MMT cdbase mmt arith MMTWS mmtget saxon
% LocalWords:  Sacerdoti Coen Zacchiroli kwarc
