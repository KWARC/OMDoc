\documentclass[11pt]{bluenote}
\usepackage{url,graphicx,wrapfig}
\usepackage{amsfonts,amstext}
\usepackage{lstomdoc}
\usepackage{tikz,standalone}
\usetikzlibrary{mmt}
\usepackage{calbf}
\usepackage[show]{ed}
\usepackage[hyperref=auto,style=alphabetic,isbn=false]{biblatex}
\usepackage{bibtweaks}
\addbibresource{kwarcpubs.bib}
\addbibresource{extpubs.bib}
\addbibresource{kwarccrossrefs.bib}
\addbibresource{extcrossrefs.bib}

\def\modelsof#1{\overline{#1}}
\def\setdiff#1#2{#1\kern.2em\backslash\kern.1em #2}
\def\cn#1{\mathsf{#1}}

\usepackage{stex-logo}
\usepackage{hyperref} 
\def\KWARCslides#1{/Users/kohlhase/stc/slides/#1} 
\def\mmt{MMT\xspace}
\def\omdoc{OMDoc\xspace}
\def\omdocv#1{OMDoc#1\xspace}
\def\smglom{SMGloM}

\lstset{language=[1.6]OMDoc,basicstyle=\footnotesize\sf,columns=fullflexible}
\lstset{mathescape,aboveskip=-.5em,belowskip=-.5em}
\lstset{morekeywords={uses,termref}}

\def\om#1{\fbox{\ensuremath{#1}}}
\def\meta#1{\ensuremath{\langle\kern-.2em\langle\text{#1}\rangle\kern-.2em\rangle}}

\newenvironment{inpe}{\begin{inparaenum}[\em i\,\rm )]\ignorespaces}{\ignorespaces\end{inparaenum}\ignorespaces}
\newcommand\inperef[1]{\emph{\ref{inpe:#1}})}
\def\omdoc{OMDoc\xspace}
\def\defemph{\textbf}
\def\thmo#1#2{\cn{#1}\colon\kern-.15em{#2}}
\def\id{\text{Id}}

\title{Examples and Counterexamples in \omdoc} 
\author{Michael Kohlhase\\Computer Science,  Jacobs University\\\url{http://kwarc.info/kohlhase}}

\begin{document}
\maketitle
\begin{abstract}
  We look at examples in \omdoc, and the tension between examples as views in theory
  graphs and the practice of writing down examples in mathematical documents. 

  This note is part of the ongoing \omdocv2 language design~\cite{Kohlhase:old13} and
  studies an important aspect in more detail.
\end{abstract}
\tableofcontents\newpage

\section{Introduction}

Mathematical knowledge management has focused almost entirely on the ``holy trinity'' of
statements: axiom/definition, theorem, proof. This practice neglects a type of statement
that is at least as important for mathematical practice: examples.  Alison Pease and
Ursula Martin have a chart of the polymath discussion on a math olympiad problem and find
that 33\% of the dialogue turns in the proof is about examples. And they keep being strong
all across the 74 minute length of the proof endeavour~\cite{PeaMar:sfmm12}. In another
study, Ursula records that about 15\% of the questions on Mathoverflow are about examples
for some property~\cite{MarPea:wmtapm13}. In this note, we are most interested in the
structure and epistemic links of examples, so that we can model them adequately in
\omdoc2. 



\subsection{Examples as Mathematical Structures}
In mathematical documents, examples come in various forms. We can distinguish them by
epistemic level into the following categories: 
\begin{compactenum}
\item \emph{specific object-level}, e.g. ``13 is a prime number'', here the object ``13''
  is an example the property of being a ``prime number'', i.e. the predicate ``prime''
  evaluates true on ``13'' or in other words $13\in\cn{prime}$. We observe that every
  example carries a proof obligation. We have to prove that the object we exhibit indeed
  has the desired property it exemplifies. Note that in many instances in the literatures
  the proofs are elided -- presumably because they are considered trivial. 
\item \emph{generic object-level}, e.g. ``All differentiable functions are continuous.''
  here any object that is a differentiable function is an example for being continuous. In
  other words the set of differentiable functions is a subset of continuous functions. In
  this case, the proof obligation is more conspicuous, the $\epsilon/\delta$ proof of this
  fact is one of the first in any elementary calculus course.
\item \emph{specific theory-level} ``$(\mathbb{N},+)$ is a Monoid'', here we interpret
  $(\mathbb{N},+)$ as a mathematical ``theory'' given by the symbols $\mathbb{N}$ (natural
  numbers) and $+$ (addition). Here the relation given by the words ``is a'' has to be
  understood as ``can be interpreted as'' in the sense that there is a meaning-preserving
  mapping from the theory of monoids to the theory $(\mathbb{N},+)$.
\item \emph{generic theory level} A similar situation obtains with ``Boolean Algebras are
  Complete Lattices'', even though the phrase uses the plural as in generic object-level
  case.  In both cases, we have theory morphisms that allow
  framing~\cite{KohKoh:sifemp09}, we will elaborate on this in Section~\ref{sec:thygraph}.
\end{compactenum}
In general, an example states a satisfaction relation between objects and properties. At
the object level, either between a specific object (we call it the \defemph{exemplar}) or
a class of generic objects (which we will call the \defemph{exemplar class}\ednote{MK: I
  just made this up, is there any traditional name for this?}) and a property (which we
call the \defemph{exemplified property}). Thus from an epistemological perspective, an
example is an assertion of the form $p(e)$ or $\forall{x} E(x)\Rightarrow p(x)$, where $p$
is the exemplified property, $e$ the exemplar and $E$ the exemplar class, or if we want to
see things more set-theoretically: $e\in p$ or $E\subseteq p$. We call this assertion the
\defemph{proof obligation} of the example.


\subsection{Polarity of Examples}
Moreover we can distinguish counter-examples from supporting ones, we speak of the
\textbf{polarity} of the example. The two polarities exist at all four levels we discussed
above. To look at just the first, we can use 13 as an example for primality and 12 as a
counter-example. In the latter case we have the proof obligation $13\notin\cn{prime}$,
i.e. the negation of the obligation of the positive example. Counter-examples are most 
\begin{compactenum}
\item[5.] \emph{at the statement-level}, most saliently as supporting evidence for or
  counter-example against conjectures.  Here \ednote{continue, we need examples for
    counterexamples, recount the analysis from the OMDoc book in a nutshell.}
\end{compactenum}

\subsection{Examples as Narrative Structures}
Examples can come in different syntactic forms in a mathematical discourse
\begin{compactenum} 
\item \emph{Embellished examples}\ednote{MK: is there a good word for this?} are given as
  special paragraphs marked up with special cues -- prefixed by the emphasized keyword
  ``Example'', possibly numbered for reference, and sometimes even marked up in a special
  font family.
\item Examples in the \emph{narrative flow} are usually single sentences that state the
  exemplified property of the exemplar.
\item \emph{Inline examples} just consist of a phrase, often an interjection like ``for
  instance 13''. 
\end{compactenum}
No matter the syntactic structure, they always have the structural parts (which may be
elided) identified above. 

\subsection{Examples as Cognitive Devices}
In the process of mathematical discovery, examples are used to strengthen conjectures, to
show that concepts are non-vacuous or consistent, and to elucidate concepts.

Counterexamples are used to delineate the boundaries of possible theorems. By using
counterexamples to show that certain conjectures are false, mathematical researchers avoid
going down blind alleys and learn how to modify conjectures to produce provable
theorems. \ednote{integrate Lakatos view on counterexamples
  from~\cite[pp. 10-11]{Lakatos76}}

\section{Parallel Markup for Immediate Examples}\label{sec:immediate}

Following~\cite{KohIan:hlpmo13}, we will use \omdoc markup for the narrative aspects of
examples and \mmt markup for the formal ones. For object-level example, we have the
parallel markup in Figure~\ref{fig:simple-object}: we have a narrative representation,
which represents the structure of the presentation of the example as an (embellished)
paragraph with a statement and a proof as the justification. In this simple case, the
statement of the justifying assertion is identical to the example text, so we can just
represent the justification as a proof. The content representation is just a special
constant representing the proof obligation. We have indicated the parallels between
content and narrative by cross-references (the \lstinline|<meta>| elements with the
\lstinline|o:formalizes| relation) on the content side.

\begin{figure}[ht]\centering
\lstset{mathescape}
\begin{tabular}{|p{7.1cm}|p{7.5cm}|}\hline
  narrative & content\\\hline
\begin{lstlisting}
<definition for="?prim">
     A prime number is a natural number that 
     has only the divisors 1 and itself.
</definition>
<example name="p13" for="prim">
  <assertion xml:id="p13.st">
    13 is a prime number.
  </assertion>
  <proof for="p13" name="p13.p">
    <CMP>13 is indivisible by 2, 3, and 4
      and $\om{4^2>13}$.</CMP>
  </proof>
<example>
\end{lstlisting}
&
\begin{lstlisting}
<constant name="prim">$\ldots$</constant>
<constant name="prim13">
  <meta rel="o:formalizes" resource="?p13"/>
  <type name="prim13.t">
     <meta rel="o:formalizes" resource="?p13.st"/>
     $\om{prim(13)}$
  </type>
  <definiens name="prim13.p">
    <meta rel="o:formalizes" resource="?p13.p"/>
     $\ldots$
  </definiens>
</constant>
\end{lstlisting}\\\hline
\end{tabular}
\caption{A Simple Object-Level Example}\label{fig:simple-object}
\end{figure}

The situation in Figure~\ref{fig:simple-object} is also simple, since the both the
exemplar and the exemplified property naturally come from the same theory: natural numbers
arithmetic (which we have omitted).

\begin{wrapfigure}r{3.5cm}\centering%\vspace*{-1em}
\begin{tikzpicture}
  \node[thy] (cont) at (0,0) {\textsf{cont}};
  \node[thy] (diff) at (2,0) {\textsf{diff}};
  \node[thy] (cd) at (1,1) {\textsf{cont-diff.ex}};
  \draw[include] (cont) -- (cd);
  \draw[include] (diff) -- (cd);
\end{tikzpicture}\vspace*{-1em}
\caption{Structure}\label{fig:diffcont}\vspace*{-1em}
\end{wrapfigure}
For the next level of complexity let us consider the generic object-level
example\footnote{Nothing hinges on the genericity here, we could have made a similar point
  with specific Mersenne numbers being prime, which would be specific object-level} ``All
differentiable functions are continuous.'', where the exemplified property ``continuous''
naturally lives in the theory \textsf{cont}, but the example invokes the concept of
differentiable functions from the theory \textsf{diff} of differentiable functions. As we
do not want to include \textsf{diff} into \textsf{cont}, we have to provide a separate
theory \textsf{cont-diff.ex} for the example. This includes \textsf{diff} for the exemplar
class and \textsf{cont} for the exemplified property and we get the markup in
Figure~\ref{fig:complex-object}. Here we made use of the \lstinline|o:verbalizes| relation
-- which is the converse to \lstinline|o:formalizes| -- for statement-level parallel
markup and thus resides in the narrative representation. For the remaining examples below,
we assume one of the two representations, but omit them to avoid clutter.

\begin{figure}[ht]\centering
\begin{tabular}{|p{7.5cm}|p{6cm}|}\hline
  narrative & content\\\hline
\begin{lstlisting}
<theory name="cont">
  $\ldots$
  <example name="diffcont.ex" for="continuous">
    <meta rel="o:verbalizes" 
               resource="?cont-diff.ex"/>
     <uses from="?diff"/>
     <assertion name="cd.ex.p">
       All differentiable functions are continuous.
     </assertion>
     <proof for="#cd.ex.p">$\ldots$</proof>
  <example>
  $\ldots$
</theory>
\end{lstlisting}
&
\begin{lstlisting}
<theory name="cont-diff.ex">
  <structure from="?diff"/>
  <structure from="?cont"/>
  <constant name="cd.ex.p">
    <type>$\om{\mathcal{C}^1(\mathbb{R},\mathbb{R})\subseteq\mathcal{C}^0(\mathbb{R},\mathbb{R})}$</type>
    <definiens>$\ldots$</definiens>
  </constant>
</theory>
\end{lstlisting}\\\hline
\end{tabular}
\caption{A Generic Object-Level Example}\label{fig:complex-object}
\end{figure}
The content side in Figure~\ref{fig:complex-object} shows us what is really happening from
a structural perspective: The example object is a theory of its own, which inherits from
theories \lstinline|cont| and \lstinline|diff| as shown in Figure~\ref{fig:diffcont}. This
structure is somewhat disguised in the narrative part: the \lstinline|example| element
implicitly introduces a scope (of visibility of symbols and assumptions in effect) that
includes all from theory \lstinline|cont| it is nested in and includes the material from
theory \lstinline|diff| via the \lstinline|uses| directive.

The examples we have looked up so far are \defemph{immediate} in that they do not involve
any structural substitutions: they correspond to simple set membership or subsumption.


\section{Examples via Morphisms}\label{sec:thygraph}

We will now turn to examples that involve structural substitutions. We have already seen
above that we best employ theory structures to model examples. So we make full use of
\omdoc/MMT theory graphs~\cite{RabKoh:WSMSML13} for the content structures. We model
examples as views and make use of assignments in theory morphisms to account for
structural substitutions.

\begin{figure}[ht]\centering
\providecommand\myxscale{4.3}
\providecommand\myyscale{2.5}
\documentclass{standalone}
\usepackage{tikz,stex,amstext,amssymb}

\input{localpaths}
\begin{document}
\usetikzlibrary{shapes,arrows,mmt}
\def\thmo#1#2{\mathsf{#1}\colon\kern-.15em{#2}}
\providecommand\myxscale{3.9}
\providecommand\myyscale{2.2}
\providecommand\myfontsize{\footnotesize}
\begin{tikzpicture}[xscale=\myxscale,yscale=\myyscale]\myfontsize
\node[thy] (magma) at (1,-1.5) {\begin{tabular}{l}
                             \textsf{Magma}\\\hline
                             $G,\circ$\\\hline
                             $\scriptstyle x\circ y\in G$
                           \end{tabular}};
\node[thy] (sg) at (1.1,-.7) {\begin{tabular}{l}
                             \textsf{SemiGrp}\\\hline
                             \\\hline
                             $\scriptstyle \mathsf{assoc}: (x\circ y)\circ z=x\circ (y\circ z)$
                           \end{tabular}};
\node[thy] (m) at (1,.1) {\begin{tabular}{l}
                             \textsf{Monoid}\\\hline
                             $e$\\\hline
                             $\scriptstyle e\circ x=x$
                           \end{tabular}};
\node[thy] (g) at (1.5,1) {\begin{tabular}{l}
                             \textsf{Group}\\\hline
                             $i := \scriptstyle\lambda x.\tau y. x\circ y=e$\\\hline
                            $\scriptstyle\forall x:G.\exists y:G.x\circ y=e$
                           \end{tabular}};
\node[thy] (n) at (.5,1) {\begin{tabular}{l}
                             \textsf{NonGrpMon}\\\hline
                             \\\hline
                           $\scriptstyle\exists x:G.\forall y:G.x\circ y\ne e$
                           \end{tabular}};
\node[thy] (cg) at (1.5 ,2) {\begin{tabular}{l}
                             \textsf{CGroup}\\\hline
                             \\\hline
                            $\scriptstyle \mathsf{comm} : x\circ y=y\circ x$
                           \end{tabular}};
\node[thy] (r) at (.4,2) {\def\mr#1#2{#1\,\mathsf{m}\kern-.1em/\kern-.2em\circ\, #2}
                                       \def\ar#1#2{#1\,\mathsf{a}\kern-.1em/\kern-.2em\circ\, #2}
                            \begin{tabular}{l}
                             \textsf{Ring}\\\hline
                             \\\hline
                            $\scriptstyle \mr{x}{(\ar{y}z)}=\ar{(\mr{x}z)}{(\mr{y}z)}$\\
                            $\scriptstyle \ar{x}{(\mr{y}z)}=\rm{(\ar{x}z)}{(\ar{y}z)}$\\
                           \end{tabular}};
\node[thy] (N) at (-.7,-.9) {\begin{tabular}{l}
                             \textsf{NatNums}\\\hline
                             $\mathbb{N},s,0$\\\hline
                             $\scriptstyle P1$,\ldots $\scriptstyle P5$
                           \end{tabular}};
\node[thy] (Np) at (-.7,0) {\begin{tabular}{l}
                             \textsf{NatPlus}\\\hline
                             $+$\\\hline
                             $\scriptstyle n+0=n$,\\
                             $\scriptstyle n+s(m)=s(n+m)$
                           \end{tabular}};
\node[thy] (Nt) at (-.7,1) {\begin{tabular}{l}
                             \textsf{NatTimes}\\\hline
                             $\cdot$\\\hline
                             $\scriptstyle n\cdot1=n$,\\
                             $\scriptstyle n\cdot s(m)=n\cdot m+n$
                           \end{tabular}};
\node[thy] (ia) at (-.7,2) {\begin{tabular}{l}
                             \textsf{IntArith}\\\hline
                             $-$\\
                             $\mathbb{Z} := \mathbb{N}\cup-\mathbb{N}$\\\hline
                             $\scriptstyle-0=0$
                          \end{tabular}};
\node (phi) at (.25,-.3) {$\varphi=\left\{\begin{array}{l}
                                    G\mapsto\mathbb{N}\\
                                    \circ\mapsto \cdot\\
                                    e\mapsto 1\end{array}\right\}$};
\node (psi) at (.25,-1) {$\psi=\left\{\begin{array}{l}
                                    G\mapsto\mathbb{N}\\
                                    \circ\mapsto +\\
                                    e\mapsto 0\end{array}\right\}$};

\node (psip) at (.25,-1.55) {$\psi'=\left\{\begin{array}{l} 
                                           i\mapsto -\\
                                           \mathsf{g}\mapsto\mathsf{f}
                                         \end{array}\right\}$};
\node (theta) at (-.7,-1.55) {$\vartheta=\left\{\begin{array}{l}
                                    \mathsf{m}\mapsto\mathsf{e}\\
                                    \mathsf{a}\mapsto\mathsf{c}\end{array}\right\}$};
\draw[include] (N) -- (Np);                          
\draw[include] (Np) -- (Nt);                          
\draw[include] (Nt) -- (ia);                          
\draw[include] (magma) -- (sg);
\draw[include] (sg) -- (m);
\draw[include] (g) -- (cg);
\draw[view] (m) -- node[above] {$\thmo{e}\varphi$} (Nt);                          
\draw[view] (m) -- node[above,near end] {$\thmo{f}\psi$} (Np);                          
\draw[view] (cg) to[bend right=40] node[above] {$\thmo{d}{\psi'}$} (ia);                          
\draw[include] (m) -- node[left] (mg) {$\mathsf{g}$} (g);
\draw[view] (n) -- node[above] {$\thmo{c}\varphi$} (Nt);
\draw[include] (m) --  node[right] (mn){$\mathsf{ng}$} (n);
\draw[include] (cg) -- node[above] {$\mathsf{a}$} (r);
\draw[include] (m) to [bend right=20] node[right, near end] {$\mathsf{m}$} (r);
\draw[view] (r) -- node[above] {$\thmo{i}\vartheta$} (ia);
\draw[<->,dotted] (mg) -- (mn);
\draw[<->,dotted] (g) -- (n);
\end{tikzpicture}
\end{document}
%%% Local Variables: 
%%% mode: latex
%%% TeX-master: t
%%% End: 

\caption{A Theory Graph with Examples}\label{fig:elal}
\end{figure}

\subsection{Examples as Theory Morphisms in Theory Graphs}

For the monoids example above we will use the theory graph in Figure~\ref{fig:elal}.
There we have two examples for monoids: natural numbers with addition and multiplication,
represented by the views $\cn{f}$ and $\cn{e}$. The fact that $\cn{e}$ and
$\cn{f}$ are views means that every axiom in source theory $\cn{Monoid}$ is
provable in the respective target theories $\cn{NatPlus}$ and $\cn{NatTimes}$. 

\begin{figure}[ht]\centering
\begin{tabular}{|p{8.4cm}|p{5.7cm}|}\hline
  narrative & content\\\hline
\begin{lstlisting}
<example name="natmon.ex">
  <uses from="?NatPlus"/>
  <meta rel="o:verbalizes" resource="?natmon"/>
  <assertion name="natmon.ass">$\om{(\mathbb{N},+)}$ is a 
     <termref ref="?Monoid?monoid">monoid</termref>
     with neutral element $\om{0}$.
   </assertion>
   <proof for="natmon.ex">$\ldots$</proof>
<example>
\end{lstlisting}
&
\begin{lstlisting}
<view name="natmon" 
       from="?Monoid" to="NatPlus">
  <conass name="set">$\om{\mathbb{N}}$</conass>
  <conass name="op">$\om{+}$</conass>
  <conass name="neut">$\om{0}$</conass>
  <conass name="neutax">$\pi$</conass>
</view>
\end{lstlisting}\\\hline
\end{tabular}
\caption{A Theory-Level Example}\label{fig:theories}
\end{figure}
The narrative markup is just as in the case of simple examples from
Section~\ref{sec:immediate}, only that the content markup is in the form of a view, which
clarifies the structural substitutions. This example shows nicely that the narrative
structure does two things: it marks up the text in a mathematical document and it also
gives a view that is present in the theory graph a didactic label as ``suitable for
exemplifying the concept of a monoid'' (in a particular narrative situation). Other views
from \lstinline|monoid| might not be suitable examples -- in the situation; e.g. since the
reader does not know their targets, or if they are too trivial. For instance in
Figure~\ref{fig:elal} we have an inclusion of \lstinline|Monoid| into \lstinline|group|,
which by construction is also a theory morphism. But in an situation where groups have
just been introduced as special monoids, this example might be considered too immediate to
mention it.

\subsection{Compiling Narrative Examples into Theory Graphs}

Note that immediate examples can be ``compiled'' into a theory graph. For instance the one
in Figure~\ref{fig:simple-object} directly corresponds to the theory graph in
Figure~\ref{fig:so-tg}. Background: the theory $\cn{prime}$ introduces the predicate
$\cn{prim}$ for prime numbers based on natural number arithmetic, in this situation, we
want to give an object-level example for the prediate $\cn{prim}$, i.e. a numer $n\in\bbN$
that is prime. For this, we need to ``abstract'' the situation in theory $\cn{prime}$ into
a general situation with a predicate in $\cn{propset}$, which is extended to be non-empty
(with a witness $n$) in $\cn{neprop}$. Then we construct the push-out theory $\cn{neprim}$
on top of Figure~\ref{fig:so-tg} that instantiates $S$ with $\bbN$ and $p$ with
$\cn{prim}$. This is the natural source of an example view $\psi$ that instantiates $n$ to
$13$ and $\cn{ne}$ to the primality proof for it -- see Figure~\ref{fig:simple-object} on
the lower left.

Note that this construction is fully general the theories $\cn{propset}$ and $\cn{neprop}$
just formalize the intuition that this kind of example is concerned with showing the
non-emptiness of a given property by exhibiting a witness and proving that the property
holds on it.

\begin{figure}[ht]\centering
\begin{tikzpicture}[xscale=1 .5, yscale=1 .9]
  \node[thy] (arith) at (-1.5,0) {$\mmtthy{NatArith}{\mathbb{N},+,*,0,1,<}{}$};
  \node[thy] (prime) at (-1.5,1) {$\mmtthy{prime}{\cn{prim}:\bbN\rightarrow \bbB}{\cn{prim}(x)\Leftrightarrow \not\exists i<n,m<x.n\cdot m=x}$};
  \node[thy] (ps) at (1,0) {$\mmtthy{propset}{S:\cn{type}, p:S\rightarrow \bbB}{}$};
  \node[thy] (neprop) at (2.5,1) {$\mmtthy{neprop}{n:S}{\cn{ne}:p(n)}$};
  \node[thy] (pne) at (1,2) {$\mmtthy{neprim}{}{}$};
  \node at (-1.9,1.9) {$\begin{array}{rcl}
      \phi & :=& \{S\mapsto\bbN,p\mapsto\cn{prim}\}\\
      \psi & := & \{n\mapsto13,\cn{ne}\mapsto\cn{p13.p}\}
                      \end{array}$};
  \draw[conservative] (arith) -- (prime);
  \draw[include] (prime) to[bend right=10] (pne);
  \draw[view,red] (pne) --  node[above] {$\psi$} (prime);
  \draw[include] (ps) -- (neprop);
  \draw[view] (ps) -- node[above,near start]{$\phi$}(prime);
  \draw[view] (neprop) -- node[above,near start]{$\phi$}(pne);
  \npushout[.85]{ps}{pne};
\end{tikzpicture}
\caption{The Immediate Example from Figure~\ref{fig:simple-object} as a Theory
  Graph}\label{fig:so-tg}
\end{figure}

\subsection{Counterexamples in Theory Graphs}

The case of counterexamples is a completely different matter; naturally, the theory
morphism property forbids a direct interpretation. But if we add a new
\defemph{complementarity relation} to theory graphs (the dotted arrows in
Figure~\ref{fig:elal}; either between theories or between extension inclusions), then we
can characterize the counterexample relation as a composition of a complementarity link
with a view. In our theory graph, the natural numbers with multiplication are a
counter-example to group as evidenced by the dotted arrow (left to right) and the view
$\thmo{c}\varphi$.\ednote{We probably need concrete syntax for this. The complementarity
  relation could be a meta-relation, and we could give the counterexample an attribute
  that points to it.}

But to understand the counter-example relation, we take a step back and develop a notion
that is akin to the role the theory morphisms play for examples. Let us build our
intuition about these first. If we have a morphism $A\nmmtar{morph}\sigma B$, then we
have $B\models\sigma(A)$ proof-theoretically or
$\modelsof{\sigma(A)}\subseteq\modelsof{B}$, where $\modelsof{A}$ is the set of models of
$A$.


\begin{wrapfigure}r{4.6cm}\vspace*{-1em}
\begin{tikzpicture}[xscale=3,yscale=1.3]
  \node[thy] (ns) at (0,1) {$\cn{NatTimes}$};
  \node[thy] (m) at (1,0) {$\cn{Monoid}$};
  \node[thy] (g) at (1,1) {$\cn{Group}$};
  \draw[include] (m) -- node[right]{$\cn{j}$} (g);
  \draw[view] (m) -- node[below]{$\cn{i}\colon\sigma$} (ns);
  \draw[anti,morph] (g) -- node[above]{$\cn{k}\colon\setdiff\sigma{\cn{j}}$} (ns);
\end{tikzpicture}
\end{wrapfigure}
The situation we want to model is one of mathematical complementarity, where we want to
state that $\cn{NatTimes}$ is ``not a $\cn{Group}$'' (but a $\cn{Monoid}$). In terms of
models this means that $\modelsof{\cn{NatTimes}}\subseteq
\setdiff{\modelsof{\cn{Monoid}}}{\modelsof{\cn{Group}}}$.  In terms of theory
presentations we want to have a morphism from $\cn{group}$, such that
$\cn{NatTimes}\models\sigma(G)$, where $G$ are those declarations that are in
$\cn{Group}$, but not in $\cn{Monoid}$, i.e. those that have not been introduced by the
theory morphism $j$\footnote{In our case this is an inclusion, I guess that a structure
  would have worked just as well. I am not sure about the situation where $\cn{j}$ is a
  view, that should be explored further.}. We call such a mapping an
\defemph{antimorphism} and depict it with a stricken-through arrow and write it as
$\cn{i}\colon\setdiff\sigma{\cn{j}}$: $\cn{i}$ is the name, $\sigma$ the assignment, and
$j$ the \defemph{kernel}\ednote{MK: I do not think that kernel is a good name, but I did
  not have any better at the moment.} morphism.

In Figure~\ref{fig:elal}, we can see the complementarity relation depicted by the dotted
bidirectional arrow as a pair of antimorphisms $\cn{j}\colon\setdiff\id{\cn{g}}$ and
$\cn{k}\colon\setdiff\id{\cn{ng}}$ between $\cn{NonMonGrp}$ and $\cn{Group}$. Antimorphisms
compose with theory morphisms in the obvious ways: the set of antimorphisms is closed
under extension with morphisms by post-composition and by pre-composition of kernel
morphisms.

\subsection{Example-Only Theories}
\ednote{MK: there must be a better name for such theories}
\begin{wrapfigure}r{3.7cm}\vspace*{-1em}
  \begin{tikzpicture}
    \node[thy] (set) at (2,0) {$\cn{set}$};
    \node[thy] (f) at (1,1) {$\cn{field}$};
    \node[thy] (cs) at (3,1) {$\cn{CS}$};
    \node[thy] (csf) at (2,2) {$\cn{CSF}$};
    \node[thy] (vr) at (0,2) {$\cn{VR}$};
    \node[thy] (nat) at (3,0) {$\cn{Nat}$};
    \draw[struct] (set) -- (f);
    \draw[include] (set) -- (cs);
    \draw[include] (nat) -- (cs);
    \draw[struct] (f) -- (vr);
    \draw[include] (f) -- (csf);
    \draw[struct] (cs) -- (csf);
    \draw[view] (vr) -- node[above] {$\cn{e}:\phi$}(csf);
    \npushout[.8]{set}{csf};
  \end{tikzpicture}\vspace*{-1em}
\end{wrapfigure}
In the previous sections, we claimed that ``examples are views'' (of various kinds),
mainly because the motivating examples for examples, the target theories of the views were
always interesting in their own right. But this is not always the case, consider for
instance the case of the cartesian space $F^n$ which forms a (an example of a) vector
space if $F$ is a field. Arguably the theory of ``cartesian spaces of fields'' is not
``interesting'' in its own right -- though its instances $\mathbb{R}^n$ and $\mathbb{C}^n$
may be. We model this in the diagram on the right, where $\cn{set}$ is a base theory that
only introduces a set $S$. The theory $\cn{CS}$ introduces the Cartesian space $S^n$. On
the other hand $\cn{set}$ is used in $\cn{field}$ base the base set, and $\cn{field}$ is
in turn used as the scalar field in the theory $\cn{VR}$ of vector spaces. In this
situation, the example we have in mind can be modeled as the theory $\cn{CSF}$ together
with the view $\cn{e}$, which has the assignment
\[
\phi:=\left\{\begin{array}{rcl}
a\cdot(v_1,\ldots,v_n) & \mapsto & (a\cdot v_1,\ldots a\cdot v_n)\\
(v_1,\ldots,v_n)+(w_1,\ldots,w_n) & \mapsto & (v_1+w_1,\ldots v_n\cdot w_n)\\
0 & \mapsto & (0,\ldots,0) \\
-(v_1,\ldots,v_n) & \mapsto & (-\cdot v_1,\ldots -\cdot v_n)\\
             \end{array}\right.
\]
where $+$, $\cdot$, $0$, and $-$ on the right hand side are the operations of the field
$F$.

Note that incidentally, in this example the example target theory $\cn{CSF}$ arises as the
push-out of the theories $\cn{field}$ and $\cn{CS}$ over the base $\cn{set}$, which means
that it is indeed ``uninteresting'' since it can be computed from the two morphisms
between these three.  Given theory expression operators on theories in \mmt, (see
e.g.~\cite{CarCon:tpc12} for a proposal in the context of MathScheme), this could be
representing $\cn{CSV}$ as the ``unnamed'' theory expression
$\cn{F}\uplus_{\cn{set}}\cn{CS}$, which would consequently not be presented in active
documents except as an example.

It would be interesting to see whether all cases of ``example-only theories'' can be
expressed as theory expressions.

In \smglom we can express example-only theories by module signatures and just not give
them language bindings. That has the effect that they are not shown in narrative
documents and only serve a role as (syntactic) target theories. 

\section{Induced Examples and Practical Matters}

\begin{wrapfigure}r{5.5cm}\vspace*{-1.5em}
  \begin{tikzpicture}[yscale=.9,xscale=1.3]
    \node[thy] (monoid) at (0,3) {\textsf{monoid}};
    \node[thy] (cgp) at (0,4) {\textsf{cgp}};
    \node[thy] (field) at (0,5.5) {\textsf{field}};
    \draw[include] (monoid) -- (cgp);
    \draw[struct] (cgp) to[bend left=20] node[left] {\textsf{\footnotesize add}} (field);
    \draw[struct] (cgp) to[bend right=20] node[right] {\textsf{\footnotesize mul}} (field);

    \node[thy] (rp) at (-2,3.8) {$\mathbb{R}^+$};
    \node[thy] (rm) at (-3,4.2) {$\mathbb{R}^*$};
    \node[thy] (rpm) at (-2.5,5.5) {$\mathbb{R}^{+*}$};
    \draw[view] (cgp) to[bend left=15] node[above,near end] {\textsf{\footnotesize i}} (rp);
    \draw[view] (cgp) to[bend right=15] node[above] {\textsf{\footnotesize j}} (rm);
    \draw[include] (rp) -- (rpm);
    \draw[include] (rm) -- (rpm);
    \draw[view] (field) -- node[above] {$\scriptsize\mathsf{i}\cup\mathsf{j}$} (rpm);
    \draw[view,red] (monoid) to[bend left=25] node[above,near start] {$\scriptsize\mathsf{i}'$} (rp);
    \draw[view,red] (monoid) to[bend left=35] node[above,near end] {$\scriptsize\mathsf{j}'$} (rm);
  \end{tikzpicture}
  \caption{Induced Examples}\label{fig:induced}\vspace*{-1em}
\end{wrapfigure}
Another advantage of the theory graph realization of examples is that we can induce
examples from the represented structures. Take for instance the situation in
Figure~\ref{fig:induced}. Here we have the example morphisms \textsf{i} and \textsf{j} for
commutative groups (theory \textsf{cgp}). Composing them with the inclusion from
\textsf{monoid} to \textsf{cgp} yields the two red views $\mathsf{i}'$ and $\mathsf{j}'$
that make $\mathbb{R}^+$ and $\mathbb{R}^*$ examples for a monoid. We call such example
morphisms that are formed by composition \defemph{induced examples}. Note that induced
examples are redundant in a theory graph since they can be computed from material in the
remaining graph.

Obviously, the possibility of inducing examples suggests that examples should be specified
as ``high'' in the theory graph as possible, since that maximize induced
examples. However, this only seems to work along structures and inclusions as the
construction with fields in Figure~\ref{fig:induced}\ednote{MK: the graph does not
  work. $\mathsf{i}\cup\mathsf{j}$ is not the right graph. We seem to have R+* as the push
  out of i and j somehow; fix the graph and then argue why this precludes the top-down
  induction of i and j from the R+* example} shows.

\section{Flexiformal Theory Graphs and Narrative Examples}

In \smglom we would split the example morphism into a signature and a language binding,
where the language binding would say something like 
\begin{quote}
  $\langle\mathbb{R},+,0,-\rangle$ is a commutative group, since [proofs of the cgroup
  axioms in $\langle\mathbb{R},+,0,-\rangle$]
\end{quote}
In the purely formal setting of MMT, we can compute the composition morphisms like
$\mathsf{i}'$. But we in general cannot compute the language bindings for $\mathsf{i}'$,
which would be something like 
\begin{quote}
  $\langle\mathbb{R},+,0\rangle$ is a monoid, since [proofs of the monoid axioms in $\langle\mathbb{R},+,0\rangle$]
\end{quote}
since natural language is opaque to the theory graph algorithms. What we can however
compute is an explanation like 
\begin{quote}
  $\langle\mathbb{R},+,0,-\rangle$ is a commutative group, since [proofs of the monoid
  axioms in $\langle\mathbb{R},+,0,-\rangle$] and commutative groups are monoids by
  definition.
\end{quote}
by combining the language binding of $\mathsf{i}$ with an explanation for the inclusion
from \textsf{monoid} to \textsf{cgp}. Note that this explanation directly corresponds to
the structure of the composition that defines $\mathsf{i}'$.

\section{Conclusion}
\ednote{Say something about~\cite{Rabe:ttbor12} that mediates between the theory-level and
  object-level representations of examples.}  

We have presented an approach of modeling examples as theory morphisms in the \omdoc/\mmt
representation format for mathematical knowledge.  It turns out that \omdoc-style parallel
markup together with \mmt theory morphisms allow to adequately capture the surface
structure and the relational aspects of the meaning of examples.

We propose to extend \omdoc/\mmt with the concept of antimorphisms capture the essence of
counter-examples in a primitive that is commensurate with the design of \mmt. The details
of this proposal still have to be worked out, but the examples of counterexamples we have
studied seem to bear out the conjecture that antimorphisms are for counter-examples what
morphisms are for supporting examples. 

I believe that examples and counter-examples form an important part of the mathematical
domain knowledge that plays a great role in the formation of mathematical intuitions and
the exploration of mathematical domains. Therefore any attempt to flexiformalize
mathematical knowledge should take them seriously. In particular, universal digital
mathematical libraries like the one called for in~\cite{Farmer:mkm11} or realized
in~\cite{IanJucKoh:sdm14,MathHub:on} should collect examples and counterexamples as
first-class citizens -- which the model proposed in this note allows to do transparently
and conservatively.

In the future we want to study how the model of examples proposed here interacts with
realms~\cite{CarFarKoh:rsckmt14}, whether we can extend it to counter-examples for
conjectures, and which mathematical practices with examples can be supported in active
documents. Furthermore, there is clearly a relation between antimorphisms, the semantic
antonym relation (see~\cite{Kohlhase:dmesmgm14} for a discussion of lexical relations and
their relation with theory morphisms), and counter-examples, which we want to study in the
future.

\printbibliography
\end{document}


% LocalWords:  maketitle printbibliography mkm05 emph KohKoh tfndc11 flexiforms omdocv2
% LocalWords:  compactenum textbf wrapfigure vspace KWARCslides adp hypergraph Pease cn
% LocalWords:  KohDavGin psewads11 digm StaKoh tlcspx10 GinJucAnc alsaacl09 sfmm12 notin
% LocalWords:  NorKoh efnrsmk07 Flexiformalist inparaenum renewcommand omdoc Mathoverflow
% LocalWords:  baselinestretch tableofcontents newpage defemph ednote forall wmtapm13 mmt
% LocalWords:  Rightarrow subseteq compactitem mathsf mathbb sifemp09 thygraph KohIan rel
% LocalWords:  Lakatos Lakatos76 RabKoh elal thmo varphi tikz lstset lstlisting hlpmo13
% LocalWords:  ttbor12 mathescape hline p13.st ldots om tikzpicture cont textsf cont neut
%  LocalWords:  diffcont Mersenne mathcal mathcal providecommand myxscale myyscale natmon
%  LocalWords:  natmon.ex termref termref conass conass neutax nmmtar modelsof modelsof
%  LocalWords:  modelsof xscale yscale ns setdiff antimorphism elalg antimorphisms ng mtt
%  LocalWords:  NonMonGrp Rabe mkm11 IanJucKoh sdm14 CarFarKoh rsckmt14 monoids lstinline

%%% Local Variables:
%%% mode: latex
%%% TeX-master: t
%%% End:
%  LocalWords:  emphasized centering so-tg arith ool cdot tteen characterize csf npushout
%  LocalWords:  rcl mapsto tpc12 uplus cgp cgp rp realization Flexiformal langle cgroup
%  LocalWords:  flexiformalize realized dmesmgm14 propset neprop neprim mmtthy pne smglom
