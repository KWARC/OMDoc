\documentclass[12pt]{article}
\usepackage{a4wide,url}
\usepackage[hyper]{acronyms}
\usepackage{lstomdoc,xmlindex}
\usepackage[show]{ed}
\usepackage{hyperref}
\lstset{language=[1.6]OMDoc,basicstyle=\scriptsize}
\def\llquote#1{\ensuremath{\langle\kern-.25em\langle\hbox{\sl{#1}}\rangle\kern-.25em\rangle}}

\title{MMT URIs for OMDoc 1.6}
\author{Michael Kohlhase}

\begin{document}
\maketitle
\begin{abstract}
  We propose a new URI scheme tailored to referencing semantic objects in the {\omdocv2}
  format and discuss implementation issues.
\end{abstract}

\section{Introduction}

Uniform Resource Identifiers (URI~\cite{BerFie:uri98}\footnote{Actually --- following
  general W3C practice --- we use IRIs (Internationalized Resource
  Identifiers~\cite{DueSui:iri05}) which extend the URI character set to almost all of
  {\unicode} throughout this note without explicitly saying so. IRIs us to use more
  sensible names in {\omdoc}. The only processing overhead involved in this is that IRIs
  must be URI-encoded before being sent over the wire.}) are sometimes called the
``plumbing of the web'', and indeed, the ability to identify and retrieve resources via a
universal reference is one of the key innovations that made the world wide web
possible. This is even more true for the ``Semantic Web''~\cite{Berners-Lee:SemanticWeb},
where relations between resources are made explicit. So it is no surprise that the
foundations RDF~\cite{manola04:rdfprimer} and OWL~\cite{McGvHa:owl04} use URIs as
identifiers of semantic objects.

The main utility of URIs comes from the the fact that they can be used to locate resources
deployed by a large set of methods, ranging from static files on web servers to virtual
resources that are created by web services upon requests: static files on web servers
lower the entry barrier for authors participating in web content creation and web services
allow for much of the richness of the data web environment.

In this note we propose a new URI scheme tailored to referencing semantic objects in the
{\omdocv2} format and discuss implementation issues.

\section{Reference in {\omdocv1}}

In {\omdocv{1.2}}~\cite{Kohlhase:omdoc1.2}, we also make use of URIs for identifying
objects, e.g. in the {\attribute{for}{imports}} attribute in {\element{imports}}
elements. Implicitly we restricted the use of URIs in {\omdocv{1.2}} to URLs (Uniform
Resource Locators; i.e. they actually locate a resource that can be retrieved); basically
restricting ourselves to the {\texttt{html}} web, where document fragments are identified
by {\attributeshort[ns-attr=xml]{id}} attributes and identified by fragment identifiers of
the form {\texttt{\#foo}}.  In {\omdocv{1.1}} we even had special URI fragment identifiers
the form {\texttt{\#byctx(cd,name)}} that allowed to reference --- i.e. identify; not
necessarily locate --- any {\omdoc} element that has a {\attributeshort{name}} attribute
(basically only the {\element{symbol}}). Reference ``by context'' is attractive, since
\begin{itemize}
\item the uniqueness conditions on {\attributeshort{name}} attributes are less severe than
  those of {\attributeshort[ns-attr=xml]{id}} attributes which must be document-unique,
  leading to more composable documents
\item the referencing scheme is more semantic and thus less brittle against moving
  document fragments around.
\end{itemize}
But as ``reference by context'' this was slightly under-defined, of limited utility, and
difficult to implement, this proposal was dropped in {\omdocv{1.2}}.

\section{Semantic Reference in {\omdocv2}} 

For {\omdocv{2}}, we are fundamentally rethinking how we want to reference semantic
objects in {\omdoc}. In particular, in the MMT module system~\cite{RabKoh:WSMSML08} --- a
new and improved version of the development graph model~\cite{MosAutHut:edgwh01} that
underlies {\omdocv{1.2}} --- we have extended the objects that have
{\attributeshort{name}} attributes to all semantic objects, including theory
morphisms. This leads to totally new application of semantic reference. We can identify
{\omdoc} document fragments that are not explicitly represented in the {\omdoc} document
(collection), but only virtually induced by the surface representation as part of the
``flattened document (collection)''. Note that flattening can lead to an exponential
increase in size of the document, conversely MMT modularization can lead to extended
theory reuse and to much more concise documents. Moreover, non-explicit versions of this
technique have been used extensively in developments of mathematics like the Bourbaki
collection~\cite{Bourbaki:a68,Bourbaki:a74}; see~\cite{Laubner:utgmm07} for a discussion.

The ability to reference the whole flattened document collection considerably increases
the utility of semantic referencing in {\omdoc}, but also highlights the implementation
problems. I claim that the the latter are aggravated by our attempts to get by with
Uniform Resource {\emph{Locators}} rather than fully embracing the the idea to use URIs to
{\emph{identify resources}} and leave the retrieval of the resources to another part of
the information architecture.

\section{A new Resource Identification Scheme or {\omdoc}}

I propose a new URI scheme for identifying semantic objects in {\omdocv2}, based on the
MMT system, which will be a central organizing principle in {\omdocv2}. The MMT system
uses triples $\langle g,t,n\rangle$ for referencing, where $g$ is the URI of a document,
$t$ is a theory path, and $n$ is a complex name. These correspond to, but greatly extend
the {\openmath}~\cite{BusCapCar:2oms04} referencing triple given by the
{\attribute{cdbase}{OMS}}\ednote{There is still a mismatch here between the use of
  {\tt{cdbase}} in {\openmath} and the MMT usage. This should be taken care of before
  deploying.}, {\attribute{cd}{OMS}}, and {\attribute{name}{OMS}} attributes on
{\element{OMS}} elements. {\cite{RabKoh:WSMSML08}} also suggests a scheme for encoding
these as regular URIs, but the scheme has the serious drawback that the MMT triple cannot
be parsed back from URIs. BTW, the URI encoding of {\openmath} triples currently proposed
in the MathML3 specification has the same problem.

The new URI scheme for a MMT triple $\langle g,t,n\rangle$ is
{\texttt{mmt-$\llquote{g}$?cd=$\llquote{t}$;name=$\llquote{n}$}} for instance, the MMT URI
for the symbol {\texttt{plus}} in the {\texttt{arith1}} content dictionary would be 

\begin{verbatim}
  mmt-http://cds.omdoc.org/omstd/arith1?cd=arith1;name=plus
\end{verbatim}

which is still quite manageable to write by hand. The beauty of this approach which puts
the semantic information into the query part of a URI is that we can add other
specification items to it. Most importantly for us: version information, e.e. 

\begin{verbatim}
  mmt-http://cds.omdoc.org/omstd/arith1?cd=arith1;name=plus;revision=4711
\end{verbatim}

Of course, we can have relative URIs as well: if $g$ (the ``cdbase'') is known, then we
just have {\texttt{mmt:?cd=arith1;name=plus}}. We might even go the extra mile and define
{\texttt{mmt:plus@arith1}} as a synonym of the above.

\section{Retrieval of MMT URIs}

A great problem of using URIs is that it becomes totally unclear how to retrieve them. For
some URI schemes, this can be solved by a simple cataloging service. In the {\omdoc}
context this is much more difficult, since referencing the flattened collection requires
non-trivial transformations. On the web services side, this functionality can be offered
by the {\jomdoc} library~\cite{JOMDoc:webpage} or --- based on that --- by the {\ombase}
system~\cite{OMBase:webpage}. But for a low entry barrier into {\omdoc} publication, we
also want to accommodate for a way to deploy static {\omdoc} documents on simple web
servers. On the other hand to achieve a low entry barrier for application developers mmt
URIs should be easy to handle in applications like {\xslt} style sheets.

The main idea to achieve this is to externalize the flattening from the retrieval process
which is relegated to regular retrieval methods. In the simplest case, we have a MMT web
service MMTWS running at \url{http://mmt.omdoc.org/mmtws}. Then we can process the MMT URL
above by rewriting it to

\begin{verbatim}
  http://mmt.omdoc.org/mmtws?
     doc=http://cds.omdoc.org/omstd/arith1;cd=arith1;name=plus
\end{verbatim}

by dropping the {\texttt{mmt}} prefix and moving the URI part before the query into the
query\ednote{I am not sure here, do we need to URI-encode the URI here? I think now}. This
URI the MMTWS processes by loading \url{http://cds.omdoc.org/omstd/arith1.omdoc},
flattening the theory {\texttt{arith1}} --- possibly loading imported theories, and
returning the element with name {\texttt{plus}} from the flattened theory as an {\omdoc}
fragment\ednote{we need to specify how to transport {\omdoc} fragments over the wire, so
  that they remain contextualized}. The beauty of this setup is that
\begin{itemize}
\item the rewriting step is the only think that needs to be done by the application
  developer. Any URI library or just raw string processing should this a relatively simple
  task.
\item {\omdoc} authors can still deploy {\omdoc} documents or collections as static files
  on web servers without losing MMT modularity.
\end{itemize}
Thus it satisfies the ``low entry barrier'' requirement stated above.

\section{Building MMTWS with the KWARC Toolkit}

The MMTWS should be a relatively simple extension of the {\jomdoc} library once it can do
theory flattening. Probably Florian already has a {\scala} implementation that is close to
working for this, and he has been playing with the {\scala} web services toolkit I hear. 

The MMTWS functionality could be enhanced in two ways in the future. The first idea is
that we can just export a {\texttt{mmtget}} function from the {\jomdoc} library that can
be directly called from say the {\texttt{saxon}} {\xslt} processor. This would probably be
a good interface function for building MMTWS anyway. 

The MMTWS web service could be based on {\ombase} instead of just {\jomdoc}, then it would
have a caching facility and the web service would not have to load the relevant files over
and over again. Assuming that most of the URIs used in the {\omdoc} documents are MMT URIs
that are hosted by {\ombase} knowledge base services, most of the load operations will be
{\svn} update operations that only move diffs over the wire. I can see this as the
beginning of a very effective and attractive {\ombase} cache network and I imagine always
running a MMTWS/{\ombase} on my laptop, which would make me relatively independent of the
network when I write {\omdoc} while traveling. 

\section{Caveats and Text Roadmap}

This note is still in a very early stage, and is intended mainly as a vehicle or
discussion between {\omdoc} developers. In particular, the URI scheme and all names are
provisional and will probably evolve over time. At a later and more mature stage, part of
the text might go into~\cite{RabKoh:WSMSML08} and/or the {\omdocv{1.6}} specification.

\section{Acknowledgements}
This proposal has been greatly influenced by the HTTP Getter utility~\cite{getter:webpage}
developed by Claudio Sacerdoti Coen and Stefano Zacchiroli for the HELM
project~\cite{AspPad:hsmw01} at Bologna. Though I have been collaborating with them for
years, I have never quite understood the relevance of the HTTP Getter until now.

\bibliographystyle{alphahurl} 
\bibliography{kwarc}
\end{document}

% LocalWords:  ns attr xml byctx cd MMT cdbase mmt arith MMTWS mmtget saxon
% LocalWords:  Sacerdoti Coen Zacchiroli kwarc

%%% Local Variables:
%%% mode: latex
%%% TeX-master: "vi"
%%% End:
