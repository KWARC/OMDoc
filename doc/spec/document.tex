%%%%%%%%%%%%%%%%%%%%%%%%%%%%%%%%%%%%%%%%%%%%%%%%%%%%%%%%%%%%%%%%%%%%%%%%%
% This file is part of the LaTeX sources of the OMDoc 1.6 specification
% Copyright (c) 2006 Michael Kohlhase
% This work is licensed by the Creative Commons Share-Alike license
% see http://creativecommons.org/licenses/by-sa/2.5/ for details
% The source original is at https://github.com/KWARC/OMDoc/doc/spec 
%%%%%%%%%%%%%%%%%%%%%%%%%%%%%%%%%%%%%%%%%%%%%%%%%%%%%%%%%%%%%%%%%%%%%%%%%

\begin{omgroup}[id=omdoc-infrastructure,short=Document Infrastructure]
                          {Document Infrastructure (Module {\DOCmodule{spec}})}

\begin{module}[id=document-types]
Mathematical knowledge is largely communicated by way of a specialized set of documents
(e.g. e-mails, letters, pre-prints, journal articles, and textbooks).  These employ
special notational conventions and visual representations to convey the mathematical
knowledge reliably and efficiently.

When marking up mathematical knowledge, one always has the choice whether to mark up the
structure of the document itself, or the structure of the mathematical knowledge that is
conveyed in the document. Even though in most documents, the document structure is
designed to help convey the structure of the knowledge, the two structures need not be the
same.  To frame the discussion we will distinguish two aspects of mathematical
documents. In the {\emph{\twintoo{knowledge-centered}{view}}} we organize the mathematical
knowledge by its function, and do not care about a way to present it to human
recipients. In the {\emph{\twintoo{narrative-centered}{view}}} we are interested in the
structure of the argument that is used to convey the mathematical knowledge to a human
user.

\begin{definition}[display=flow,id=knowledge-structured.def]
  We will call a document {\defin{knowledge-structured}} and
  {\defin{narrative-structured}}, based on which of the two aspects is prevalent in the
  organization of the material.  Narrative-structured documents in mathematics are
  generally directed at human consumption (even without being in presentation
  markup). They have a general narrative structure: text interleaving with formal elements
  like assertions, proofs, \ldots Generally, the order of presentation plays a role in
  their effectiveness as a means of communication.  Typical examples of this class are
  course materials or introductory textbooks.  Knowledge-structured documents are
  generally directed at machine consumption or for referencing. They do not have a linear
  narrative spine and can be accessed randomly and even re-ordered without information
  loss.  Typical examples of these are formula collections, {\openmath} content
  dictionaries, technical specifications, etc.
\end{definition}

The distinction between knowledge-structured and narrative-structured documents is
reminiscent of the {\indextoo{presentation}} vs.  {\indextoo{content}} distinction
discussed in {\extref{book}{math-objects}}, but now it is on the level of document
structure.  Note that mathematical documents are often in both categories: a mathematical
textbook can be read from front to end, but it can also be used as a reference, accessing
it by the index and the table of contents.  The way humans work with knowledge also
involves a change of state. When we are taught or explore a mathematical domain, we have a
linear/narrative path through the material, from which we abstract more and more, finally
settling for a semantic representation that is relatively independent from the path we
acquired it by.  Systems like {\activemath} (see {\extref{projects}{activemath}}) use the {\omdoc}
format in exactly that way playing on the difference between the two classes and
generating narrative-structured representations from knowledge-structured ones on the fly.

So, maybe the best way to think about this is that the question whether a document
is narrative- or knowledge-structured is not a property of the document itself,
but a property of the application processing this document.

{\omdoc} provides markup infrastructure for both aspects. In this chapter, we will discuss
the infrastructure for the narrative aspect --- for a working example we refer the reader
to {\extref{primer}{courseware}}. We will look at markup elements for knowledge-structured
documents in {\sref{theories-contexts}}.

Even though the infrastructure for narrative aspects of mathematical documents is somewhat
presentation-oriented, we will concentrate on content-markup for it. In particular, we
will not concern ourselves with questions like font families, sizes, alignment, or
positioning of text fragments. Like in most other {\xml} applications, this kind of
information can be specified in the {\css} {\attributeshort{style}} and
{\attributeshort{class}} attributes described in {\sref{common-attribs}}.
\end{module}

\begin{omgroup}[id=root]{The Document Root}
\begin{module}[id=omdoc-root]

\begin{definition}[id=omdoc.def]
  The {\xml} root element of the {\omdoc} format is the {\eldef{omdoc}} element, it
  contains all other elements described here.  We call an {\omdoc} element a
  {\twindef{top-level}{element}}, if it can appear as a direct child of the
  {\element{omdoc}} element.  The {\element{omdoc}} element has an optional attribute
  {\attribute[ns-attr=xml]{id}{omdoc}} that can be used to reference the whole
  document. The optional attribute {\attribute{version}{omdoc}} is used to specify the
  version of the {\omdoc} format the contens of the element conforms to.  It is fixed to
  the string {\snippet{1.6}} by this specification. This will prevent validation with a
  different version of the {\indextoo{DTD}} or {\indextoo{schema}}, or processing with an
  application using a different version of the {\omdoc} specification. The (optional)
  attribute {\attributeshort{modules}} allows to specify the {\omdoc} modules that are
  used in this element. The value of this attribute is a whitespace-separated list of
  module identifiers (e.g. {\MOBJmodule{spec}} the left column in
  {\myfigref{omdoc-modules}}), {\omdoc} sub-language identifiers (see
  {\myfigref{omdoc-sub-languages}}), or {\twintoo{URI}{reference}s} for externally given
  {\omdoc} modules or sub-language identifiers.\footnote{Allowing these external module
    references keeps the {\omdoc} format extensible. Like in the case with
    {\twintoo{namespace}{URI}s} {\omdoc} do not mandate that these
    {\twintoo{URI}{reference}s} reference an actual resource. They merely act as
    identifiers for the modules.} The intention is that if present, the
  {\attributeshort{modules}} specifies the list of all the modules used in the document
  (fragment).  If a {\attributeshort{modules}} attribute is present, then it is an error,
  if the content of this element contains elements from a module that is not specified;
  spurious module declarations in the {\attribute{modules}{omdoc}} attributes are allowed.
\end{definition}

Here and in the following we will use tables as the one in {\myfigref{qtgeneral}} to give
an overview over the respective {\omdoc} elements described in a chapter or section. The
first column gives the element name, the second and third columns specify the required and
optional attributes. We will use the fourth column labeled ``MD'' to indicate whether an
{\omdoc} element can have a {\element{metadata}} child (see {\sref{eldef.metadata}}), which
will be described in the next section. Finally the fifth column describes the content
model --- i.e. the allowable children --- of the element.  For this, we will use a form of
\twintoo{Backus Naur form}{notation} also used in the DTD: {\snippet{\#PCDATA}} stands for
``{\twintoo{parsed}{character data}}'', i.e. text intermixed with legal {\omdoc}
elements.) A synopsis of all elements is provided in {\sref{table}}.

\begin{presonly}
\begin{myfig}{qtgeneral}{{\omdoc} Elements for Specifying  Document Structure.}
\begin{scriptsize}
\begin{tabular}{|>{\tt}l|>{\tt}p{1truecm}|>{\tt}p{4truecm}|c|>{\tt}p{3.1truecm}|}\hline
{\rm Element}& \multicolumn{2}{l|}{Attributes\hspace*{2.25cm}} & M & Content  \\\hline
             & {\rm Required}  & {\rm Optional}      & D  &           \\\hline\hline
 omdoc       &  & xml:id, type, class, style, version, modules
                                                     & +  & (\llquote{top-level})* \\\hline
 ref         & xref & type, class, style             & -- &     \\\hline
 ignore      &      & type, comment                  & -- & ANY\\\hline
 \multicolumn{5}{|p{11cm}|}{where \llquote{top-level} stands for top-level {\omdoc} elements}\\\hline
\end{tabular}
\end{scriptsize}
\end{myfig}
\end{presonly}
\end{module}
\end{omgroup}

\begin{omgroup}[id=docmetadata]{Metadata}
\begin{module}[id=metadata]

The {\twintoo{World Wide}{Web}} was originally built for human consumption, and although
everything on it is machine-readable, most of it is not machine-understandable.  The
accepted solution is to provide {\indextoo{metadata}} ({\twintoo{data}{about data}}) to
describe the documents on the web in a machine-understandable format that can be processed
automatically. Metadata commonly specifies aspects of a document like title, authorship,
language usage, and administrative aspects like modification dates, distribution rights,
and identifiers.
  
In general, metadata can either be embedded in the respective document, or be stated in a
separate one. The first facilitates maintenance and control (metadata is always at your
fingertips, and it can only be manipulated by the document's authors), the second one
enables inference and distribution. {\omdoc} allows to embed metadata into the document,
from where it can be harvested for external metadata formats, such as the {\xml}
{\indextoo{resource description format}} ({\rdf}~\cite{LasSwi:rdf99}).  We use one of the
best-known metadata schemata for documents -- the {\emph{\indextoo{Dublin Core}}} (cf.
{\srefs{dc-elements}{dc-roles}}). The purpose of annotating metadata in {\omdoc} is to
facilitate the administration of documents, e.g.  digital {\twintoo{rights}{management}},
and to generate input for metadata-based tools, e.g.  RDF-based navigation and indexing of
document collections. Unlike most other document formats {\omdoc} allows to add metadata
at many levels, also making use of the metadata for document-internal markup purposes to
ensure consistency.
  
\begin{definition}[id=metadata.def]
  The {\eldef{metadata}} element contains elements for various metadata formats including
  bibliographic data from the Dublin Core vocabulary (as mentioned above), licensing
  information from the {\twintoo{Creative Commons}{Initiative}} (see
  {\sref{creativecommons}}), as well as information for {\openmath} content dictionary
  management. Application-specific metadata elements can be specified by adding
  corresponding {\omdoc} modules that extend the content model of the {\element{metadata}}
  element.
\end{definition}

The {\omdoc} {\element{metadata}} element can be used to provide information about the
document as a whole (as the first child of the {\element{omdoc}} element), as well as
about specific fragments of the document, and even about the top-level mathematical
elements in {\omdoc}. This reinterpretation of bibliographic metadata as general data
about knowledge items allows us to extract document fragments and re-assemble them to new
aggregates without losing information about authorship, source, etc.
\end{module}
\end{omgroup}

\begin{omgroup}[id=comment]{Document Comments}
\begin{module}[id=comments]

Many content markup formats rely on {\indextoo{comment}ing} the source for human
understanding; in fact {\twintoo{source}{comment}s} are considered a vital part of
document markup. However, as {\xml} \twinalt{comments}{XML}{comment} (i.e. anything
between ``{\snippetin{<\char33--}}'' and ``{\snippetin{-->}}'' in a document) need not
even be read by some {\xml} \twinalt{parsers}{XML}{parser}, we cannot guarantee that they
will survive any {\xml} manipulation of the {\omdoc} source.

Therefore, anything that would normally go into comments should be modeled with an
{\element{omtext}} element ({\attribute{type}{omtext}} {\attval{comment}{type}{omtext}},
if it is a text-level comment; see {\sref{omtext}}) or with the {\element{ignore}}
element for {\twintoo{persistent}{comment}s}, i.e.  comments that survive processing. 

\begin{definition}[id=ignore.def]
  The content of the {\eldef{ignore}} element can be any well-formed {\omdoc}, it can
  occur as an {\omdoc} top-level element or inside mathematical texts (see
  {\sref{mtext}}).
\end{definition}
This element should be used if the author wants to comment the {\omdoc} representation,
but the end user should not see their content in a final presentation of the document, so
that {\omdoc} text elements are not suitable, e.g. in

\begin{lstlisting}[numbers=none,index={ignore},mathescape]
<ignore type="todo" comment="this does not make sense yet, rework">
  <assertion xml:id="heureka">$\ldots$</assertion>
</ignore>
\end{lstlisting}

Of course, {\element{ignore}} elements can be nested, e.g. if we want to mark up
the comment text (a pure string as used in the example above is not enough to
express the mathematics). This might lead to markup like 

\begin{lstlisting}[label=nested-ignore,numbers=none,index={ignore},mathescape]
<ignore type="todo" comment="rework">
  <ignore type="todo-comment">
    <h:p>This does not make sense yet, in particular, the equation 
      $\psom\ldots$ cannot be true, think of $\psom\ldots$
    </h:p>
  </ignore>
  <assertion xml:id="heureka">$\ldots$</assertion>
</ignore>
\end{lstlisting}

\begin{oldpart}{MK: cannot make this example any more, think of a new one}
\begin{example}
  Another good use of the {\element{ignore}} element is to use it as an analogon to the
  {\atwintoo{in-place}{error}{markup}} in {\openmath} objects (see {\sref{om.error}}). In
  this case, we use the {\attribute{type}{ignore}} attribute to specify the kind of error
  and the content for the faulty {\omdoc} fragment. Note that since the whole object must
  be {\xml} valid. As a consequence, the {\element{ignore}} element can only be used for
  ``{\twintoo{mathematical}{error}s}'' like sibling {\element{CMP}} or {\element{FMP}}
  elements that do not have the same meaning as in
  {\mylstref{ignore-error}}. {\xml}-well-formedness and validity errors will have to be
  handled by the {\xml} tools involved.

\begin{lstlisting}[label=lst:ignore-error,
  caption={Marking up Mathematical Errors Using {\element{ignore}}},
  numbers=none,index={ignore}]
  <ignore type="CMP-lang-error" 
          comment="multilingual CMPs are not translations of each other">
    <assertion xml:id="ass1">
      <CMP>The proof is trivial</CMP>
      <CMP xml:lang="de">Der Beweis ist extrem schwer</CMP>
    </assertion>
  </ignore>
\end{lstlisting}    
For another use of the {\element{ignore}} element, see {\myfigref{flatten}} in
{\sref{sharing}}.
\end{example}
\end{oldpart}
\end{module}
\end{omgroup}

\begin{omgroup}[id=sectioning]{Document Structure}
\begin{module}[id=sectioning]

Like other documents mathematical ones are often divided into units like chapters,
sections, and paragraphs by tags and nesting information. {\omdoc} makes these document
relations explicit by using the {\element{omdoc}} element with an optional attribute
{\attribute{type}{omdoc}}. It can take the values
\begin{description}
\item[\attval{sequence}{type}{omdoc}] for a succession of paragraphs. This is the
  default, and the normal way narrative texts are built up from paragraphs, mathematical
  statements, figures, etc. Thus, if no {\attribute{type}{omdoc}} is given the type
  {\attval{sequence}{type}{omdoc}} is assumed.
\item[\attval{itemize}{type}{omdoc}] for unordered lists. The children of this type of
  {\element{omdoc}} will usually be presented to the user as indented paragraphs
  preceded by a {\twintoo{bullet}{symbol}}. Since the choice of this symbol is purely
  presentational, {\omdoc} use the {\css} {\attributeshort{style}} or
  {\attributeshort{class}} \twinalt{attributes}{CSS}{attribute} on the children to specify the
  presentation of the bullet symbols (see {\sref{common-attribs}}).
\item[\attval{enumeration}{type}{omdoc}] for ordered lists. The children of this
  type of {\element{omdoc}} are usually presented like unordered lists, only
  that they are preceded by a running number of some kind (e.g. ``1.'', ``2.''\ldots
  or ``a)'', ``b)''\ldots; again the
  {\attributeshort{style}} or {\attributeshort{class}} attributes apply).
\item[\attval{sectioning}{type}{omdoc}] The children of this type of
  {\element{omdoc}} will be interpreted as sections. This means that the
  children will be usually numbered hierarchically, and their metadata will be
  interpreted as section heading information. For instance the
  {\element{metadata}}/{\element[ns-elt=dc]{title}} information (see {\sref{dc-elements}}
  for details) will be used as the section title. Note that {\omdoc} does not
  provide direct markup for particular hierarchical levels like
  ``{\indextoo{chapter}}'', ``{\indextoo{section}}'', or
  ``{\indextoo{paragraph}}'', but assumes that these are determined by the
  application that presents the content to the human or specified using the {\css}
  attributes.
\end{description}
Other values for the {\attributeshort{type}} attribute are also admissible, they should be
{\twintoo{URI}{reference}s} to documents explaining their intension.
  
We consider the {\element{omdoc}} element as an implicit {\element{omdoc}}, in order to
allow plugging together the content of different {\omdoc} documents as
{\element{omdoc}s} in a larger document. Therefore, all the attributes of the
{\element{omdoc}} element also appear on {\element{omdoc}} elements and behave exactly
like those.
\end{module}
\end{omgroup}

\begin{module}[id=sharing]
\begin{omgroup}[id=sharing,short=Sharing Document Parts] 
{Sharing and Referring to Document Parts}

\begin{definition}[id=ref.def]
  As the document structure need not be a tree in hypertext documents, {\element{omdoc}}
  elements also allow empty {\eldef{ref}} elements whose {\attribute{xref}{ref}} attribute
  can be used to reference {\omdoc} elements defined elsewhere. The optional
  {\attribute[ns-attr=xml]{id}{ref}} (its value must be document-unique) attribute
  identifies it and can be used for building reference labels for the included parts. Even
  though this attribute is optional, it is highly recommended to supply it.  The
  {\attribute{type}{ref}} attribute can be used to describe the reference type.  Currently
  {\omdoc} supports two values: {\attval{include}{type}{ref}} (the default) for in-text
  replacement and {\attval{cite}{type}{ref}} for a proper reference. The first kind of
  reference requires the {\omdoc} application to process the document as if the
  {\element{ref}} element were replaced with the {\omdoc} fragment specified in the
  {\attribute{xref}{ref}}. The processing of the type {\attval{cite}{type}{ref}} is
  application specific. It is recommended to generate an appropriate label and
  (optionally) supply a hyper-reference.  There may be more supported values for
  {\attribute{type}{ref}} in time.
\end{definition}

\begin{oldpart}{reconsider this, we may change the {\element{omgroup}} element to
    {\element{omdoc}}}
  Let $R$ be a {\element{ref}} element of type {\attval{include}{type}{ref}}.
  \begin{definition}[display=flow,id=ref-target.def]
    We call the element the URI in the {\attribute{xref}{ref}} points to its
    {\defin[ref-target]{target}} unless it is an {\element{omdoc}} element; in this case,
    the target is an {\element{omdoc}} element which has the same children as the original
    {\element{omdoc}} element\footnote{This transformation is necessary, since {\omdoc}
      does not allow to nest {\element{omdoc}} elements, which would be the case if we
      allowed verbatim replacement for {\element{omdoc}} elements. As we have stated
      above, the {\element{omdoc}} has an implicit {\element{omdoc}} element, and thus
      behaves like one.}.  We call the process of replacing a {\element{ref}} element by
    its target in a document {\twindef{ref}{reduction}} and the document resulting from
    the process of systematically and recursively reducing all the {\element{ref}}
    elements the {\twindef{ref-normal}{form}} of the source document. Note that
    {\element{ref}}-normalization may not always be possible, e.g.  if the
    {\twinref{ref}{target}s} do not exist or are inaccessible --- or worse yet, if the
    relation given by the {\element{ref}} elements is {\indextoo{cyclic}}. Moreover, even
    if it is possible to {\element{ref}}-normalize, this may not lead to a valid {\omdoc}
    document, e.g.  since \twinalt{\snippet{ID} type}{ID}{type} attributes that were
    unique in the target documents are no longer in the {\element{ref}}-reduced one. We
    will call a document {\twindef{ref}{reducible}}, iff its {\element{ref}}-normal form
    exists, and {\twindef{ref}{valid}}, iff the {\element{ref}} normal form exists and is
    a valid {\omdoc} document.
  \end{definition}
\end{oldpart}
  
Note that it may make sense to use documents that are not {\element{ref}}-valid for
{\indextoo{narrative-centered}} documents, such as courseware or slides for talks that
only allude to, but do not fully specify the knowledge structure of the mathematical
knowledge involved. For instance the slides discussed in {\extref{primer}{coursewmare.narrative-structured}}
do not contain the {\element{theory}} elements that would be needed to make the documents
{\element{ref}}-valid.

The {\element{ref}} elements also allow to ``{\indextoo{flatten}}'' the tree
structure in a document into a list of leaves and relation declarations (see
{\myfigref{flatten}} for an example). It also makes it possible to have more than
one view on a document using {\element{omdoc}} structures that reference a
shared set of {\omdoc} elements. Note that we have embedded the
{\element{ref}}-targets of the top-level {\element{omdoc}} element into an
{\element{ignore}} comment, so that an {\omdoc} transformation (e.g. to text form)
does not encounter the same content twice.

\begin{myfig}{flatten}{Flattening a Tree Structure}
\lstset{mathescape,frame=none,numbers=none,index={omdoc,omtext}}
\begin{tabular}{|p{5.1cm}|p{5.5cm}|}\hline
\begin{lstlisting}
<omdoc xml:id="text" 
         type="sequence">
  <omtext xml:id="t1">$T_1$</omtext>
  <omdoc xml:id="enum" 
            type="enumeration">
    <omtext xml:id="t2">$T_2$</omtext>
    <omtext xml:id="t3">$T_3$</omtext>
  </omdoc>
  <omtext xml:id="t4">$T_4$</omtext>
</omdoc>
\end{lstlisting}
& 
\begin{lstlisting}
<omdoc xml:id="text" type="sequence">
  <ref xref="#t1"/>
  <ref xref="#enum"/>
  <ref xref="#t4"/>
</omdoc>

<ignore type="targets"
        comment="already referenced"> 
  <omtext xml:id="t1">$T_1$</omtext>
  <omtext xml:id="t2">$T_2$</omtext>
  <omtext xml:id="t3">$T_3$</omtext>
  <omtext xml:id="t4">$T_4$</omtext>

  <omdoc xml:id="enum" 
           type="enumeration">
    <ref xref="#t2"/>
    <ref xref="#t3"/>
  </omdoc>
</ignore>
\end{lstlisting}\\\hline
\end{tabular}
\end{myfig}

While the {\omdoc} approach to specifying document structure is a much more flexible
(database-like) approach to representing structured documents\footnote{The simple tree
  model is sufficient for simple markup of existing mathematical texts and to replay them
  verbatim in a browser, but is insufficient e.g. for generating individualized
  presentations at multiple levels of abstractions from the representation. The {\omdoc}
  text model --- if taken to its extreme --- allows to specify the respective role and
  contributions of smaller text units, even down to the sub-sentence level, and to make
  the structure of mathematical texts machine-understandable. Thus, an advanced
  presentation engine like the {\activemath} system~\cite{SieBen:acgap00} can --- for
  instance --- extract document fragments based on the preferences of the respective
  user.}  than the tree model, it puts a much heavier load on a system for presenting the
text to humans. In essence the presentation system must be able to recover the left
representation from the right one in {\myfigref{flatten}}.  Generally, any {\omdoc}
element defines a fragment of the {\omdoc} it is contained in: everything between the
start and end tags and (recursively) those elements that are reached from it by following
the {\indextoo{cross-reference}s} specified in {\element{ref}} elements.  In particular,
the text fragment corresponding to the element with
{\attribute[ns-attr=xml]{id}{omtext}}={\snippet{"text"}} in the right {\omdoc}
of~\myfigref{flatten} is just the one on the left.\ednote{make the model of first
  normalizing and then presentation and what this entais. cf. the TRAC issues.}

In {\sref{common-attribs}} we have introduced the {\css}
\twinalt{attributes}{CSS}{attribute} {\attribute{style}{ref}} and {\attribute{class}{ref}},
which are present on all {\omdoc} elements. In the case of the {\element{ref}} element,
there is a problem, since the content of these can be incompatible. In general, the rule
for determining the style information for an element is that we treat the replacement
element as if it were a child of the {\element{ref}} element, and then determine the
values of the {\css} \twinalt{properties}{CSS}{property} of the {\element{ref}} element by
inheritance.
\end{omgroup}
\end{module}

\begin{module}[id=docalt]
\begin{omgroup}{Abstract Documents}

\begin{definition}[id=docalt.def]
  To be able to support {\twintoo{abstract}{documents}} that can be concretized, {\omdoc}
  supplies the {\eldef{docalt}}\ednote{name is provisional, Christine will develop this
    more fully} element that groups alternative document fragments so that the
  presentation process can chose among them.
 \end{definition}

\begin{example}
  One very simple example is to group language variants\footnote{i.e. all the document
    fragments in this group are direct translations of each other.} using the optional
  {\attributeshort[ns-attr=xml]{lang}} attribute to specify the language they are written
  in. Conforming with the {\xml} recommendation, we use the \atwinalt{ISO
    639}{ISO}{639}{norm} two-letter {\twintoo{country}{code}s}
  ({\snippetin{de}}$\;\widehat=\;$German, {\snippetin{en}}$\;\widehat=\;$English,
  {\snippetin{fr}}$\;\widehat=\;$French, {\snippetin{nl}}$\;\widehat=\;$Dutch, \ldots). If
  no {\attributeshort[ns-attr=xml]{lang}} is given, then {\attvalshort{en}{xml:lang}} is
  assumed as the default value. 

\begin{lstlisting}[escapechar=\%,label=lst:multiling,mathescape,
  caption={A Multilingual Group of {\element{CMP}} Elements},
  index={trl,xml:lang}]
 <omtext>
   Let <om:OMV id="set"name="V"/> be a set. 
   A <term role="definiendum">unary operation</term> on 
   <om:OMR href="#set"/> is a function <om:OMV  id="fun" name="F"/> with
  <om:OMA id="im">
     <om:OMS cd="relations1" name="eq"/>
     <om:OMA><om:OMS cd="fns1" name="domain"/><om:OMV name="F"/></om:OMA>
     <om:OMV name="V"/>
   </om:OMA>
   and 
  <om:OMA id="ran">
     <om:OMS cd="relations1" name="eq"/>
     <om:OMA><om:OMS cd="fns1" name="range"/><om:OMV name="F"/></om:OMA>
     <om:OMV name="V"/>
   </om:OMA>.
 </omtext>
 <omtext xml:lang="de">
   Sei <om:OMR href="#set"/> eine Menge. 
   Eine <term role="definiendum">un%\"a%re Operation</term> 
   ist eine Funktion <om:OMR href="#fun"/>, so dass
   <om:OMR href="#im"/> und <om:OMR href="#ran"/>.
 </omtext>
 <omtext xml:lang="fr">
   Soit <om:OMR href="#set"/> un ensemble. 
   Une <term role="definiendum">op%\'e%ration unaire</term> s%\^u%r
   <om:OMR href="#set"/> est une fonction <om:OMR href="#fun"/> 
   avec <om:OMR href="#im"/> et <om:OMR href="#ran"/>.
 </omtext>
\end{lstlisting}

  {\mylstref{multiling}} shows an example of a {\twintoo{multilingual}{group}}. Here, the
  {\openmath} extension by {\indextoo{DAG}} representation (see {\sref{openmath}})
  facilitates multi-language support: Only the language-dependent parts of the text have
  to be rewritten, the (language-independent) formulae can simply be re-used by
  \indexalt{cross-referencing}{cross-reference}.
\end{example}
\end{omgroup}
\begin{omgroup}[id=docmatter]{Frontmatter and Backmatter}
\ednote{describe index and tableofcontents, and backport it to OMDoc1.3}
\end{omgroup}
\end{module}
\end{omgroup}
%%% Local Variables: 
%%% mode: latex
%%% TeX-master: "main"
%%% End: 

% LocalWords:  xmlns RDF dtd lst ATTLIST DC qtmetadata ref ns attr
% LocalWords:  otheromdoctype xref mathescape enum qtgeneral dc activemath rt
% LocalWords:  ing openmath attribs xsi schemaLocation struc tu DCelement mtxt
% LocalWords:  todo heureka lang CMPs Der Beweis ist extrem schwer omdoc BNF cf
% LocalWords:  courseware namespace omdoc PCDATA truecm DOCTYPE CDATA xml CMP
% LocalWords:  omtext FMP de flattena flattenb
% LocalWords:  MDelt creativecommons elt metadata Bachus Naur dataset
