\begin{omgroup}[id=omdoc-ns]{The OMDoc Namespaces}

The namespace for the {\omdocv2} format is the \atwinalt{URI}{OMDoc}{namespace}{URI}
\url{http://omdoc.org/ns}. Note that the \omdoc \twinalt{namespace}{OMDoc}{namespace}
does not reflect the versions\footnote{The namespace is different from the {\omdocv1}
  formats (versions 1.0, 1.1, and 1.2), which was
  {\snippet{http://www.mathweb.org/omdoc}}, but the {\omdocv2} namespace will stay
  constant over all versions of the {\omdocv2} format.}, this is done in the
{\attributeshort{version}} attribute on the {\twintoo{document}{root}} element
\element{omdoc} (see \sref{omdoc-infrastructure}).  As a consequence, the
\omdoc vocabulary identified by this namespace is not static, it can change with each
new \omdoc version. However, if it does, the changes will be documented in later
versions of the specification: the latest released version can be found
at~\cite{URL:omdocspec}.

In an \omdoc document, the \omdoc namespace must be specified either using a
{\twintoo{namespace}{declaration}} of the form
{\snippet{xmlns="}}\url{http://omdoc.org/ns}{\snippet{"}} on the \element{omdoc} element
or by prefixing the {\twintoo{local}{name}s} of the \omdoc elements by a namespace
prefix (\omdoc customarily use the prefixes {\snippet{omdoc:}} or {\snippet{o:}}) that
is declared by a {\atwintoo{namespace}{prefix}{declaration}} of the form
{\snippet{xmlns:o="}}\url{http://omdoc.org/ns}{\snippet{"}} on some element dominating the
\omdoc element in question (see {\extref{book}{xml}} for an introduction). \omdoc also
uses the following namespaces\footnote{In this specification we will use the
  {\twintoo{namespace}{prefix}es} above on all the elements we reference in text unless
  they are in the \omdoc namespace.}:

\begin{scriptsize}
  \begin{center}
    \begin{tabular}{|l|l|l|}\hline
      Format      & namespace URI & see \\\hline\hline
      Dublin Core & \url{http://purl.org/dc/elements/1.1/} &   {\srefs{dc-elements}{dc-roles}}\\\hline
      Creative Commons & \url{http://creativecommons.org/ns} & \sref{creativecommons}\\\hline
      {\mathml} & \url{http://www.w3.org/1998/Math/MathML} & \sref{cmml}\\\hline
      {\openmath} & \url{http://www.openmath.org/OpenMath} & \sref{openmath}\\\hline
      {\xslt} & \url{http://www.w3.org/1999/XSL/Transform} & \sref{pres}\\\hline
    \end{tabular}
  \end{center}
\end{scriptsize}
Thus a typical document root of an \omdoc document looks as follows:
  \begin{lstlisting}[mathescape]
<?xml version="1.0" encoding="utf-8"?>
<omdoc xml:id="test.omdoc" version="1.6"
  xmlns="http://omdoc.org/ns"
  xmlns:cc="http://creativecommons.org/ns"
  xmlns:dc="http://purl.org/dc/elements/1.1/"
  xmlns:om="http://www.openmath.org/OpenMath"
  xmlns:m="http://www.w3.org/1998/Math/MathML">
$\ldots$
</omdoc>
\end{lstlisting}  
\end{omgroup}

%%% Local Variables:
%%% mode: latex
%%% TeX-master: t
%%% End:
