\begin{omgroup}[id=spec-intro.modular]{\omdoc as a Modular Format}

A modular approach to design is generally accepted as best practice in the development of
any type of complex application. It separates the application's functionality into a
number of "{\indextoo{building blocks}}" or "{\indextoo{module}s}", which are subsequently
combined according to specific rules to form the entire application. This approach offers
numerous advantages: The increased {\indextoo{conceptual clarity}} allows developers to
share ideas and code, and it encourages reuse by creating well-defined modules that
perform a particular task. Modularization also reduces complexity by decomposition of the
application's functionality and thus decreases debugging time by localizing errors due to
design changes. Finally, flexibility and maintainability of the application are increased
because single modules can be upgraded or replaced independently of others.

The \omdoc vocabulary has been split by thematic role, which we will briefly overview in
{\myfigref{omdoc-modules}} before we go into the specifics of the respective modules in
{\srefl{mobj}{quiz}}. To avoid repetition, we will introduce some attributes already in
this chapter that are shared by elements from all modules. In \sref{document-model} we
will discuss the \omdoc document model and possible sub-languages of \omdoc that only
make use of parts of the functionality (\sref{sub-languages}).

\begin{myfig}{omdoc-modules}{The \omdoc Modules}
\begin{small}
\begin{tabular}{|l|l|l|l|}\hline
  Module & Title & Required? & Chapter\\\hline\hline
  {\presbf\MOBJmodule{spec}} &  Mathematical Objects & yes & \sref{mobj}\\\hline
    \multicolumn{4}{|p{11cm}|}{\presem Formulae are a central part of mathematical
       documents; this module integrates the content-oriented representation
       formats {\openmath} and {\mathml} into \omdoc}\\\hline\hline
  {\presbf\MTXTmodule{spec}} &  Mathematical Text & yes & \sref{mtext}\\\hline
    \multicolumn{4}{|p{11cm}|}{\presem Mathematical vernacular,
  i.e. natural language with embedded formulae}\\\hline\hline
  {\presbf\DOCmodule{spec}} & Document Infrastructure & yes & \sref{omdoc-infrastructure}\\\hline
    \multicolumn{4}{|p{11cm}|}{\presem  A basic infrastructure for
      assembling pieces of  mathematical knowledge into functional documents and 
      referencing their parts }\\\hline\hline
  {\presbf\METAmodule{spec}} &  Metadata & yes &   {\srefs{dc-elements}{dc-roles}}\\\hline
    \multicolumn{4}{|p{11cm}|}{\presem Contains bibliographical and licensing metadata 
      (``{\twintoo{data}{about data}}'') 
      which can be used to annotate many \omdoc elements by descriptive and
      administrative information that facilitates navigation and organization}\\\hline\hline 
  {\presbf\RTmodule{spec}} & Rich Text Structure & no & \sref{rt}\\\hline
    \multicolumn{4}{|p{11cm}|}{\presem Rich text structure in
  mathematical vernacular (lists, paragraphs, tables, \ldots)}\\\hline\hline
  {\presbf\STmodule{spec}} &  Mathematical Statements & no  & \sref{statements}\\\hline
    \multicolumn{4}{|p{11cm}|}{\presem Markup for mathematical forms like 
      {\indextoo{theorem}s},  {\indextoo{axiom}s}, {\indextoo{definition}s}, 
      and {\indextoo{example}s} that can be used to specify or define properties
      of given mathematical objects and theories to group mathematical
  statements and provide a notion of context.}\\\hline\hline
  {\presbf\PFmodule{spec}} &  Proofs and proof objects & no & \sref{proofs}\\\hline 
    \multicolumn{4}{|p{11cm}|}{\presem Structure of proofs
     and argumentations at various levels of details and formality}\\\hline\hline
  {\presbf\ADTmodule{spec}} &  Abstract Data Types & no & \sref{adt}\\\hline 
    \multicolumn{4}{|p{11cm}|}{\presem  Definition schemata for
      sets that are built up inductively from constructor symbols}\\\hline\hline 
  {\presbf\CTHmodule{spec}} & Complex Theories & no & \sref{complex-theories}\\\hline
    \multicolumn{4}{|p{11cm}|}{\presem Theory morphisms; they can be used
    to structure mathematical theories}\\\hline\hline
  {\presbf\DGmodule{spec}} & Development Graphs & no & \sref{development-graphs}\\\hline
    \multicolumn{4}{|p{11cm}|}{\presem Infrastructure for managing theory
  inclusions, change management}\\\hline\hline
  {\presbf\EXTmodule{spec}} & Applets, Code, and Data & no & \sref{ext}\\\hline
    \multicolumn{4}{|p{11cm}|}{\presem Markup for applets, program code,
  and data (e.g. images, measurements, \ldots)}\\\hline\hline
  {\presbf\PRESmodule{spec}} & Presentation Information & no &  \sref{pres}\\\hline
    \multicolumn{4}{|p{11cm}|}{\presem Limited functionality for
    specifying presentation and notation information for local typographic
      conventions  that cannot be determined by general principles alone}\\\hline\hline
  {\presbf\QUIZmodule{spec}} &  Infrastructure for Assessments & no & \sref{quiz}\\\hline
    \multicolumn{4}{|p{11cm}|}{\presem Markup for exercises integrated
    into the \omdoc document model}\\\hline 
  \end{tabular}
\end{small}
\end{myfig}
The first four modules in {\myfigref{omdoc-modules}} are required (mathematical documents
without them do not really make sense), the other ones are optional. The
document-structuring elements in module {\DOCmodule{spec}} have an attribute
{\attributeshort{modules}} that allows to specify which of the modules are used in a
particular document (see \sref{omdoc-infrastructure} and
\sref{sub-languages}).
\end{omgroup}

%%% Local Variables:
%%% mode: latex
%%% TeX-master: t
%%% End:
