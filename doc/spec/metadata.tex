%%%%%%%%%%%%%%%%%%%%%%%%%%%%%%%%%%%%%%%%%%%%%%%%%%%%%%%%%%%%%%%%%%%%%%%%%
% This file is part of the LaTeX sources of the OMDoc 1.6 specification
% Copyright (c) 2006 Michael Kohlhase
% This work is licensed by the Creative Commons Share-Alike license
% see http://creativecommons.org/licenses/by-sa/2.5/ for details
% The source original is at https://github.com/KWARC/OMDoc/doc/spec 
%%%%%%%%%%%%%%%%%%%%%%%%%%%%%%%%%%%%%%%%%%%%%%%%%%%%%%%%%%%%%%%%%%%%%%%%%

\begin{module}[id=genmeta]
\begin{omgroup}[creators={miko,clange},short={Metadata},id=metadatachap]
  {Metadata (Module {\METAmodule{spec}})}

\begin{newpart}{@CL, please re-read and expand}
  \indexalt{Metadata}{metadata} is ``{\twintoo{data}{about data}}'' --- in the case of
  \omdoc data about documents fragments, such as titles, authorship, language usage, or
  administrative aspects like modification dates, distribution rights, identifiers, or
  version information. \omdoc documents also contain data about realations between
  mathematical objects, statements, and theories, that other applications would consider
  as metadata. To ensure interoperability with such applications and the Semantic Web,
  \omdoc supports the extraction of \omdoc-specific metadata to the RDF
  format\ednote{MK@CL write up something about processing in projects or processing and
    reference it here} and the annotation of many \omdoc elements with web-ontology
  metadata.

  In this section we will introduce the metadata framework for {\twintoo{strict}{omdoc}},
  which provides a generic, extensible infrastructure for adding metadata based on the
  recently stabilized RDFa~\cite{AdidaEtAl08:RDFa} a standard for flexibly embedding
  metadata into X(HT)ML documents. This design decision allows us to separate the
  {\emph{syntax}} (which is standardized in RDFa) from the {\emph{semantics}}, which we
  externalize in metadata ontologies, which can be encoded in \omdoc; see
  {\extref{primer}{ontology}}.



\begin{omtext}
  The \omdoc format will incorporate various concrete metadata vocabularies as part of
  the document format. These are given as documented ontologies encoded as \omdoc
  theories and are normative parts of the format specification. Note that these metadata
  theories\ednote{MK@MK: reference theory framework (probably the strict one)} can be
  thought of as the minimal specifications of the metadata: Refinements are admissible, if
  they are formulated as \omdoc theories that entertain views into the normative
  theories.\ednote{MK@MK: dream up a framework how to include the ontologies into the spec
    (probably as the DC and CC chapters; but do not forget the assertion type ontology,
    see \tracticket{1511}).}
\end{omtext}

\begin{myfig}{qtmetadata}{Metadata in \omdoc}
  \begin{scriptsize}
\begin{tabular}{|>{\tt}l|>{\tt}l|>{\tt}l|>{\tt}l|}\hline
{\rm Element}& \multicolumn{2}{l|}{Attributes\hspace*{2.25cm}} & Content  \\\hline
                 & {\rm Req.}  & {\rm Optional}   &           \\\hline\hline
metadata  &             &                     &  (meta|link)*\\\hline
meta         & property,content         &  datatype               & ANY \\\hline
link           & rel & href & (resource|mlink|meta)*\\\hline
resource   & & typeof, about & (meta|link)*\\\hline
\end{tabular}
\end{scriptsize}
\end{myfig}

\begin{definition}[id=metadata.def]
  The {\eldef{metadta}} element contains content encodings for RDF triples and
  resources. \ednote{@CL: need to verbalize this.}
\end{definition}
\ednote{@CL: explain curies here.}  
\ednote{@CL: need a simple example here, best DC  metadata that we can take up later. }

\begin{omgroup}{Pragmatic to Strict Translation}
  Every \omdoc element that admits a \element{metadata} child can serve as a metadata
  subject, so we can offer the following pragmatic (abbreviative) syntax:
\begin{center}\lstset{frame=none,numbers=none,lineskip=-.7ex,aboveskip=-.5em,belowskip=-1em,mathescape}
 \begin{tabular}{|p{5cm}|p{5cm}|}\hline
    pragmatic & strict equivalent\\\hline
\begin{lstlisting}
<$\llquote{elt}$ rel="$\llquote{r}$" href="$\llquote{h}$">
  $\llquote{body}$
</$\llquote{elt}$>
\end{lstlisting}
&
\begin{lstlisting}
<$\llquote{elt}$>
  <metadata>
    <link rel="$\llquote{r}$" href="$\llquote{h}$"/>
  </metadata>
  $\llquote{body}$
</$\llquote{elt}$>
\end{lstlisting}
\\\hline 
\begin{lstlisting}
<$\llquote{elt}$ rel="$\llquote{r}$" href="$\llquote{h}$">
  <metadata>
   $\llquote{meta}$
  </metadata>
  $\llquote{body}$
</$\llquote{elt}$>
\end{lstlisting}
&
\begin{lstlisting}
<$\llquote{elt}$>
  <metadata>
    <link rel="$\llquote{r}$" href="$\llquote{h}$"/>
    $\llquote{meta}$
 </metadata>
  $\llquote{body}$
</$\llquote{elt}$>
\end{lstlisting}
\\\hline 
\end{tabular}
\end{center}
\end{omgroup}


{\omdocv{1.2}} had some descriptions of metadata inhertance for Dublin Core Metadata. In
the new metadata framework we do not need to explicitly present this, since it is part of
the metadata ontology: If a metadatum can be inferred it is defined to be
present. \ednote{@CL, make a simple concrete example.}
\end{newpart}
\end{omgroup}
\end{module}

%%% Local Variables: 
%%% mode: latex
%%% TeX-master: "main"
%%% End: 

