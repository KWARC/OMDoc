%%%%%%%%%%%%%%%%%%%%%%%%%%%%%%%%%%%%%%%%%%%%%%%%%%%%%%%%%%%%%%%%%%%%%%%%%
% This file is part of the LaTeX sources of the OMDoc 1.6 specification
% Copyright (c) 2006 Michael Kohlhase
% This work is licensed by the Creative Commons Share-Alike license
% see http://creativecommons.org/licenses/by-sa/2.5/ for details
% The source original is at https://github.com/KWARC/OMDoc/doc/spec 
%%%%%%%%%%%%%%%%%%%%%%%%%%%%%%%%%%%%%%%%%%%%%%%%%%%%%%%%%%%%%%%%%%%%%%%%%

\begin{omgroup}[id=att-table,short=Table of Attributes]
               {Quick-Reference Table to the {\omdoc} Attributes}
\def\atabelt#1#2#3#4{\hline{}{#1}&{#2}&{#3}\\\hline&\multicolumn{2}{|p{9cm}|}{#4}\\\hline}
\begin{footnotesize}
\begin{longtable}{|>{\tt}p{2.5cm}|>{\tt}p{4cm}|>{\tt}p{5cm}|}\hline
{\rm Attribute} & {\emph{element}} & Values \\\hline
\atabelt{action}{dc:date}{unspecified}{specifies the action taken on the document on this date.}

\atabelt{action}{omlet}{execute, display, other}
 {specifies the action to be taken when executing the {\element{omlet}}, the value is
     application-defined.}

\atabelt{actuate}{omlet}{onPresent, onLoad, onRequest, other}{specifies the timing of the
  action specified in the {\attribute{action}{omlet}} attribute}

\atabelt{assertion}{example}{}
 {specifies the assertion that states that the objects given in the example really have
   the expected properties.}

\atabelt{assertion}{obligation}{}
 {specifies the assertion that states that the translation of the statement in the
  source theory specified by the {\attributeshort{induced-by}} attribute is valid in the
  target theory.}

\atabelt{attributes}{use}{}
 {the attribute string for the start tag of the {\xml} element  substituted for
 the brackets (this is specified in the {\element{element}} attribute).}
\atabelt{attribution}{cc:requirements}{required, not\_required}
{Specifies whether the copyright holder/author must be given credit in derivative works}

\atabelt{base}{morphism}{}
 {specifies another morphism that should be used as a base for expansion in the
  definition of this morphism}

\atabelt{bracket-style}{presentation, use}{lisp, math}
 {specifies whether a function application is of the form $f(a,b)$ or $(f a b)$}

\atabelt{cd}{om:OMS}{}
 {specifies the content dictionary of an {\openmath} symbol}

\atabelt{cd}{term}{}
 {specifies the content dictionary of a technical term}

\atabelt{cdbase}{om:*}{}
 {specifies the base URI of the content dictionaries used in an {\openmath} object}

\atabelt{cdreviewdate}{theory}{}
  {specifies the date until which the content dictionary will remain unchanged}

\atabelt{cdrevision}{theory}{}
  {specifies the minor version number of the content dictionary}

\atabelt{cdstatus}{theory}{official, experimental, private, obsolete}
  {specifies the content dictionary status}

\atabelt{cdurl}{theory}{}
 {the main URL, where the newest version of the content dictionary can be found}

\atabelt{cdversion}{theory}{}
  {specifies the major version number of the content dictionary}

\atabelt{comment}{ignore}{}
 {specifies a reason why we want to ignore the contents}

\atabelt{crossref-symbol}{presentation, use}
 {all, brackets, lbrack, no, rbrack, separator, yes}
 {specifies whether {\indextoo{cross-reference}s} to
  the symbol definition should be generated in the output format.}

\atabelt{class}{*}{}{specifies the {\css} class}
\atabelt{commercial\_use}{cc:permissions}{permitted, prohibited}{specifies, whether commercial use of the
  document with this license is permitted}
\atabelt{consistency}{morphism, definition}{OMDoc reference}{points to an {\element{assertion}}
  stating that the cases are consistent, i.e. that they give the same values, where they
  overlap}
\atabelt{copyleft}{cc:restrictions}{required, not\_required}{specifies whether derived works must be licensed
  with the same license as the current document.}
\atabelt{cr}{element}{yes/no}{specifies whether an {\attributeshort{xlink:href}}
  cross-reference should be set on the result element.}
\atabelt{crid}{element}{{\xpath} expression}{the path to the sub-element that corresponds
  to the result element.}
\atabelt{crossref-symbol}{presentation, use}{no, yes, brackets, separator, lbrack, rbrack, all}
{specifies which generated presentation elements should carry cross-references to the definition.}
\atabelt{data}{omlet}{}
 {points to a {\element{private}} element that contains the data for this {\element{omlet}}}
\atabelt{definitionURL}{m:*}{URI}{points to the definition of a mathematical concept}
\atabelt{derivative\_works}{cc:permissions}{permitted, not\_permitted}{specifies whether
  the document may be used for making derivative works.}
\atabelt{distribution}{cc:permissions}{permitted,not\_permitted}{specifies whether
  distribution of the current document fragment is permitted.}
\atabelt{element}{use}{}
 {the {\xml} element tags to be substituted for the brackets.}
\atabelt{element}{omstyle}{}
 {the {\xml} element, the presentation information contained in the {\element{omstyle}}
   element should be applied to.}
\atabelt{encoding}{m:annotation,om:OMFOREIGN}{MIME type of the content}{specifies the
  format of the content}
\atabelt{entails, entailed-by}{alternative}{}
 {specifies the equivalent formulations of a definition or axiom}

\atabelt{entails-thm, entailed-by-thm}{alternative}{}
 {specifies the entailment statements for equivalent formulations of a 
  definition or axiom}
\atabelt{exhaustivity}{morphism, definition}{OMDoc reference}{points to an assertion that
 states that the cases are exhaustive.}

\atabelt{existence}{definition}{OMDoc reference}{points to an assertion that
 states that the symbol described in an implicit definition exists}

\atabelt{fixity}{presentation}{assoc, infix, postfix, prefix}
  {specifies where the function symbol-of a function application should be 
   displayed in the output format}

\atabelt{function}{omlet}{}
 {specifies the function to be called when this {\element{omlet}} is activated.}
 
 \atabelt{format}{data}{} {specifies the format of the data specified by a {\element{data}}
   element. The value should e.g. be a {\twintoo{MIME}{type}}~\cite{FreBor:MIME96}.}

\atabelt{for}{*}{}
 {can be used to reference an element by its unique identifier given in its 
  {\attributeshort[ns-attr=xml]{id}} attribute.}

\atabelt{formalism}{legacy}{URI reference}
 {specifies the formalism in which the content is expressed}

\atabelt{format}{legacy}{URI reference}
 {specifies the encoding format of the content}

\atabelt{format}{use}{cmml, default, html, mathematica, pmml, TeX,\ldots}
 {specifies the output format for which the notation is specified}

\atabelt{from}{imports, theory-inclusion, axiom-inclusion}{URI reference}{pointer to source
  {\element{theory}} of a theory morphism}
\atabelt{from}{omtext}{URI reference}{points to the source of a relation given by a text type}

\atabelt{generated-from}{top-level elements}{URI reference}
 {points to a higher-level syntax element, that generates this statement.}

\atabelt{generated-via}{top-level elements,\ldots}{URI reference}
 {points to a theory-morphism, via which it is translated from the element pointed to by
 the {\attributeshort{generated-from}} attribute.}

\atabelt{globals}{path-just}{}
  {points to the {\element{axiom-inclusion}s} or {\element{theory-inclusion}s} that is the rest of the inclusion path.}

\atabelt{hiding}{morphism}{}
 {specifies the names of symbols that are in the domain of the morphism}

\atabelt{href}{data, link, om:OMR}{URI reference}
 {a URI to an external file containing the data.}

\atabelt{xml:id}{}{}
 {associates a unique identifier to an element, which can thus be referenced 
  by an {\attributeshort{for}}  or {\attributeshort{xref}} attribute.}

\atabelt{xml:base}{}{}
 {specifies a base URL for a resource fragment}

\atabelt{index}{on {\RTmodule{spec}} elements}{}{A path identifier to establish multilingual
  correspondence}

\atabelt{induced-by}{obligation}{}
 {points to the statement in the source theory that induces this proof obligation}

\atabelt{inductive}{assumption, hypothesis}{yes, no}
 {Marks an assumption or hypothesis inductive.}

\atabelt{inherits}{metadata}{URI reference}{points to a metadata element from which this
  one inherits.}

\atabelt{jurisdiction}{cc:license}{IANA Top level Domain designator}{specifies the country
  of jurisdiction for a Creative Commons license}

\atabelt{just-by}{type}{}
 {points to an assertion that states the type property in question.}

\atabelt{role}{symbol, constructor, recognizer, selector, sortdef}{object, type, sort, binder,
 attribution, semantic-attribution, error}
 {specifies the role (possible syntactic roles) of the symbol  in this declaration.}

\atabelt{role}{dc:creator,dc:contributor}{MARC relators}
 {specifies the role of a person who has contributed to the document}

\atabelt{role}{presentation}{applied, binding, key}{specifies which role of the symbol is
  annotated with notation information}

\atabelt{lbrack}{presentation, use}{}
 {the left bracket to use in the notation of a function symbol}

\atabelt{links}{decomposition}{}
 {specifies a list of theory-  or axiom-inclusions that justify (by decomposition)
 the {\element{theory-inclusion}} specified  in the {\attributeshort{for}} attribute.}

\atabelt{local}{path-just}{}
 {points to the {\element{axiom-inclusion}} that is the first element in the path.}

\atabelt{logic}{FMP}{{\rm token}}
 {specifies the logical system used to encode the property.} 

\atabelt{modules}{omdoc, omdoc}{module and sub-language shorthands, URI
 reference}{specifies the modules or {\omdoc} sub-language used in this document fragment}

\atabelt{name}{om:OMS, om:OMV, symbol, term}{}
 {the name of a concept referenced by a symbol, variable, or technical term.}

\atabelt{name}{attribute, element}{}
 {the local name of generated element.}

\atabelt{name}{param}{}
 {the name of a parameter for an external object.}

\atabelt{notice}{cc:requirements}{required, not\_required}{specifies whether copyright and
  license notices must be kept intact in distributed copies of this document}

\atabelt{ns}{element, attribute}{URI}{specifies the namespace URI of the generated element
or attribute node}

\atabelt{original}{data}{local, external}{specifies whether the local copy in the
  {\element{data}} element is the original or the external resource pointed to by the
  {\attribute{href}{data}} attribute.}

\atabelt{parameters}{adt}{}
  {The list of formal parameters of a higher-order abstract data type}

\atabelt{precedence}{presentation}{}
 {the precedence of a function symbol (for elision of brackets)}
 
 \atabelt{just-by}{assertion}{} {specifies a list of URIs to proofs or other
   justifications for the proof status given in the {\attribute{status}{assertion}}
   attribute.}

\atabelt{pto, pto-version}{private, code}{}
 {specifies the system and its version this data or code is private to}

\atabelt{rank}{premise}{}{specifies the rank (importance) of a premise}

\atabelt{rbrack}{presentation, use}{}
 {the right bracket  to use in the notation of a function symbol}

\atabelt{reformulates}{private}{}
 {points to a set of  elements whose content is reformulated by the content 
  of the {\element{private}} element for the system.}

\atabelt{reproduction}{cc:permissions}{permitted,not\_permitted}{specifies whether
  reproduction of the current document fragment is permitted by the licensor}

\atabelt{requires}{private, code, use, xslt, style}{URI reference}
 {points to a {\element{code}} element that is needed for the execution of this data by
  the system.}

\atabelt{role}{dc:creator, dc:collaborator}
 {aft, ant, aqt, aui, aut, clb, edt, ths, trc, trl}
 {the MARC relator code for the contribution of the individual.}

\atabelt{role}{phrase, term}{}{the role of the phrase annotation}

\atabelt{role}{presentation}{applied, binding, key}
  {specifies for which role (as the head of a function application, as a binding
 symbol, or as a key in a attribution, or as a stand-alone symbol (the default)) of
 the symbol presentation is intended}

\atabelt{scheme}{dc:identifier}{scheme name}{specifies the identification scheme
  (e.g. ISBN) of a resource}

\atabelt{scope}{symbol}{global, local}
 {specifies the visibility of the symbol declared. This is a very crude
  specification, it is better to use theories  and importing to specify symbol 
  accessibility.}

\atabelt{select}{map, recurse, value-of}{{\xpath} expression}{specifies the path to the
  sub-expression to act on}

\atabelt{separator}{presentation, use}{}
 {the separator for the arguments to use in the notation of a function symbol}

\atabelt{show}{omlet}{new, replace, embed, other}{specifies the desired presentation of the
  external object.}

\atabelt{size}{data}{}
 {specifies the size the data specified by a {\element{data}} element. The value should
  be number of kilobytes}

\atabelt{sort}{argument}{}
 {specifies the argument sort of the constructor}

\atabelt{style}{*}{}
 {specifies a token for a presentation style to be picked up in a
 {\element{presentation}} element.}

\atabelt{system}{type}{}
 {A token that specifies the logical type system that governs the type specified
 in the type element.}

\atabelt{theory}{*}{}
 {specifies the home theory of an {\omdoc} statement.}

\atabelt{to}{theory-inclusion, axiom-inclusion}{}
 {specifies the target theory}

\atabelt{total}{selector}{no, yes}
 {specifies whether the symbol declared here is a total or partial function.}

\atabelt{type}{adt}{free, generated, loose}
 {defines the semantics of an abstract data type {\snippet{free}} = no junk, no confusion,
 {\snippet{generated}} = no junk, {\snippet{loose}} is the general case.}

\atabelt{type}{assertion} 
 {theorem, lemma, corollary, conjecture, false-conjecture,
  obligation, postulate, formula, assumption, proposition}
 {tells you more about the intention of the assertion}

\atabelt{type}{definition}{implicit, inductive, obj, recursive, simple}
 {specifies the definition principle}

\atabelt{type}{derive}{conclusion, gap}
 {singles out special proof steps: conclusions and gaps (unjustified proof steps)}

\atabelt{type}{example}{against, for}
 {specifies whether the objects in this example support or falsify some conjecture}

\atabelt{type}{ignore}{}{specifies the type of error, if ignore is used for in-place error
markup}

\atabelt{type}{imports}{global, local}
 {{\snippet{local}} imports only concern the assumptions directly stated in the
   theory. {\snippet{global}} imports also concern the ones the source theory inherits.}

\atabelt{type}{morphism}{}{specifies whether the morphism is recursive or merely pattern-defined}

\atabelt{type}{omdoc, omdoc}
 {enumeration, sequence, itemize}
 {the first three give the text category, the second three are used for generalized tables}

\atabelt{type}{omtext}
 {abstract, antithesis, comment, conclusion, elaboration, evidence, 
  introduction,  motivation, thesis}
 {a specification of the intention of the text fragment, in reference to context.}

\atabelt{type}{phrase}{}{the linguistic or mathematical type of the phrase}

\atabelt{type}{ref}{include, cite}{specifies whether to replace the {\element{ref} element
  by the fragment referenced by {\attribute{href}{ref}} attribute or to merely cite it.}}

\atabelt{uniqueness}{definition}{URI reference}{points to an {\element{assertion}} that
  states the uniqueness of the concept described in an implicit definition}

\atabelt{value}{param}{}{specifies the value of the parameter}

\atabelt{valuetype}{param}{}{specifies the type of the value of the parameter}

\atabelt{verbalizes}{on {\RTmodule{spec}} elements}{URI references}{contains a
  whitespace-separated list of pointers to {\omdoc} elements that are verbalized}

\atabelt{verdict}{answer}{}
 {specifies the truth or falsity of the answer. This can be used e.g. 
  by a grading application.} 

\atabelt{version}{omdoc}{1.2}
 {specifies the version of the document, so that the right DTD is used}

\atabelt{version}{cc:license}{}
 {specifies the version of the Creative Commons license that applies, if not present, the
 newest one is assumed}

\atabelt{via}{inclusion}{}
 {points to a theory-inclusion that is required for an actualization}

\atabelt{who}{dc:date}{}{specifies who acted on the document fragment}

\atabelt{xml:lang}{CMP, dc:*}{ISO 639 code}
 {the language the text in the element is expressed in.}

\atabelt{xml:lang}{use, xslt, style}{whitespace-separated list of ISO 639 codes}
 {specifies for which language the notation is meant}

\atabelt{xlink:*}{om:OMR, m:*}{URI reference}{specify the link behavior on the elements}

 \atabelt{xref}{ref, method, premise}{URI reference}
 {Identifies the resource in question}

 \atabelt{xref}{presentation, omstyle}{URI reference}
 {The element, this URI points to should be in
   the place of the object containing this attribute.}
\end{longtable}
\end{footnotesize}
\end{omgroup}

%%% Local Variables: 
%%% mode: LaTeX
%%% TeX-master: "main"
%%% End: 

% LocalWords:  omlet onPresent onLoad onRequest omdoc omdoc cd loc cdbase cr
% LocalWords:  cdreviewdate cdrevision cdstatus cdurl cdversion crossref lbrack
% LocalWords:  rbrack xlink href crid definitionURL omstyle OMFOREIGN thm
% LocalWords:  ns attr cmml html mathematica pmml omtext globals OMR metadata
% LocalWords:  IANA sortdef relators FMP OMV param adt pto licensor xslt aqt
% LocalWords:  aui aut clb edt ths trc trl recurse valuetype lang CMP
