\begin{omgroup}{Parallel Markup}
  The idea of parallel markup has been pioneered in the MathML
  language~\cite{CarlisleEd:MathML3}. For parallel markup of a formula, MathML combines
  the presentation and content trees in a single \xml tree and marks up corresponding
  subtrees by cross-references. Parallel markup supports two workflows:
\begin{compactenum}
\item \emph{formalization}: the annotation of presentation formulae with (multiple)
  formalizations and
\item \emph{presentation}: the annotation of content formulae with (multiple)
  presentations.
\end{compactenum}
Let us look at an well-known example: The (presentation) formula $f(a+b)$ can be
interpreted in two ways: as the application of the function $f$ to the sum $a+b$ or as the
product of a scalar $f$ with the sum $a+b$. Consequently, we can annotate the presentation
tree of $f(a+b)$ with two content trees to arrive at the situation depicted in
Figure~\ref{fig:ptree-parallel}. In MathML, parallel markup is implemented by the
\lstinline|semantics| element, which has the central element as the first child and the
annotations in subsequent \lstinline|annotation-xml| children. We will disregard this
technical level here and concentrate on the concepts here.

\begin{figure}[ht]\centering
\tikzinput{tikz/ptree-parallel}
\caption{A presentation tree (center) with content annotations}\label{fig:ptree-parallel}
\end{figure}

Conversely for the presentation workflow, the product of $f$ with $a+b$ can be presented
as $f(a+b)$ (as above) but also with $f\cdot(a+b)$, so that we arrive in the situation in
Figure~\ref{fig:ctree-parallel}.  In both situations, there is a central tree annotated
with further representations in the respective other language. Which of the situations is
used depends on the context. In the content commons we would expect to see content trees
annotated with presentations (as in Figure~\ref{fig:ctree-parallel}) and in the document
commons the other way around (as in Figure~\ref{fig:ptree-parallel}). Note also that the
direction of the arrows is always from the (possibly multiple) annotations into the
center.

\begin{figure}[ht]\centering
\tikzinput{tikz/ctree-parallel}
\caption{A content tree (center) with presentation annotations}\label{fig:ctree-parallel}
\end{figure}
\end{omgroup}

\begin{omgroup}{Mixing Formal and Informal Markup in Objects}
  \ednote{Intro here; reference spec-intro}
\begin{example}
  The formula $\left\{x\in\mathbb{N}\mid|\text{$x$ is prime and $x>2$}\right\}$ can be represented
  as
\begin{lstlisting}[language=MathML]
<m:bind>
 <m:apply> 
  <m:csymbol cd="sets">setst</m:csymbol>
  <m:csymbol cd="numbersets">N</m:csymbol>
 </m:apply>
 <m:bvars><m:ci>x<m:ci></m:bvars>
 <m:semantics>
  <m:apply id="a">
   <m:ci id="ip">ipa</m:ci>
   <m:ci id="x1">x<m:ci>
   <m:apply id="xgt2">
    <m:csymbol cd="relation1" id="gt">greater</m:csymbol>
    <m:ci id="x2">x<m:ci>
    <m:cn id="n3">2</m:cn>
   </m:apply>
  </m:apply>
  <m:annotation-xml encoding="pMathML">
   <m:mrow xref="#a">
    <m:mi xref="#x1">x</m:mi>
    <m:mtext xref="#ip">is prime and</m:mtext>
    <m:mrow xref="#xgt3">
     <m:mi xref="#x2">x</m:mi>
     <m:mo xref="gt">&gt;</m:mo>
     <m:mn xref="n2">2</m:mn>
    </m:mrow>
   </m:mrow>
  </m:annotation-xml>
 <m:semantics>
</m:bind>
\end{lstlisting}
  We use content markup for the set construct and then use parallel markup for its body:
  $\text{$x$ is prime}$. The content part is partially opaque to \cmathml\ednote{explain
    opaque and/or intro above}, so we represent its as an application of an unspecified
  predicate \lstinline|ipa| applied to the content-regognizable parts $x$ and $x>2$. On
  the presentation side, we have the obvious markup, which cross-references into the
  content for parallel markup. Note that we are not assuming any fancy natural language
  processing here, which might have spotted the fact that we have a conjunction of two
  atomic requirements ``$x$ is prime'' and ``$x>2$, which would have led to a more
  strutured \cmathml expression. Instead the cross-references are established in a
  bottom-up manner; referencing corresponding sub-expressions and defaulting to the
  unspecified predicate \lstinline|ipa| in doubt.
 \end{example}
\end{omgroup}

%%% Local Variables:
%%% mode: latex
%%% TeX-master: t
%%% End:
