\begin{omgroup}[id=mobj.core]{Content Representation of Mathematical Objects}

\begin{module}[id=mobj-core]
\begin{omtext}
We will now recapitulate the representational core of \openmath and \cmathml. Both
represent mathematical \inlinedef{objects as expressions made up from
\begin{compactenum}
\item \defis{symbol} which reference previously declared mathematical objects, these are
  identified by referencing a \trefii{content}{dictionary} (see \sref{cd.def}). 
\item \defis{application} of function or relation operators to a sequence of arguments 
\item \defiis{binding}{structure} that represent functional objects with the help of
  \defiis{bound}{variable}.
\item \defis{identifier} for objects that are known locally (usually bound variables).
\end{compactenum}}
 In {\myfigref{om-commutativity}} we have put the \openmath and strict \cmathml encodings
of the law of commutativity for the real numbers side by side to show the similarities and
differences. There is an obvious line-by-line similarity for the tree constructors and
token elements.  The main difference is the treatment of annotation, which we will
describe in `\sref{annotating}. 
\end{omtext}

\begin{definition}[id=cd.def]
  A \defii[CD]{content}{dictionary} (\defi{CD}) is a \indextoo{machine-readable} document
  that defines the meaning of mathematical concepts expressed by \openmath/\mathml
  symbols.
\end{definition}

The \openmath2 standard provides a minimal data model and \xml encoding for content
dictionaries. We will not review the \openmath content dictionary format there, since
\omdoc theories fill that role in the \omdoc universe -- see \sref{identifying} for
details. 

\begin{omtext}
The central idea is that \trefis{symbol} are identified by a triple: 
\begin{compactenum}
\item \inlinedef{the URI of the file containing the CD (called the
    \defiii{content}{dictionary}{base})},
\item the \inlinedef{the name of the CD, called the \defiii{content}{dictionary}{name}},
  and 
\item the \inlinedef{local name in the CD, called the \defii{symbol}{name}}.
\end{compactenum}
\end{omtext}
\end{module}

\begin{module}[id=OpenMath]
\begin{omgroup}[id=openmath]{The Representational Core of \openmath}
\openmath is a markup language for mathematical formulae that concentrates on the
meaning of formulae building on an extremely simple kernel (markup primitive for
syntactical forms of content formulae), and adds an extension mechanism for mathematical
concepts, the content dictionaries.


We will only review the \xml encoding of {\openmath} objects here, since it is most
relevant to the \omdoc format. All elements of the {\xml} encoding live in the
\twinalt{namespace}{OpenMath}{namespace} \url{http://www.openmath.org/OpenMath}, for which
we traditionally use the \twintoo{namespace}{prefix}
\atwinalt{\snippetin{om:}}{OpenMath}{namespace}{URI}.

\begin{presonly}
\begin{myfig}{om}{{\openmath} Objects in \omdoc}
\begin{scriptsize}
\begin{tabular}{|>{\tt}l|>{\tt}l|>{\tt}l|>{\tt}l|}\hline
{\rm Element}& \multicolumn{2}{l|}{Attributes\hspace*{2.25cm}} & Content  \\\hline
             & {\rm Required}  & {\rm Optional}     &           \\\hline\hline
 \element[ns-elt=om]{OMS}          & cd, name  & id, cdbase, class, style   &  EMPTY \\\hline
  \element[ns-elt=om]{OMV}          & name & id, class, style   &  EMPTY \\\hline
  \element[ns-elt=om]{OMA}          & & id, cdbase, class, style   & \llquote{OMel}* \\\hline
  \element[ns-elt=om]{OMBIND}    & & id, cdbase, class, style   & \llquote{OMel},OMBVAR,\llquote{OMel} \\\hline
  \element[ns-elt=om]{OMBVAR}    & & id, class, style   & (OMV | OMATTR)+ \\\hline
  \element[ns-elt=om]{OMFOREIGN} & & id, cdbase, class, style   & ANY \\\hline
  \element[ns-elt=om]{OMATTR}    & & id, cdbase, class, style   & \llquote{OMel}\\\hline
  \element[ns-elt=om]{OMATP}     & & id, cdbase, class, style   & (OMS, (\llquote{OMel}|OMFOREIGN))+ \\\hline
  \element[ns-elt=om]{OMI}         & & id, class, style   &  [0-9]* \\\hline 
  \element[ns-elt=om]{OMB}        & & id,  class, style   &  \#PCDATA \\\hline 
  \element[ns-elt=om]{OMF}        & & id, class, style, dec, hex &  \#PCDATA \\\hline 
  \element[ns-elt=om]{OME}        & & id, class, style   & \llquote{OMel}?\\\hline
  \element[ns-elt=om]{OMR}       & href &      & \llquote{OMel}?\\\hline
 \multicolumn{4}{|l|}{where {\llquote{OMel}} is {\tt{(OMS|OMV|OMI|OMB|OMSTR|OMF|OMA|OMBIND|OME|OMATTR)}}}\\\hline
\end{tabular}
\end{scriptsize}
\end{myfig}
\end{presonly}

\begin{definition}
  The \eldef[ns-elt=om]{OMA} element contains representations of the function and its
  argument in ``\twinalt{prefix-}{prefix}{notation}'' or ``{\twintoo{Polish}{notation}}'',
  i.e. the first child is the representation of the function and all the subsequent ones
  are representations of the arguments in order.
\end{definition}

\begin{definition}
  Objects and concepts that carry meaning independent of the local context (they are
  called \defis{symbol}) in {\openmath}) are represented as \eldef[ns-elt=om]{OMS}
  elements, where the value of the \attribute[ns-elt=om]{name}{OMS} attribute gives the
  name of the symbol.  The \attribute[ns-elt=om]{cd}{OMS} attribute specifies the relevant
  \trefii{content} {dictionary}, the optional \attributeshort{cdbase} on an
  \element[ns-elt=om]{OMS} element contains a {\indextoo{URI}} that can be used to
  disambiguate the content dictionary.  Alternatively, the {\attributeshort{cdbase}}
  attribute can be given on an {\openmath} element that is a parent to the
  \element[ns-elt=om]{OMS} in question: The \element[ns-elt=om]{OMS} inherits the
  {\attributeshort{cdbase}} of the nearest ancestor (inducing the usual {\xml} scoping
  rules for declarations).\footnote{Note that while the {\attributeshort{cdbase}}
    inheritance mechanism described here remains in effect for {\openmath} objects
    embedded in to the \omdoc format, it is augmented by one in \omdoc. As a consequence,
    {\openmath} objects in \omdoc documents will usually not contain
    {\attributeshort{cdbase}} attributes; see \sref{identifying} for a discussion.}
\end{definition}

\begin{definition}
  Variables are represented as \eldef[ns-elt=om]{OMV} element.  As variables do not carry
  a meaning independent of their local content, \element[ns-elt=om]{OMV} only carries a
  \attribute[ns-elt=om]{name}{OMV} attribute (see \sref{sem-var} for further
  discussion).
\end{definition}

\begin{example}
  The formula $\sin(x)$ would be modeled as an application of the $\sin$ function (which
  in turn is represented as an {\openmath} symbol) to a variable:
\begin{lstlisting}[label=sinx,language=OpenMath,numbers=none,index={OMA,OMV,OMS}]
<OMA xmlns="http://www.openmath.org/OpenMath"
     cdbase="http://www.openmath.org/cd">
  <OMS cd="transc1" name="sin"/>
  <OMV name="x"/>
</OMA>
\end{lstlisting}
  In our case, the function $\sin$ is represented as an \element[ns-elt=om]{OMS} element
  with name {\snippet{sin}} from the {\indextoo{content dictionary}}
  {\snippet{transc1}}. The \element[ns-elt=om]{OMS} inherits the
  {\attributeshort{cdbase}}-value \url{http://www.openmath.org/cd}, which shows that it
  comes from the {\openmath} standard collection of content dictionaries from the
  \element[ns-elt=om]{OMA} element above.  The variable $x$ is represented in an
  \element[ns-elt=om]{OMV} element with \attribute[ns-elt=om]{name}{OMV}-value
  {\snippet{x}}.
\end{example}

\begin{example}
  For the \element[ns-elt=om]{OMBIND} element consider the following representation of the
  formula $\allcdot{x}{\sin(x)\leq\pi}$.
\begin{lstlisting}[label=allxsinx,language=OpenMath,numbers=none,
   index={OMA,OMV,OMBIND,OMBVAR}]
<OMBIND xmlns="http://www.openmath.org/cd">
  <OMS cd="quant1" name="forall"/>
  <OMBVAR><OMV name="x"/></OMBVAR>
  <OMA>
    <OMS cd="arith1" name="leq"/>
    <OMA><OMS cd="transc1" name="sin"/><OMV name="x"/></OMA>
    <OMS cd="nums1" name="pi"/>
  </OMA>
</OMBIND>
\end{lstlisting}
\end{example}

\begin{definition}[id=ombind.def]
  The {\eldef[ns-elt=om]{OMBIND}} element has exactly three children, the first one is a
  ``{\twintoo{binding}{operator}}''\footnote{%\label {foot:binding-operator}
    The binding operator must be a symbol which either has the {\indextoo{role}}
    \attvalshort{binder}{role} assigned by the {\openmath} content dictionary
    (see~\cite{BusCapCar:2oms04} for details) or the symbol declaration in the \omdoc
    content dictionary must have the value \attval{binder}{role}{symbol} for the attribute
    \attribute{role}{symbol} (see \sref{symbol-dec}).} --- in this case the universal
  quantifier, the second one is a list of {\twintoo{bound}{variable}s} that must be
  encapsulated in an {\eldef[ns-elt=om]{OMBVAR}} element, and the third is the
  {\indextoo{body}} of the binding object, in which the bound variables can be used.
  {\openmath} uses the \element[ns-elt=om]{OMBIND} element to unambiguously specify the
  scope of bound variables in expressions: the bound variables in the
  \element[ns-elt=om]{OMBVAR} element can be used only inside the mother
  \element[ns-elt=om]{OMBIND} element, moreover they can be systematically
  \twinalt{renamed}{variable}{renaming} without changing the meaning of the binding
  expression. As a consequence, bound variables in the scope of an
  \element[ns-elt=om]{OMBIND} are distinct as {\openmath} objects from any variables
  outside it, even if they share a name.
\end{definition}
\end{omgroup}
\end{module}

\begin{module}[id=cMathML]
\begin{omgroup}[id=cmml]{Strict Content MathML}

  \cmathml is a content markup format that represents the abstract structure of formulae
  in trees of logical sub-expressions much like \openmath: the \mathml3
  recommendation~\cite{CarlisleEd:MathML3} identifies a sublanguage: strict \cmathml that
  is isomorphic to \openmath2. 
\begin{presonly}
\begin{myfig}{cmml}{{\cmathml} in \omdoc}
\begin{scriptsize}
\begin{tabular}{|l|l|p{2truecm}|p{4truecm}|}\hline
{\rm Element}& \multicolumn{2}{l|}{Attributes\hspace*{2.25cm}}  & Content  \\\hline
              & {\rm Required}  & {\rm Optional}     &            \\\hline\hline
 \element[ns-elt=m]{math}       & & 
 \attribute[ns-elt=m]{id}{math}, 
 \attribute[ns-elt=m]{xlink:href}{math}                   & \llquote{CMel}+\\\hline

 \element[ns-elt=m]{apply}  & &
 \attribute[ns-elt=m]{id}{apply}, 
   \attribute[ns-elt=m]{xlink:href}{apply}                   &
 \element[ns-elt=m]{bvar?},\llquote{CMel}*\\\hline

\ \element[ns-elt=m]{csymbol}    & 
 \attribute[ns-elt=m]{definitionURL}{csymbol}  & 
\attribute[ns-elt=m]{id}{csymbol}, 
\attribute[ns-elt=m]{xlink:href}{csymbol}    & 
EMPTY \\\hline

 \element[ns-elt=m]{ci}        & & 
\attribute[ns-elt=m]{id}{ci}, 
\attribute[ns-elt=m]{xlink:href}{ci}  &
 \#PCDATA \\\hline       

 \element[ns-elt=m]{cn}         & & 
\attribute[ns-elt=m]{id}{cn}, 
\attribute[ns-elt=m]{xlink:href}{cn}         &
 ([0-9]|,|.)(*|e([0-9]|,|.)*)?\\\hline       

 \element[ns-elt=m]{bvar}       & &
 \attribute[ns-elt=m]{id}{bvar}, 
 \attribute[ns-elt=m]{xlink:href}{bvar}                   &
 \element[ns-elt=m]{ci}|
\element[ns-elt=m]{semantics}\\\hline

 \element[ns-elt=m]{semantics}  & & 
\attribute[ns-elt=m]{id}{semantics}, 
\attribute[ns-elt=m]{xlink:href}{semantics}, 
\attribute[ns-elt=m]{definitionURL}{semantics} & 
\llquote{CMel},(\element[ns-elt=m]{annotation} | 
                        \element[ns-elt=m]{annotation-xml})*\\\hline

 \element[ns-elt=m]{annotation} & & 
\attribute[ns-elt=m]{definitionURL}{annotation}, \attribute[ns-elt=m]{encoding}{annotation}      &
 \#PCDATA \\\hline

 \element[ns-elt=m]{annotation-xml} & & 
\attribute[ns-elt=m]{definitionURL}{annotation-xml}, 
\attribute[ns-elt=m]{encoding}{annotation-xml}      &
 ANY \\\hline
 \multicolumn{4}{|l|}{where {\llquote{CMel}} is 
\element[ns-elt=m]{apply}|
     \element[ns-elt=m]{csymbol}|
\element[ns-elt=m]{ci}|
\element[ns-elt=m]{cn}|\element[ns-elt=m]{semantics}}\\\hline
\end{tabular}
\end{scriptsize}
\end{myfig}
\end{presonly}

\begin{definition}[id=math.def]
  The top-level element of \mathml is the {\eldef[ns-elt=m]{math}} element it contains an
  functional expression composed
  \begin{compactenum}
  \item \trefis{identifier} (element \eldef[ns-elt=m]{ci}) corresponding to
    {\indextoo{variable}s}. The content of the \element[ns-elt=m]{ci} element is a unicode
    string used as the name of the identifier.
  \item \trefis{symbol} (element \eldef[ns-elt=m]{csymbol}) for arbitrary symbols. The
    content of the \element[ns-elt=m]{csymbol} element is the name of the symbol, its
    meaning is dermined by the \attribute{csymbol}{cd} attribute that contains a content
    dictionary name.
  \item \trefis{application} (element\eldef[ns-elt=m]{apply})    
  \item \trefiis{binding}{structure} (element \eldef[ns-elt=m]{bind} with
    \eldef[ns-elt=m]{bvars}).
  \end{compactenum}
\end{definition}
\end{omgroup}
\end{module}

\setbox0=\hbox{\begin{minipage}{5.3cm}\def\baselinestretch{.975}
\begin{lstlisting}[label=omvsmom,language=OpenMath,frame=none,numbers=none,
    index={OMBIND,OMS,OMBVAR,OMV,OMATTR,OMATP}]
<OMBIND>                          
 <OMS cd="quant1" name="forall"/> 
 <OMBVAR>                         
  <OMATTR>                        
   <OMATP>                        
    <OMS cd="sts" name="type"/>   
    <OMS cd="setname1" name="R"/>  
   </OMATP>                       
   <OMV name="a"/>                
  </OMATTR> 

                       
  <OMATTR>                        
   <OMATP>                        
    <OMS cd="sts" name="type"/>   
    <OMS cd="setname1" name="R"/>  
   </OMATP>                       
   <OMV name="b"/>                
  </OMATTR>                        
 </OMBVAR>                        
  <OMA>                           
   <OMS cd="relation" name="eq"/> 
   <OMA>                          
    <OMS cd="arith1" name="plus"/>
    <OMV name="a"/>               
    <OMV name="b"/>               
   </OMA>                         
   <OMA>                          
    <OMS cd="arith1" name="plus"/>
    <OMV name="b"/>               
    <OMV name="a"/>               
   </OMA>                         
 </OMA>                           
</OMBIND>                         
\end{lstlisting}
\end{minipage}}
\setbox1=\hbox{\begin{minipage}{7.5cm}
 \begin{lstlisting}[label=omvsmm,language=MathML,frame=none,numbers=none,index={math,apply,forall,bvar,ci,eq,plus}]
<m:apply>  
 <m:forall/>
 <m:bvar>
  <m:semantics>
   <m:ci>a</m:ci>
   <m:annotation-xml 
     definitionURL="http://www.openmath.org/cd/sts#type">
    <m:csymbol cd="setname1">R</m:csymbol>
   </m:annotation-xml>
  </m:semantics>
 </m:bvar>
 <m:bvar>
  <m:semantics>
   <m:ci>a</m:ci>
   <m:annotation-xml 
     definitionURL="http://www.openmath.org/cd/sts#type">
    <m:csymbol cd="setname1">R</m:csymbol>
   </m:annotation-xml>
  </m:semantics>
 </m:bvar>
 <m:apply>
  <m:csymobl cd="relation1">eq</m:csymbol>
  <m:apply>
   <m:csymbol cd="arith1">plus</m:csymbol>
   <m:ci>a</m:ci>
   <m:ci>b</m:ci>
  </m:apply>
  <m:apply>
   <m:csymbol cd="arith1">plus</m:csymbol>
   <m:ci>b</m:ci>
   <m:ci>a</m:ci>
  </m:apply>
 </m:apply>
</m:apply>
\end{lstlisting}
\end{minipage}}
\begin{myfig}{om-commutativity}{{\openmath} vs. C-{\mathml} for Commutativity}
\begin{tabular}{cc}
  {\large\openmath}  &  {\large\mathml}\\
  \fbox{\box0} & \fbox{\box1}
\end{tabular}
\end{myfig}
\end{omgroup}

%%% Local Variables:
%%% mode: latex
%%% TeX-master: t
%%% End:
