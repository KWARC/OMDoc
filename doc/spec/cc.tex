%%%%%%%%%%%%%%%%%%%%%%%%%%%%%%%%%%%%%%%%%%%%%%%%%%%%%%%%%%%%%%%%%%%%%%%%%
% This file is part of the LaTeX sources of the OMDoc 1.6 specification
% Copyright (c) 2015 Michael Kohlhase
% This work is licensed by the Creative Commons Share-Alike license
% see http://creativecommons.org/licenses/by-sa/2.5/ for details
% The source original is at https://github.com/KWARC/OMDoc/doc/spec 
%%%%%%%%%%%%%%%%%%%%%%%%%%%%%%%%%%%%%%%%%%%%%%%%%%%%%%%%%%%%%%%%%%%%%%%%%

\begin{module}[id=cc]
\begin{omgroup}[id=creativecommons,short=Managing Rights]
                           {Managing Rights by Creative Commons Licenses}

The Dublin Core vocabulary provides the \element[ns-elt=dc]{rights} element for
information about rights held in and over the document or element content, but leaves the
content model unspecified. While it is legally sufficient to describe this information in
natural language, a content markup format like \omdoc should support a
machine-understandable format. As one of the purposes of the \omdoc format is to support
the sharing and re-use of mathematical content, \omdoc provides markup for rights
management via the {\twinalt{Creative Commons}{Creative Commons}{license}}
({\twinalt{CC}{CC}{license}}) licenses.  \atwinalt{Digital rights
  management}{digital}{rights}{management} (\indextoo{DRM}) and licensing of
{\twintoo{intellectual}{property}} has become a hotly debated topic in the last years. We
feel that the {\twintoo{Creative Commons}{license}s} that encourage sharing of content and
enhance the (scientific) public domain while giving authors some control over their
intellectual property establish a good middle ground. Specifying rights is important,
since in the absence of an explicit or implicit (via inheritance)
\element[ns-elt=dc]{rights} element no assumptions can be made about the status of the
document or fragment.  Therefore \omdoc adds another child to the \element{metadata}
element.  

This \element[ns-elt=cc]{license} element is a symbolic representation of the Creative
Commons legal framework, adapted to the \omdoc setting: The Creative Commons Metadata
Initiative specifies various ways of embedding {\twintoo{CC}{metadata}} into documents and
{\twintoo{electronic}{artefacts}} like {\indextoo{picture}s} or
{\twintoo{MP3}{recording}s}. As \omdoc is a source format, from which various
presentation formats are generated, we need a content representation of the CC metadata
from which the end-user representations for the respective formats can be generated.

\begin{presonly}
\begin{myfig}{cctable}{The \omdoc Elements for Creative Commons Metadata}
\begin{scriptsize}
\begin{tabular}{|>{\tt}l|>{\tt}l|>{\tt}p{2.2truecm}|>{\tt}l|}\hline
{\rm Element}& \multicolumn{2}{l|}{Attributes\hspace*{2.25cm}} & Content  \\\hline
             & {\rm Req.}  & {\rm Optional}    &          \\\hline\hline
 cc:license      & & jurisdiction    &  permissions, prohibitions, requirements,h:p*  \\\hline
 cc:permissions  & & reproduction, distribution, derivative\_works & h:p*\\\hline
 cc:prohibitions & & commercial\_use & h:p* \\\hline
 cc:requirements & & notice, copyleft,  attribution & h:p* \\\hline
\end{tabular}
\end{scriptsize}
\end{myfig}
\end{presonly}

\begin{definition}[id=cc.license.def]
  The Creative Commons Metadata Initiative~\cite{CC:on} divides the license
  characteristics in three types: \defi{permissions}, \defi{prohibitions} and
  \defi{requirements}, which are represented by the three elements, which can occur as
  children of the {\eldef[ns-elt=cc]{license}} element. After these, a a natural language
  explanation of the license grant in a math text (see \sref{mtext}). The \element[ns-elt=cc]{license}
  element has two optional arguments:
\begin{description}
\item[{\attribute[ns-elt=cc]{jurisdiction}{license}}] which allows to specify the country
  in whose jurisdiction the license will be enforced\footnote{The {\twintoo{Creative
        Commons}{Initiative}} is currently in the process of adapting their licenses to
    jurisdictions other than the USA, where the licenses
    originated. See~\cite{creativecommonsWorldwide:on} for details and to check for
    progress.}. It's value is one of the {\twintoo{top-level}{domain}} codes of the
  ``Internet Assigned Names Authority (IANA)''~\cite{TLD:on}. If this attribute is
  absent, then the original US version of the license is assumed.
\item[{\attribute[ns-elt=cc]{version}{license}}] which allows to specify the version of the
  license. If the attribute is not present, then the newest released version is assumed
  (version 2.0 at the time of writing this {\report})
\end{description}
\end{definition}

The following three elements can occur as children of the \element[ns-elt=cc]{license}
element; their attribute specify the rights bestowed on the user by the license.  All
these elements have the {\twinalt{namespace}{Creative Commons}{namespace}}
\url{http://creativecommons.org/ns}, for which we traditionally use the
{\twintoo{namespace}{prefix}} {\snippetin{cc:}}. All three elements can contain a natural
language explanation of their particular contribution to the license grant in a sequence
of \element[ns-elt=h]{p} elements.

\begin{definition}[id=cc.permissions.def,title=Permissions]
  {\eldef[ns-elt=cc]{permissions}} are the rights granted by the license, to model them
  the element has three attributes, which can have the values
  {\attvalveryshort{permitted}} (the permission is granted by the license) and
  {\attvalveryshort{prohibited}} (the permission isn't):
\begin{scriptsize}
  \begin{center}
    \begin{tabular}{|l|p{6truecm}|>{\tt}l|}\hline
      Attribute & Permission & Default\\\hline\hline
      \attribute[ns-elt=cc]{reproduction}{permissions} 
      & the work may be reproduced & permitted\\\hline
      \attribute[ns-elt=cc]{distribution}{permissions}  
      & the work may be distributed, publicly displayed, and
      publicly performed & permitted \\\hline
      \attribute[ns-elt=cc]{derivative\_works}{permissions}  
      & derivative works may be created and reproduced & permitted \\\hline
    \end{tabular}
  \end{center}
\end{scriptsize}
\end{definition}

\begin{definition}[id=cc.prohibitions.def,title=Prohibitions]
  {\eldef[ns-elt=cc]{prohibitions}} are the things the license prohibits.
 \begin{scriptsize}
   \begin{center}
    \begin{tabular}{|l|p{6truecm}|>{\tt}l|}\hline
      Attribute & Prohibition & Default\\\hline\hline
      \attribute[ns-elt=cc]{commercial\_use}{permission} 
      &  stating that rights may be exercised for commercial purposes.
      & permitted \\\hline
    \end{tabular}
  \end{center}
\end{scriptsize}
\end{definition}

\begin{definition}[id=cc.requirements.def,title=Requirements]
  {\eldef[ns-elt=cc]{requirements}} are restrictions imposed by the license.
   \begin{scriptsize}
     \begin{center}
      \begin{tabular}{|l|p{6.5truecm}|>{\tt}l|}\hline
      Attribute & Requirement & Default\\\hline\hline
      \attribute[ns-elt=cc]{notice}{requirements}  
      & copyright and license notices must be kept intact & required \\\hline
      \attribute[ns-elt=cc]{attribution}{requirements}  
      & credit must be given to copyright holder and/or author & required\\\hline
      \attribute[ns-elt=cc]{copyleft}{requirements}  
      & derivative works, if authorized, must be licensed under the same terms as
      the work & required \\\hline
    \end{tabular}
  \end{center}
\end{scriptsize}
\end{definition}

This vocabulary is directly modeled after the Creative Commons
Metadata~\cite{creativecommonsMetadata:on} which defines the meaning, and provides an
{\rdf}~\cite{LasSwi:rdf99} based implementation. As we have discussed in
\sref{docmetadata}, \omdoc follows an approach that specifies metadata in the document
itself; thus we have provided the elements described here. In contrast to many other
situations in \omdoc, the rights model is not extensible, since only the current model is
backed by legal licenses provided by the creative commons initiative.

{\Mylstref{ccc-copyleft}} specifies a license grant using the Creative Commons
``share-alike'' license: The copyright is retained by the author, who licenses the content
to the world, allowing others to reproduce and distribute it without restrictions as long
as the copyright notice is kept intact. Furthermore, it allows others to create derivative
works based on the content as long as it attributes the original work of the author and
licenses the derived work under the identical license (i.e. the Creative Commons
``share-alike'' as well).
\begin{lstlisting}[label=lst:ccc-copyleft,caption={A Creative Commons License},
  index={metadata,dc:rights,license,permissions,reproduction,distribution,
%         derivative_works,commercial_use,
         notice,copyleft,attribution,prohibitions,requirements}]
<metadata>
  <dc:rights>Copyright (c) 2004 Michael Kohlhase</dc:rights>
  <license jurisdiction="de" xmlns="http://creativecommons.org/ns">
    <permissions reproduction="permitted" distribution="permitted" 
                 derivative_works="permitted"/>
    <prohibitions commercial_use="permitted"/>
    <requirements notice="required" copyleft="required" attribution="required"/>
  </license>
</metadata>
\end{lstlisting}
\end{omgroup}
\end{module}

%%% Local Variables: 
%%% mode: latex
%%% TeX-master: "main"
%%% End: 

% LocalWords: DCperson trl dateTime CC YY DD hh ss sss ISBN ISSN isbn IPR dc LocalWords:
% MiKo aut clb edt ths trc lst sec qtmetadata lang mathescape Req LocalWords: mtext
% camelcase natlist en fr nl creativecommons DRM LocalWords: cctable comercial RDF CNX
% NNat mtxt ref xmlns inheritancea ns un LocalWords: inheritanceb elt attr Dataset
% omdoc metadata YYYY de eBook LocalWords: creativecommons dc CC DRM cc Req IANA RDF ccc
% lst xmlns LocalWords: th
