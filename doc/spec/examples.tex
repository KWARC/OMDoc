%%%%%%%%%%%%%%%%%%%%%%%%%%%%%%%%%%%%%%%%%%%%%%%%%%%%%%%%%%%%%%%%%%%%%%%%%
% This file is part of the LaTeX sources of the OMDoc 1.6 specification
% Copyright (c) 2006 Michael Kohlhase
% This work is licensed by the Creative Commons Share-Alike license
% see http://creativecommons.org/licenses/by-sa/2.5/ for details
% The source original is at https://github.com/KWARC/OMDoc/doc/spec 
%%%%%%%%%%%%%%%%%%%%%%%%%%%%%%%%%%%%%%%%%%%%%%%%%%%%%%%%%%%%%%%%%%%%%%%%%

\begin{omgroup}[id=examples]{The Listings of the Primer Examples}

  We list the full text of the examples discussed in the \omdoc primer and
  specification.

\begin{omgroup}[id=examples.algebra]{Bourbaki's Algebra  Fragment}

\begin{omgroup}[id=examples.algebra1]{Top-Level Markup}
\lstinputlisting[nolol,index={DOCTYPE,omdoc,metadata,dc:title,dc:creator,dc:date,dc:description,
  dc:source,dc:type,dc:format,theory,symbol,definition,omtext,omdoc,example}]
  {../../examples/spec/combined/algebra1.omdoc}
\end{omgroup}

\begin{omgroup}[id=examples.algebra2]{Formula Markup}
\lstinputlisting[nolol,index={DOCTYPE,omdoc,metadata,dc:title,dc:creator,dc:date,dc:description,
  dc:source,dc:type,dc:format,theory,symbol,definition,omtext,omdoc,example,
  presentation,OMA,OMS,OMBIND,OMBVAR}]
  {../../examples/spec/combined/algebra2.omdoc}
\end{omgroup}

\begin{omgroup}[id=examples.algebra3]{Full Formalization}
\lstinputlisting[nolol,index={DOCTYPE,omdoc,metadata,dc:title,dc:creator,dc:date,dc:description,
  dc:source,dc:type,dc:format,theory,symbol,definition,omtext,omdoc,example,
  presentation,OMA,OMS,OMBIND,OMBVAR}]
  {../../examples/spec/combined/algebra3.omdoc}
\end{omgroup}

\begin{omgroup}[id=examples.background]{The Background Theories}
\lstinputlisting[nolol,index={DOCTYPE,omdoc,metadata,dc:title,dc:creator,dc:date,dc:description,
  dc:source,dc:type,dc:format,theory,symbol,definition,omtext,omdoc,example,
  presentation,OMA,OMS,OMBIND,OMBVAR}]
  {../../examples/spec/combined/background.omdoc}
\end{omgroup}
\end{omgroup}

\begin{omgroup}[id=examples.arith1]{A Fragment from a Content Dictionary}

\lstinputlisting[nolol,index={DOCTYPE,omdoc,metadata,dc:title,dc:creator,dc:date,dc:description,
  dc:source,dc:type,dc:format,theory,symbol,definition,omtext,omdoc,example,
  presentation,OMA,OMS,OMBIND,OMBVAR}]
  {../../examples/spec/combined/arith1.omdoc}
\end{omgroup}

\begin{omgroup}[id=examples.courseware]{Courseware}

This example comes in two parts, we first list the data-structured document and
then the narrative-structured one.
\lstinputlisting[nolol,index={DOCTYPE,omdoc,metadata,dc:title,dc:creator,dc:date,dc:description,
  dc:source,dc:type,dc:format,theory,symbol,definition,omtext,omdoc,example,
  presentation,OMA,OMS,OMBIND,OMBVAR}] {../../examples/spec/combined/15-211-thy.omdoc}

\lstinputlisting[nolol,index={DOCTYPE,omdoc,metadata,dc:title,dc:creator,dc:date,dc:description,
  dc:source,dc:type,dc:format,theory,symbol,definition,omtext,omdoc,example,
  presentation,OMA,OMS,OMBIND,OMBVAR}]
  {../../examples/spec/combined/15-211-narrative.omdoc}
\end{omgroup}

\begin{omgroup}[id=examples.natlist,short=Lists of Natural Numbers]
                           {A Parameterized Theory  of Lists of Natural Numbers}

\lstinputlisting[nolol,index={DOCTYPE,omdoc,metadata,dc:title,dc:creator,dc:date,dc:description,
  dc:source,dc:type,dc:format,theory,symbol,definition,omtext,omdoc,example,
  presentation,OMA,OMS,OMBIND,OMBVAR}]
  {../../examples/spec/combined/natlist.omdoc}
\end{omgroup}
\end{omgroup}
%%% Local Variables: 
%%% mode: latex
%%% TeX-master: "main"
%%% End: 

% LocalWords:  arith dc natlist nolol
