%%%%%%%%%%%%%%%%%%%%%%%%%%%%%%%%%%%%%%%%%%%%%%%%%%%%%%%%%%%%%%%%%%%%%%%%%
% This file is part of the LaTeX sources of the OMDoc 1.6 specification
% Copyright (c) 2006 Michael Kohlhase
% This work is licensed by the Creative Commons Share-Alike license
% see http://creativecommons.org/licenses/by-sa/2.5/ for details
% The source original is at https://github.com/KWARC/OMDoc/doc/spec 
%%%%%%%%%%%%%%%%%%%%%%%%%%%%%%%%%%%%%%%%%%%%%%%%%%%%%%%%%%%%%%%%%%%%%%%%%

\begin{module}[id=dc-elements]
  \begin{omgroup}[id=dc-elements]{Dublin Core Metadata in \omdoc}

The most commonly used metadata standard is the Dublin Core vocabulary, which is
supported in some form by most formats. \omdoc uses this vocabulary for compatibility
with other metadata applications and extends it for {\twintoo{document}{management}}
purposes in \omdoc.  Most importantly \omdoc extends the use of metadata from
documents to other (even mathematical) elements and {\twintoo{document}{fragment}s} to
ensure a fine-grained authorship and {\twintoo{rights}{management}}.


\begin{newpart}{MK@CL, please discuss}
\begin{omgroup}{Dublin Core Ontology}

\end{omgroup}

\begin{omgroup}{Pragmatic Dublin Core Elements}

In the following we will describe the variant of Dublin Core metadata elements used in
\omdoc.  Here, the \element{metadata} element can contain any number of instances of
any Dublin Core elements described below in any order. In fact, multiple instances of the
same element type (multiple \element[ns-elt=dc]{creator} elements for example) can be
interspersed with other elements without change of meaning.  \omdoc extends the Dublin
Core framework with a set of roles (from the MARC relator set~\cite{Marc:relators03}) on
the authorship elements and with a rights management system based on the Creative Commons
Initiative.


The descriptions in this section are adapted from~\cite{DCMI:dmt03}, and augmented for the
application in \omdoc where necessary. All these elements live in the {\twintoo{Dublin
    Core}{namespace}} \url{http://purl.org/dc/elements/1.1/}, for which we traditionally
use the {\twintoo{namespace}{prefix}}\atwinalt {\snippetin{dc:}}{Dublin
  Core}{namespace}{URI}.
\begin{myfig}{qtdc}{Dublin Core Metadata in \omdoc}
  \begin{scriptsize}
\begin{tabular}{|>{\tt}l|>{\tt}l|>{\tt}l|>{\tt}l|}\hline
{\rm Element}& \multicolumn{2}{l|}{Attributes\hspace*{2.25cm}} & Content  \\\hline
             & {\rm Req.}  & {\rm Optional}     &           \\\hline\hline
 dc:creator     &  & xml:id, class, style, role    &  ANY \\\hline
 dc:contributor &  & xml:id, class, style, role    &  ANY\\\hline
 dc:title       &  & xml:lang    &  \llquote{math vernacular}  \\\hline
 dc:subject     &  & xml:lang    &  \llquote{math vernacular}  \\\hline
 dc:description &  & xml:lang    &  \llquote{math vernacular}  \\\hline
 dc:publisher   &  & xml:id, class, style          &  ANY  \\\hline
 dc:date        &  & action, who &  {\twintoo{ISO}{8601}}  \\\hline
 dc:type        &  &             &  {\rm fixed:} "Dataset" {\rm or\ } "Text" \\\hline
 dc:format      &  &             &  {\rm fixed:} "application/omdoc+xml"  \\\hline
 dc:identifier  &  & scheme      &  ANY  \\\hline
 dc:source      &  &             &  ANY  \\\hline
 dc:language    &  &             &  {\twintoo{ISO}{639}} \\\hline
 dc:relation    &  &             &  ANY  \\\hline
 dc:rights      &  &             &  ANY  \\\hline\hline
 \multicolumn{4}{|l|}{for \llquote{math vernacular} see {\sref{mtext}}}\\\hline
\end{tabular}
\end{scriptsize}
\end{myfig}

\begin{definition}[id=dc.title.def,title=Titles]
  The title of the element --- note that \omdoc metadata can be specified at multiple
  levels, not only at the document level, in particular, the Dublin Core
  {\eldef[ns-elt=dc]{title}} element can be given to assign a title to a theorem, e.g. the
  ``Substitution Value Theorem''.
  
  The \element[ns-elt=dc]{title} element can contain
  {\twintoo{mathematical}{vernacular}} (see {\sref{mtext}}). Multiple
  \element[ns-elt=dc]{title} elements inside a \element{metadata} element are assumed
  to be translations of each other.\ednote{do we still want this?}
\end{definition}

\begin{definition}[id=dc.creator.def,title=Creators]
  A primary creator or author of the publication.  Additional contributors whose
  contributions are secondary to those listed in {\eldef[ns-elt=dc]{creator}} elements
  should be named in \element[ns-elt=dc]{contributor} elements.  Documents with multiple
  co-authors should provide multiple \element[ns-elt=dc]{creator} elements, each
  containing one author.  The order of \element[ns-elt=dc]{creator} elements is presumed
  to define the order in which the creators' names should be presented.
  
  As markup for names across cultures is still un-standardized, \omdoc recommends that
  the content of a \element[ns-elt=dc]{creator} element consists in a single name (as it
  would be presented to the user). The \element[ns-elt=dc]{creator} element has an
  optional attribute \attribute[ns-elt=xml,ns-attr=dc]{id}{creator} so that it can be
  {\indextoo{cross-reference}d} and a {\attributeshort{role}} attribute to further
  classify the concrete contribution to the element. We will discuss its values in
  {\sref{dc-roles}}.
\end{definition}

\begin{definition}[id=dc.contributor.def,title=Contributors]
  A party whose contribution to the publication is secondary to those named in
  \element[ns-elt=dc]{creator} elements.  Apart from the significance of contribution,
  the semantics of the {\eldef[ns-elt=dc]{contributor}} is identical to that of
  \element[ns-elt=dc]{creator}, it has the same restriction content and carries the same
  attributes plus a \attribute[ns-elt=xml,ns-attr=dc]{lang}{contributor} attribute that
  specifies the target language in case the contribution is a translation.
\end{definition}

\begin{definition}[id=dc.subject.def,title=Subjects]
  This element contains an arbitrary phrase or keyword, the attribute
  \attribute[ns-elt=xml,ns-attr=dc]{lang}{subject} is used for the language. Multiple
  instances of the {\eldef[ns-elt=dc]{subject}} element are supported per
  \attribute[ns-elt=xml,ns-attr=dc]{lang}{subject} for multiple keywords.
\end{definition}

\begin{definition}[id=dc.description.def,title=Descriptions]
  A text describing the containing element's content; the attribute
  \attribute[ns-elt=xml,ns-attr=dc]{lang}{description} is used for the language. As
  description of mathematical objects or \omdoc fragments may contain formulae, the
  content of this element is of the form ``{\twintoo{mathematical}{text}}'' described in
  {\sref{mtext}}.
\end{definition}

\begin{definition}[id=dc.publisher.def,title=Publishers]
  The entity for making the document available in its present form, such as a publishing
  house, a university department, or a corporate entity. The
  {\eldef[ns-elt=dc]{publisher}} element only applies if the \element{metadata} is a
  direct child of the root element (\element{omdoc}) of a
  \twinalt{document}{document}{root}.
\end{definition}

\begin{definition}[id=dc.date.def,title=Dates]
  The date and time a certain action was performed on the element that contains this. The
  content is in the format defined by {\xml} Schema data type {\snippetin{date\-Time}}
  (see~\cite{BirMal:XMLSchema:Datatypes} for a discussion), which is based on the
  {\atwintoo{ISO}{8601}{norm}} for dates and times.

  Concretely, the format is
  {\snippet{\llquote{YYYY}-\llquote{MM}-\llquote{DD}T\llquote{hh}:\llquote{mm}:\llquote{ss}}}
  where {\llquote{YYYY}} represents the year, {\llquote{MM}} the month, and {\llquote{DD}}
  the day, preceded by an optional leading ``{\snippet{-}}'' sign to indicate a negative
  number. If the sign is omitted, ``{\snippet{+}}'' is assumed.  The letter
  ``{\snippet{T}}'' is the date/time separator and {\llquote{hh}}, {\llquote{mm}},
  {\llquote{ss}} represent hour, minutes, and seconds respectively.  Additional digits can
  be used to increase the precision of fractional seconds if desired, i.e the format
  {\snippet{\llquote{ss}.\llquote{sss\ldots}}} with any number of digits after the decimal
  point is supported.  The {\eldef[ns-elt=dc]{date}} element has the attributes
  \attribute[ns-elt=dc]{action}{date} and \attribute[ns-elt=dc]{who}{date} to specify
  who did what: The value of \attribute[ns-elt=dc]{who}{date} is a reference to a
  \element[ns-elt=dc]{creator} or \element[ns-elt=dc]{contributor} element and
  \attribute[ns-elt=dc]{action}{date} is a keyword for the action
  undertaken. Recommended values include the short forms
  \attval[ns-elt=dc]{updated}{action}{date},
  \attval[ns-elt=dc]{created}{action}{date},
  \attval[ns-elt=dc]{imported}{action}{date},
  \attval[ns-elt=dc]{frozen}{action}{date},
  \attval[ns-elt=dc]{review-on}{action}{date},
  \attval[ns-elt=dc]{normed}{action}{date} with the obvious meanings. Other actions may
  be specified by {\indextoo{URI}s} pointing to documents that explain the action.
\end{definition}

\begin{definition}[id=dc.type.def,title=Types]
  Dublin Core defines a vocabulary for the document types in {\cite{DCMI:dtv03}}. The best
  fit values for \omdoc are
  \begin{description}
  \item[{\snippetin{Dataset}}] defined as ``{\presem{information encoded in a defined
        structure (for example lists, tables, and databases), intended to be useful for
        direct machine processing}}.''
  \item[{\snippetin{Text}}] ``{\presem{a resource whose content is primarily words for
        reading. For example -- books, letters, dissertations, poems, newspapers,
        articles, archives of mailing lists. Note that facsimiles or images of texts are
        still of the genre text.}}''
  \item[{\snippetin{Collection}}] defined as ``{\presem{an aggregation of items. The term
        collection means that the resource is described as a group; its parts may be
        separately described and navigated}}''.
  \end{description}
  The more appropriate should be selected for the element that contains the
  {\eldef[ns-elt=dc]{type}}. If it consists mainly of formal mathematical formulae, then
  {\snippetin{Dataset}} is better, if it is mainly given as text, then {\snippetin{Text}}
  should be used. More specifically, in \omdoc the value {\snippetin{Dataset}} signals
  that the order of children in the parent of the \element{metadata} is not relevant to
  the meaning. This is the case for instance in formal developments of mathematical
  theories, such as the specifications in {\sref{complex-theories}}.
\end{definition}

\begin{definition}[id=dc.format.def,title=Formats]
  The physical or digital manifestation of the resource.  Dublin Core suggests using
  {\twintoo{MIME}{type}s}~\cite{FreBor:MIME96}.  Following~\cite{MurLau:xmt01} we fix the
  content of the {\eldef[ns-elt=dc]{format}} element to be the string
  {\snippet{application/omdoc+xml}} as the {\twintoo{MIME}{type}} for \omdoc.
\end{definition}

\begin{definition}[id=dc.identifier.def,title=Identifiers]
  A string or number used to uniquely identify the element.  The
  {\eldef[ns-elt=dc]{identifier}} element should only be used for public identifiers like
  {\indextoo{ISBN}} or {\indextoo{ISSN}} numbers. The numbering scheme can be specified in
  the \attribute[ns-elt=dc]{scheme}{identifier} attribute.
\end{definition}

\begin{definition}[id=dc.source.def,title=Sources]
  Information regarding a prior resource from which the publication was derived. We
  recommend using either a {\indextoo{URI}} or a scientific reference including
  identifiers like ISBN numbers for the content of the {\eldef[ns-elt=dc]{source}}
  element.
\end{definition}

\begin{definition}[id=dc.relation.def,title=Relations]
  Relation of this document to others.  The content model of the
  {\eldef[ns-elt=dc]{relation}} element is not specified in the \omdoc format.
\end{definition}

\begin{definition}[id=dc.language.def,title=Languages]
  If there is a primary language of the document or element, this can be specified
  here. The content of the {\eldef[ns-elt=dc]{language}} element must be an
  {\atwintoo{ISO}{639}{norm}} two-letter language specifier, like
  {\snippetin{en}}$\;\widehat=\;$English, {\snippetin{de}}$\;\widehat=\;$German,
  {\snippetin{fr}}$\;\widehat=\;$French, {\snippetin{nl}}$\;\widehat=\;$Dutch, \ldots.
\end{definition}

\begin{definition}[id=dc.rights.def,title=Rights]
  Information about rights held in and over the document or element content or a reference
  to such a statement. Typically, a {\eldef[ns-elt=dc]{rights}} element will contain a
  rights management statement, or reference a service providing such
  information. \element[ns-elt=dc]{rights} information often encompasses Intellectual
  Property rights (IPR), Copyright, and various other property rights. If the
  \element[ns-elt=dc]{rights} element is absent (and no \element[ns-elt=dc]{rights}
  information is inherited), no assumptions can be made about the status of these and
  other rights with respect to the document or element.
  
  \omdoc supplies specialized elements for the Creative Commons licenses to support the
  sharing of mathematical content. We will discuss them in {\sref{creativecommons}}.
\end{definition}

Note that Dublin Core also defines a {\oldelement{Coverage}{1.1}} element that specifies
the place or time which the publication's contents addresses. This does not seem
appropriate for the mathematical content of \omdoc, which is largely independent of time
and geography.

\begin{omgroup}[id=dc-roles]{Roles in Dublin Core Elements}

Because the Dublin Core metadata fields for \element[ns-elt=dc]{creator} and
\element[ns-elt=dc]{contributor} do not distinguish roles of specific parties (such as
author, editor, and illustrator), we will follow the {\indextoo{Open eBook}}
specification~\cite{OpenEBook:oeps99} and use an optional \attribute[ns-elt=dc]{role}{*}
attribute for this purpose, which is adapted for \omdoc from the MARC relator code
list~\cite{Marc:relators03}.
\begin{description}
\item[{\attval[ns-elt=dc]{aut}{role}{*}}] ({\indextoo{author}}) Use for a
  person or corporate body chiefly responsible for the intellectual
  content of an element. This term may also be used when more than one person or body
  bears such responsibility.
\item[{\attval[ns-elt=dc]{ant}{role}{*}}]
  (\twinalt{bibliographic}{bibliographic}{antecedent}/\twinalt{scientific}{scientific}{antecedent}
  antecedent) Use for the author responsible for a work upon which the element is based.
\item[{\attval[ns-elt=dc]{clb}{role}{*}}] ({\indextoo{collaborator}}) Use
  for a person or corporate body that takes a limited part in the elaboration of a
  work of another author or that brings complements (e.g., appendices, notes) to
  the work of another author.
\item[{\attval[ns-elt=dc]{edt}{role}{*}}] ({\indextoo{editor}}) Use for a
  person who prepares a document not primarily his/her own for publication, such
  as by elucidating text, adding introductory or other critical matter, or
  technically directing an editorial staff.
\item[{\attval[ns-elt=dc]{ths}{role}{*}}] ({\twintoo{thesis}{advisor}}) Use for the person under
  whose supervision a degree candidate develops and presents a thesis, memoir, or text of
  a dissertation.
\item[{\attval[ns-elt=dc]{trc}{role}{*}}] ({\indextoo{transcriber}}) Use
  for a person who prepares a handwritten or typewritten copy from original
  material, including from dictated or orally recorded material. This is also the
  role (on the \element[ns-elt=dc]{creator} element) for someone who prepares the \omdoc
  version of some mathematical content.
\item[{\attval[ns-elt=dc]{trl}{role}{*}}] ({\indextoo{translator}}) Use
  for a person who renders a text from one language into another, or from an older
  form of a language into the modern form. The target language can be specified by
  \attribute[ns-elt=xml,ns-attr=dc]{lang}{*}.
\end{description}
As \omdoc documents are often used to formalize existing mathematical texts for use in
mechanized reasoning and computation systems, it is sometimes subtle to specify
authorship.  We will discuss some typical examples to give a guiding intuition.
{\mylstref{sec-edt}} shows metadata for a situation where editor $R$ gives the sources
(e.g. in {\LaTeX}) of an element written by author $A$ to secretary $S$ for conversion
into \omdoc format.
\begin{lstlisting}[label=lst:sec-edt,mathescape,
  caption={A Document with Editor ({\snippet{edt}}) and  Transcriber ({\snippet{trc}})},
  index={metadata,dc:title,dc:creator,dc:contributor}]
<metadata>
  <dc:title>The Joy of Jordan $C\sp{*}$ Triples</dc:title>
  <dc:creator role="aut">$A$</dc:creator>
  <dc:contributor role="edt">$R$</dc:contributor>
  <dc:contributor role="trc">$S$</dc:contributor>
</metadata>
\end{lstlisting}

In {\mylstref{formalize}} researcher $R$ formalizes the theory of natural numbers
following the standard textbook $B$ (written by author $A$). In this case we
recommend the first declaration for the whole document and the second one for
specific math elements, e.g. a definition inspired by or adapted from one in book
$B$.

\begin{lstlisting}[label=lst:formalize,mathescape,
  caption={A Formalization with Scientific Antecedent ({\snippet{ant}})},
  index={metadata,dc:title,dc:creator}]
<omdoc xml:id="NNat" version="1.6" xmlns:dc="http://purl.org/dc/elements/1.1/">
  <metadata><dc:title>Natural Numbers</dc:title></metadata>                              
  $\ldots$
  <theory xml:id="NNat.thy">
    <metadata>
      <dc:title>Natural Numbers</dc:title>
      <dc:creator role="aut">$R$</dc:creator>
      <dc:contributor role="ant">$A$</dc:contributor>
      <dc:source>$B$</dc:source>
    </metadata>
    $\ldots$
  </theory>
  $\ldots$
</omdoc>
\end{lstlisting} 
\end{omgroup}
\end{omgroup}
\end{newpart}
\end{omgroup}
\end{module}

%%% Local Variables: 
%%% mode: latex
%%% TeX-master: "main"
%%% End: 

% LocalWords: DCperson trl dateTime CC YY DD hh ss sss ISBN ISSN isbn IPR dc LocalWords:
% MiKo aut clb edt ths trc lst sec qtmetadata lang mathescape Req LocalWords: mtext
% camelcase natlist en fr nl creativecommons DRM LocalWords: cctable comercial RDF CNX
% NNat mtxt ref xmlns inheritancea ns un LocalWords: inheritanceb elt attr Dataset
% omdoc metadata YYYY de eBook LocalWords: creativecommons dc CC DRM cc Req IANA RDF ccc
% lst xmlns LocalWords: th
