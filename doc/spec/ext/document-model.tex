%%%%%%%%%%%%%%%%%%%%%%%%%%%%%%%%%%%%%%%%%%%%%%%%%%%%%%%%%%%%%%%%%%%%%%%%%
% This file is part of the LaTeX sources of the OMDoc 1.6 specification
% Copyright (c) 2015 Michael Kohlhase
% This work is licensed by the Creative Commons Share-Alike license
% see http://creativecommons.org/licenses/by-sa/2.5/ for details
% The source original is at https://github.com/KWARC/OMDoc/doc/spec 
%%%%%%%%%%%%%%%%%%%%%%%%%%%%%%%%%%%%%%%%%%%%%%%%%%%%%%%%%%%%%%%%%%%%%%%%%

\begin{module}[id=document-model]
\begin{omgroup}[id=document-model]{Document Models for OMDoc}

In almost all {\xml} applications, there is a tension between the document view and the
object view of data; after all, {\xml} is a document-oriented interoperability framework
for exchanging data objects. The question, which view is the correct one for {\xml} in
general is hotly debated among {\xml} theorists. In \omdoc, actually both views make
sense in various ways. Mathematical documents are the objects we try to formalize, they
contain knowledge about mathematical objects that are encoded as formulae, and we arrive
at content markup for mathematical documents by treating knowledge fragments (statements
and theories) as objects in their own right that can be inspected and reasoned about.

\begin{omtext}
  In {\srefl{mobj}{quiz}}, we have defined what \omdoc documents look like and motivated
  this by the mathematical objects they encode. But we have not really defined the
  properties of these documents as objects themselves (\inlinedef{we will speak of the
    \omdoc \defii{document}{object model} ({\indextoo{OMDOM}})}). To get a feeling
  for the issues involved, let us take stock of what we mean by the object view of
  data. In mathematics, when we define a class of mathematical objects (e.g.  vector
  spaces), we have to say which objects belong to this class, and when they are to be
  considered equal (e.g.  vector spaces are equal, iff they are isomorphic). When defining
  the intended behavior of operations, we need to care only about objects of this class,
  and we can only make use of properties that are invariant under object equality. In
  particular, we cannot use properties of a particular realization of a vector space that
  are not preserved under isomorphism. For document models, we do the same, only that the
  objects are documents.
\end{omtext}

\begin{omgroup}[id=xml-DOM]{XML Document Models}
\begin{module}[id=XMLDom]

{\xml} supports the task of defining a particular class of documents (e.g. the class of
\omdoc documents) with formal grammars such as the {\twintoo{document type}{definition}}
({\indextoo{DTD}}) or an {\xml} \twinalt{schema}{XML}{schema}, that can be used for
mechanical document validation.  Surprisingly, {\xml} leaves the task of specifying
document equality to be clarified in the (informal) specifications, such as this \omdoc
specification. As a consequence, current practice for {\xml}
\twinalt{applications}{application}{XML} is quite varied. For instance, the {\openmath}
standard (see~\cite{BusCapCar:2oms04} and \sref{openmath}) gives a mathematical
object model for {\openmath} objects that is specified independently of the {\xml}
encoding. Other {\xml} applications like e.g.  presentation
{\mathml}~\cite{CarIon:MathML03} or {\xhtml}~\cite{W3C:xhtml2000} specify models in form
of the intended screen presentation, while still others like the
{\xslt}~\cite{Clark:xslt99} give the operational semantics.

For a formal definition
\begin{definition}[display=flow,id=document-model.def]
  let $\cK$ be a set of documents. We take a \defii{document}{model} to be a partial
  equivalence relation\footnote{A partial equivalence relation is a symmetric transitive
    relation. We will use $[d]_\cX$ for the \defii{equivalence}{class} of $d$, i.e.
    $[d]_\cX\colon=\{e|d{\cX}e\}$ $\cX$ on documents, such that $\{d|d{\cX}d\}=\cK$}. In
  particular, a relation $\cX$ is an equivalence relation on $\cK$.  For a given document
  model $\cX$, let us say that two documents $d$ and $d'$ are {\defemph{$\cX$-equal}}, iff
  $d{\cX}d'$. We call a property $p$ $\cX$-\defi{invariant}, iff for all $d{\cX}d'$,
  $p$ holds on $d$ whenever $p$ holds on $d'$.
\end{definition}

A possible source of confusion is that documents can admit more than one document model
(see~\cite{KohKoh:esmk05} for an exploration of possible document models for
mathematics). Concretely, \omdoc documents admit the \omdoc document model that we
will specify in section \sref{omdom} and also the following four {\xml} document
models that can be restricted to \omdoc documents (as a relation).\footnote{Here we
  follow Eliotte Rusty Harold's classification of layers of {\xml} processing
  in~\cite{Harold:ex03}, where he distinguishes the binary, lexical, sequence, structure,
  and semantic layer, the latter being the document model of the {\xml} application}
\begin{description}
\item[The {\twintoo{binary}{document model}}] interprets files as sequences of
  bytes.  Two documents are equal, iff they are equal as byte sequence. This
  is the most concrete and fine-grained (and thus weakest) document model
  imaginable.
\item[The {\atwintoo{lexical}{document}{model}}] interprets binary files as sequences of
  {\indextoo{Unicode}} characters~\cite{Unicode:tuc03} using an encoding table.  Two files
  may be considered equal by this document model even though they differ as binary files,
  if they have different encodings that map the byte sequences to the same sequence of
  {\unicode} characters.
\item[The \atwinalt{{\xml} syntax document model}{XML}{syntax}{document model}] interprets
  {\unicode} Files as sequences consisting of an {\xml} declaration, a DOCTYPE
  declaration, tags, entity references, character references, CDATA sections, PCDATA
  comments, and processing instructions. At this level, for instance, whitespace
  characters between {\xml} tags are irrelevant, and {\xml} documents may be considered
  the same, if they are different as {\unicode} sequences.
\item[The \atwinalt{{\xml} structure document model}{XML}{structure}{document model}]
  interprets documents as {\xml} trees of elements, attributes, text nodes, processing
  instructions, and sometimes comments. In this document model the order of attribute
  declarations in {\xml} elements is immaterial, double and single quotes can be used
  interchangeably for strings, and {\xml} \twinalt{comments}{XML}{comment}
  (\snippetin{<\char33--}\ldots\snippetin{-->}) are ignored.
\end{description}
Each of these document models, is suitable for different applications, for instance the
lexical document model is the appropriate one for {\indextoo{Unicode}}-aware editors that
interpret the encoding string in the {\xml} declaration and present the appropriate glyphs
to the user, while the binary document model would be appropriate for a simple ASCII
editor. Since the last three document models are refinements of the {\xml} document model,
we will recap this in the next section and define the \omdoc document model in
\sref{omdom}.

To get a feeling for the issues involved, let us compare the \omdoc elements in
{\mylstlref{first}{third}} below. For instance, the serialization in {\mylstref{second}}
is {\xml}-equal to the one in {\mylstref{first}}, but not to the one in
{\mylstref{third}}.

\begin{lstlisting}[escapechar=\%,label=lst:first,
   index={definition,OMA},caption={An \omdoc Definition}]
<docalt>
  <definition xml:id="comm.def.en" for="comm">
    <h:p xml:lang="en">
      An operation <OMV name="op" id="op"/> is called commutative, iff 
      <OMA id="comm1"><OMS cd="relation1" name="eq"/>
        <OMA><OMV name="op"/><OMV name="X"/><OMV name="Y"/></OMA>
        <OMA><OMV name="op"/><OMV name="Y"/><OMV name="X"/></OMA>
      </OMA>
      for all <OMV id="x" name="X"/> and <OMV id="y" name="Y"/>.
    </h:p>
  </definition
  <definition xml:id="comm.def.de" for="#comm">
    <h:p xml:lang="de">
      Eine Operation <OMR href="#op"/> hei%\ss%t kommutativ, falls 
      <OMR href="#comm1"/> f%\"u%r alle <OMR href="#x"/> und <OMR href="#y"/>.
    </h:p>
  </definition>
</docalt>
\end{lstlisting}

\begin{lstlisting}[escapechar=\%,label=lst:second,mathescape,
   index={definition,h:p},
   caption={An {\xml}-equal serialization for {\mylstref{first}}}]
<definition for="comm" xml:id="comm.def.de" >
 <h:p xml:lang='de'> <!-- Note the unabbreviated empty element -->
   Eine Operation <OMR href="#op"/>  hei%\ss%t 
   kommutativ, falls <OMR href='comm1'/> f%\"u%r alle 
   <OMR href="#x"/> und <OMR href='y'/>.
  </h:p>
</definition>
\end{lstlisting}
\end{module}
\end{omgroup}

\begin{omgroup}[id=omdom]{The OMDoc Document Model}
\begin{module}[id=omdom]

The \omdoc document model extends the {\xml} structure document model in various ways.
We will specify the equality relation in the table below, and discuss a few general issues
here.

The \omdoc document model is guided by the notion of content markup for mathematical
documents. Thus, two document fragments will only be considered equal, if they have the
same abstract structure. For instance, the order of children of an \element{docalt}
element is irrelevant, since they form a {\twintoo{multilingual}{group}} which form the
base for multilingual text assembly. Other facets of the \omdoc document model are
motivated by presentation-independence, for instance the distribution of whitespace is
irrelevant even in text nodes, to allow formatting and reflow in the source code, which is
not considered to change the information content of a text.

\begin{lstlisting}[escapechar=\%,label=lst:third,
   index={definition,docalt,OMA},
   caption={An \omdoc-Equal Representation for {\mylstsref{first}{second}}}]
<docalt>
  <definition xml:id="comm.def.de" for="comm">
    <h:p xml:lang="de">Eine Operation <OMR href="#op"/> 
      hei%\ss%t kommutativ, falls 
      <OMA id="comm1"><OMS cd="relation1" name="eq"/>
        <OMA><OMV name="op"/><OMV name="X"/><OMV name="Y"/></OMA>
        <OMA><OMV name="op"/><OMV name="Y"/><OMV name="X"/></OMA>
      </OMA>
      f%\"u%r alle <OMR href="#x"/> und <OMR href="#y"/>.
  </h:p>
  </definition>
  <definition xml:id="comm.def.en" for="#comm">
    <h:p xml:lang="en">
      An operation <OMV id="op" name="op"/> is called commutative, 
      iff <OMR href="#comm1"/> for all <OMV id="x" name="X"/> and 
      <OMV id="y" name="Y"/>.
    </h:p>
  </definition>
</docalt>
\end{lstlisting}

Compared to other document models, this is a rather weak (but general) notion of
equality. Note in particular, that the \omdoc document model does {\emph{not}}
use mathematical equality here, which would make the formula $X+Y=Y+X$ (the
\element[ns-elt=om]{OMA} with {\snippet{xml:id="comm1"}} in
{\mylstref{third}} instantiated with addition for {\snippetin{op}}) mathematically
equal to the trivial condition $X+Y=X+Y$, obtained by exchanging the right hand
side $Y+X$ of the equality by $X+Y$, which is mathematically equal (but not
\omdoc-equal).

Let us now specify (part of) the equality relation by the rules in the table in
{\myfigref{omdom}}.  We have discussed a machine-readable form of these equality
constraints in the {\xml} schema for \omdoc in~\cite{KohAng:tccmvc03}.

\begin{myfig}{omdom}{The \omdoc Document Model}
\begin{scriptsize}
\begin{tabular}{|l|p{1.2truecm}|p{5.5truecm}|p{3cm}|}\hline
  \#& Rule & comment & elements \\\hline\hline
  1 & unordered
  & The order of children of this element is irrelevant (as far as permitted by
  the content model). For instance only the order of \element{obligation}
  elements in the \element{axiom-inclusion} element  is arbitrary, since the
  others must precede them in the content model. 
  & \element{adt} \element{axiom-inclusion} \element{metadata}
  \element{symbol} \element{code} \element{private} \element{presentation}
  \element{omstyle}\\\hline
  2 & multi-group 
  & The order between siblings elements does not matter, as long as the values of
  the key attributes differ. 
  & \element{requation} \element[ns-elt=dc]{description}
  \element{sortdef} \element{data}   \element[ns-elt=dc]{title} \element{solution} \\\hline 
  3 & DAG encoding 
  & \atwinalt{Directedacyclicgraphs}{directed}{acyclic}{graph} built up using
    \element[ns-elt=om]{OMR} elements are equal, iff their {\twintoo{tree}{expansion}s}
    are equal.  
  & \element[ns-elt=om]{OMR} \element{ref}     \\\hline   
  4 & Dataset 
  & If the content of the \element[ns-elt=dc]{type} element is {\snippetin{Dataset}}, then
  the order of the siblings of the parent \element{metadata} element is
  irrelevant. 
  & \element[ns-elt=dc]{type} \\\hline
\end{tabular}
\end{scriptsize}
\end{myfig}

\begin{omtext}
  The last rule in {\myfigref{omdom}} is probably the most interesting, as we have seen in
  \sref{omdoc-infrastructure}, \omdoc documents have both formal and informal aspects,
  they can contain {\emph{\indextoo{narrative}}} as well as
  {\emph{\indextoo{narrative-structured}}} information.  The latter kind of document
  contains a formalization of a mathematical theory, as a reference for automated theorem
  proving systems. There, logical dependencies play a much greater role than the order of
  serialization in mathematical objects.  \inlinedef{We call such documents
    \defi{content OMDoc}} and specify the value {\snippetin{Dataset}} in the
  \element[ns-elt=dc]{type} element of the \omdoc metadata for such documents.  On the
  other extreme we have human-oriented presentations of mathematical knowledge, e.g.  for
  educational purposes, where didactic considerations determine the order of
  presentation. \inlinedef{We call such documents \defi{narrative-structured}} and
  specify this by the value {\snippetin{Text}} (also see the discussion in
  \sref{dc-elements})
\end{omtext}
\end{module}
\end{omgroup}

\begin{omgroup}[id=sub-languages]{OMDoc Sub-Languages}
\begin{module}[id=sub-languages]

  In the last chapters we have described the \omdoc modules. Together, they make up the
  \omdoc document format, a very rich format for marking up the content of a wide
  variety of mathematical documents. (see {\extref{primer}{all}} for some worked
  examples). Of course not all documents need the full breadth of \omdoc functionality,
  and on the other hand, not all \omdoc applications (see {\extref{projects}{all}} for
  examples) support the whole language.

One of the advantages of a modular language design is that it becomes easy to address this
situation by specifying sub-languages that only include part of the functionality. We will
discuss plausible \omdoc sub-languages and their applications that can be obtained by
dropping optional modules from \omdoc.  {\myfigref{omdoc-sub-languages}} visualizes the
sub-languages we will present in this chapter. The full language \omdoc is at the top,
at the bottom is a minimal sub-language \omdoc Basic, which only contains the required
modules (mathematical documents without them do not really make sense). The arrows signify
language inclusion and are marked with the modules acquired in the extension.

\begin{myfig}{omdoc-sub-languages}{\omdoc Sub-Languages and Modules}
 \begin{tikzpicture}
     \begin{scope}
       \tikzstyle{every node}=[draw]
       \node (basic) at (3,0) {{\sc Basic} ({\small{\MOBJmodule{spec}},
           {\DOCmodule{spec}}, {\METAmodule{spec}}, {\MTXTmodule{spec}}, {\RTmodule{spec}}})};
       \node (cd) at (3,1.5) {{\sc{Content Dictionaries}}};
       \node (mathweb) at (1,2.8) {{\sc{MathWeb}}};
       \node (edu) at (1,4.4) {{\sc{Education}}};
       \node (spec) at (5,3.5) {{\sc{Specification}}};
       \node (full) at (3,5.5) {\omdoc};
     \end{scope}
    \draw[->] (basic) -- node[left]{\footnotesize{\PRESmodule{spec}}, {\STmodule{spec}}}  (cd);
    \draw[->] (cd) -- node[left]{\footnotesize{\EXTmodule{spec}}} (mathweb);
    \draw[->] (mathweb) --  node[left] {\footnotesize{\QUIZmodule{spec}}} (edu);
    \draw[->] (edu) -- 
        node[left] {\footnotesize{\CTHmodule{spec}}, {\DGmodule{spec}},
                                   {\PFmodule{spec}}, {\ADTmodule{spec}}}
              (full); 
    \draw [->] (cd) -- 
        node[right] {\footnotesize{\CTHmodule{spec}}, {\DGmodule{spec}}, 
                                  {\PFmodule{spec}}, {\ADTmodule{spec}}}
               (spec);
    \draw[->] (spec) -- node[right] {\footnotesize{\EXTmodule{spec}}, {\QUIZmodule{spec}}} (full);
  \end{tikzpicture}
\end{myfig}

The sub-language identifiers can be used as values of the {\attributeshort{modules}}
attribute on the \element{omdoc} and \element{omdoc} elements. Used there, they
abbreviate the list of modules these sub-languages contain.


\begin{omgroup}[id=sub-languages.basic]{Basic \omdoc}

Basic \omdoc is sufficient for very simple mathematical documents that do not introduce
new symbols or concepts, or for early (and non-specific) stages in the migration process
from legacy representations of mathematical material (see \sref{top-level}). This
\omdoc sub-language consists of five modules: we need module {\MOBJmodule{spec}} for
mathematical objects and formulae, which are present in almost all mathematical documents.
Module {\DOCmodule{spec}} provides the document infrastructure, and in particular, the
root \twinalt{element}{document}{root} \element{omdoc}. We need {\METAmodule{spec}} for
titles, descriptions, and administrative metadata, and module {\MTXTmodule{spec}} so we
can state properties about the mathematical objects in \element{omtext} element.
Finally, module {\RTmodule{spec}} allows to structured text below the \element{omtext}
level. This module is not strictly needed for basic \omdoc, but we have included it for
convenience.
\end{omgroup}

\begin{omgroup}[id=sub-languages.cd]{\omdoc Content Dictionaries}

Content Dictionaries are used to define the meaning of {\indextoo{symbol}s} in the
{\openmath} standard~\cite{BusCapCar:2oms04}, they are the mathematical documents referred
to in the \attribute{cd}{OMS} attribute of the \element[ns-elt=om]{OMS} element.  To
express \twinalt{contentdictionaries}{content dictionary}{OMDoc} in \omdoc, we need to
add the module {\STmodule{spec}} to Basic \omdoc.  It provides the possibility to
specify the meaning of basic mathematical objects ({\indextoo{symbol}s}) by
{\indextoo{axiom}s} and {\indextoo{definition}s} together with the infrastructure for
inheritance, and grouping, and allows to reference the symbols defined via their home
theory (see the discussion in \sref{theories-contexts}).
  
With this extension alone, \omdoc content dictionaries add support for
{\twintoo{multilingual}{text}}, simple {\indextoo{inheritance}} for theories, and
{\twintoo{document}{structure}} to the functionality of {\openmath} content
dictionaries. Furthermore, \omdoc content dictionaries allow the conceptual separation
of mathematical properties into {\indextoo{constitutive}} ones and
{\twintoo{logically}{redundant}} ones. The latter of these are not strictly essential for
content dictionaries, but enhance maintainability and readability, they are included in
{\openmath} content dictionaries for documentation and explanation.

The sub-language for \omdoc content dictionaries also allows the specification of
notations for the introduced symbols (by module {\PRESmodule{spec}}). So the resulting
documents can be used for referencing (as in {\openmath}) and as a resource for deriving
presentation information for the symbols defined here.  To get a feeling for this
sub-language, see the example in the \omdoc variant of the {\openmath} content
dictionary {\snippetin{arith1}} in \sref{omcds}, which shows that the {\openmath}
content dictionary format is (isomorphic to) a subset of the \omdoc format. In fact, the
{\openmath}2 standard only presents the content dictionary format used here as one of many
encodings and specifies abstract conditions on content dictionaries that the \omdoc
encoding below also meets.  Thus \omdoc is a valid content dictionary encoding.
\end{omgroup}

\begin{omgroup}[id=sub-languages.spec]{Specification \omdoc}

\omdoc content dictionaries are still a relatively lightweight format for the
specification of meaning of mathematical symbols and objects. Large scale formal
specification efforts, e.g. for program verification need more structure to be
practical. Specification languages like {\casl} (Common Algebraic
\indexalt{Specification}{specification} Language~\cite{CoFI:2004:CASL-RM}) offer the necessary
infrastructure, but have a syntax that is not integrated with web standards.

The Specification \omdoc sub-language adds the modules {\ADTmodule{spec}} and
{\CTHmodule{spec}} to the language of \omdoc content dictionaries. The resulting
language is equivalent to the {\casl} standard,
see~\cite{AutHut:tefsduc00,Hutter:mocsv00,MAH-06-a} for the necessary theory.

The structured definition schemata from module {\ADTmodule{spec}} allow to specify
{\twintoo{abstract}{data type}s}, sets of objects that are inductively defined from
constructor symbols. The {\twintoo{development}{graph}} structure built on the
{\twintoo{theory}{morphism}s} from module {\CTHmodule{spec}} allow to make inclusion
assertions about theories that structure fragments of mathematical developments and
support a \twintoo{Management of}{change}.
\end{omgroup}

\begin{omgroup}[id=sub-languages.mathwebomdoc]{MathWeb \omdoc}

\omdoc can be used as a content-oriented basis for web publishing of
mathematics. Documents for the web often contain images, applets, code fragments,
and other data, together with mathematical statements and theories.

The \omdoc sub-language {\twinalt{MathWeb \omdoc}{MathWeb} {OMDoc}} extends the
language for \omdoc content dictionaries by the module {\EXTmodule{spec}}, which adds
infrastructure for images, applets, code fragments, and other data.
\end{omgroup}

\begin{omgroup}[id=sub-languages.edu]{Educational \omdoc}

\omdoc is currently used as a content-oriented basis for various systems for mathematics
education (see e.g. {\extref{primer}{courseware}} for an example and discussion).  The \omdoc
sub-language \twinalt{Educational \omdoc}{Educational}{OMDoc} extends MathWeb \omdoc
by the module {\QUIZmodule{spec}}, which adds infrastructure for exercises and
assessments.
\end{omgroup}
\end{module}

\begin{omgroup}[id=spec-intro.reusing]{Reusing \omdoc modules in other formats}

Another application of the modular language design is to share modules with other {\xml}
\twinalt{applications}{XML}{application}. For instance, formats like
{\docbook}~\cite{WalMue:dtdg99} or {\xhtml}~\cite{W3C:xhtml2000} could be extended with
the \omdoc statement level.  Including modules {\MOBJmodule{spec}}, {\METAmodule{spec}},
and (parts of) {\MTXTmodule{spec}}, but not {\RTmodule{spec}} and {\DOCmodule{spec}} would
result in content formats that mix the document-level structure of these formats.  Another
example is the combination of {\xmlrpc} envelopes and \omdoc documents used for
interoperability in {\extref{primer}{rpc}}.
\end{omgroup}
\end{omgroup}
\end{omgroup}
\end{module}
%%% Local Variables: 
%%% mode: latex
%%% TeX-master: "main"
%%% End: 

% LocalWords:  cd xxx arith omcds mathwebomdoc rpc ns omdom omdom omdom omdom
% LocalWords:  mobj OMDOM DOM openmath omdom Eliotte Harold's unicode PCDATA en
% LocalWords:  escapechar lst comm def lang cd eq Eine xref hei kommutativ alle
% LocalWords:  und basicstyle xsd adt  requation sortdef OMR Dataset href elt
% LocalWords:  dc mathescape concl omdom DOCTYPE CDATA omdom OMA OMV
% LocalWords:  de omtext omdom omdom metadata omstyle multi FMP omdom omdoc edu

% LocalWords:  MathWeb MathWeb MathWeb
