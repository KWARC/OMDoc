%%%%%%%%%%%%%%%%%%%%%%%%%%%%%%%%%%%%%%%%%%%%%%%%%%%%%%%%%%%%%%%%%%%%%%%%%
% This file is part of the LaTeX sources of the OMDoc 1.6 specification
% Copyright (c) 2015 Michael Kohlhase
% This work is licensed by the Creative Commons Share-Alike license
% see http://creativecommons.org/licenses/by-sa/2.5/ for details
% The source original is at https://github.com/KWARC/OMDoc/doc/spec 
%%%%%%%%%%%%%%%%%%%%%%%%%%%%%%%%%%%%%%%%%%%%%%%%%%%%%%%%%%%%%%%%%%%%%%%%%

\begin{omgroup}[short=Mathematical Statements,id=statements]
{Mathematical Statements (Module {\STmodule{spec}})}

  In this chapter we will look at the \omdoc infrastructure to mark up the
  {\emph{functional structure}} of {\twintoo{mathematical}{statement}s} and their
  interaction with a broader mathematical context.
  
\inputref{statements/types}
\inputref{statements/strict}
\inputref{statements/mtext}
\inputref{statements/constitutive}
\inputref{statements/nonconstit}
\inputref{statements/defschemata}
\inputref{statements/proofs}

\begin{omgroup}[id=st.strict]{Strict Translations}
  We will now give the a formal\ednote{do we really want to call it ``formal''?} semantics
  of the {\STmodule{spec}} elements in terms of strict \omdoc (see
  \sref{strict}).\ednote{what do we do if there is both FMP and CMPs in an
    axiom?}\ednote{what do we do if there is more than one symbol per
    definition?}\ednote{what do we do for non-simple definitions}

\begin{center}\lstset{frame=none,numbers=none,lineskip=-.7ex,aboveskip=-.5em,belowskip=-1em}
  \begin{tabular}{|p{6cm}|p{6cm}|}\hline
    pragmatic & strict\\\hline
{
\begin{lstlisting}[numbers=none,frame=none,mathescape]
<axiom name="$\llquote{n}$" xml:id="$\llquote{i}$">
  $\llquote{body}$
</axiom>
\end{lstlisting}
}&{
\begin{lstlisting}[numbers=none,frame=none,mathescape]
<object name="$\llquote{n}$" xml:id="$\llquote{i}$">
  $\llquote{body}$
</object>
\end{lstlisting}
}\\\hline{
\begin{lstlisting}[numbers=none,frame=none,mathescape]
<symbol name="$\llquote{n}$">
  <type system="$\llquote{s}$">$\llquote{t}$</type>
</symbol>
<definition type="simple"
            xml:id="$\llquote{i}$" for="$\llquote{n}$">
  $\llquote{body}$
</definition>
\end{lstlisting}
}&{
\begin{lstlisting}[numbers=none,frame=none,mathescape]
<object name="$\llquote{n}$" xml:id="$\llquote{i}$">
  <type system="$\llquote{s}$">$\llquote{t}$</type>
  <definition>$\llquote{body}$</definition>
</object>
\end{lstlisting}
}\\\hline 
\end{tabular}
\end{center}
\end{omgroup}
\end{omgroup}

%%% Local Variables: 
%%% mode: latex
%%% TeX-master: "main"
%%% End: 

% LocalWords:  lang adt lst mathescape en monoide mon qtconst def cd dc cmp om
% LocalWords:  nat eq xref int suc requation exp rec ref qttheory sst dec csat
% LocalWords:  isnt peano ness bvar mtext concat empystrg setname OMR xlink Ai
% LocalWords:  href Luehts MMiSS qaulified mythy Bi CiA CiB escapechar setstar
% LocalWords:  gim mobj CDname elAlg es td adv ci csymbol definitionURL aa cc
% LocalWords:  equiv cdbase openmath ns elt attr cdversion FMP pres af
% LocalWords:  cdrevision cdstatus cdurl cdreviewdate OMBIND OMATTR multi bool
% LocalWords:  metadata arities OMA sgrp forall thm mW inv Req wrt omcd
% LocalWords:  nmueller ple Inline omtext cthm tcon eqv noc cax ceqv ucon usat
% LocalWords:  tgroup omdoc thc
