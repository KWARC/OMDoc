%%%%%%%%%%%%%%%%%%%%%%%%%%%%%%%%%%%%%%%%%%%%%%%%%%%%%%%%%%%%%%%%%%%%%%%%%
% This file is part of the LaTeX sources of the OMDoc 1.6 specification
% Copyright (c) 2011 Michael Kohlhase
% This work is licensed by the Creative Commons Share-Alike license
% see http://creativecommons.org/licenses/by-sa/2.5/ for details
% The source original is at https://github.com/KWARC/OMDoc/doc/spec 
%%%%%%%%%%%%%%%%%%%%%%%%%%%%%%%%%%%%%%%%%%%%%%%%%%%%%%%%%%%%%%%%%%%%%%%%%

\documentclass[extrefs]{omdocspec}
\usepackage[show]{ed}     %../macros set to [hide] when publishing
\usepackage{include}
%%%%%%%%%%%%%%%%%%%%%%%%%%%%%%%%%%%%%%%%%%%%%%%%%%%%%%%%%%%%%%%%%%%%%%%%%
% This file is part of the LaTeX sources of the OMDoc 1.6 book
% Copyright (c) 2008 Michael Kohlhase
% This work is licensed by the Creative Commons Share-Alike license
% see http://creativecommons.org/licenses/by-sa/2.5/ for details
% $Id: dcmcontributors.tex 9015 2012-02-22 07:51:58Z kohlhase $
% $HeadURL: https://svn.omdoc.org/repos/omdoc/trunk/doc/macros/dcmcontributors.tex $
%%%%%%%%%%%%%%%%%%%%%%%%%%%%%%%%%%%%%%%%%%%%%%%%%%%%%%%%%%%%%%%%%%%%%%%%%

\WAinstitution[id=CSJU,
               url=http://jacobs-university.de,
               streetaddress={Campus Ring 1},
               townzip={28759 Bremen},
               countryshort=D,
               country=Germany,
               type=University,
               logo=logos/Jacobs_LOGO_RGB.pdf,
               shortname=JACU]
               {Computer Science, Jacobs University Bremen}

\WAinstitution[id=MathEUT,shortname=MathEUT]
               {Department of Mathematics and Computer Science,        
                        Eindhoven University of Technology}

\WAinstitution[id=CSUDS,shortname=CSUDS]
               {Computer Science, Saarland University}

\WAinstitution[id=CSRice,shortname=CSRice]
               {Computer Science, Rice University}

\WAinstitution[id=DFKIHB,
               url=http://dfki.de,
               streetaddress={Enrique-Schmidt-Str. 5},
               townzip={28359 Bremen},
               countryshort=D,
               country=Germany,
               type=Institute,
               shortname=DFKIHB]
               {Deutsches Forschungszentrum f\"ur K\"unstliche Intelligenz Bremen}

\WAinstitution[id=DFKISB,
               url=http://dfki.de,
               countryshort=D,
               country=Germany,
               type=Institute,
               shortname=DFKISB]
               {Deutsches Forschungszentrum f\"ur K\"unstliche Intelligenz Saarbr\"ucken}

\WAperson[id=miko,
           affiliation=CSJU,
           url=http://kwarc.info/kohlhase]
          {Michael Kohlhase}

\WAperson[id=frabe,
           affiliation=CSJU,
           url=http://kwarc.info/frabe]
          {Florian Rabe}

\WAperson[id=cmueller,
           affiliation=CSJU,
           url=http://kwarc.info/cmueller]
          {Christine M{\"u}ller}

\WAperson[id=clange,
           affiliation=CSJU,
           url=http://kwarc.info/clange]
          {Christoph Lange}

\WAperson[id=clange,
           affiliation=CSJU,
           url=http://kwarc.info/clange]
          {Christoph Lange}

\WAperson[id=amc,
           affiliation=MathEUT]
          {Arjeh M. Cohen}

\WAperson[id=cuypers,
           affiliation=MathEUT]
          {Hans Cuypers}

\WAperson[id=barreiro,
           affiliation=MathEUT]
          {E. Reinaldo Barreiro}

\WAperson[id=isucan,affiliation=CSRice]{Ioan Sucan}
\WAperson[id=nmueller,affiliation=CSJU]{Normen M\"uller}
\WAperson[id=benzmueller,affiliation=CSUDS] 
    {Christoph Benzm\"uller}
\WAperson[id=fiedler,affiliation=CSUDS]
     {Armin Fiedler}
\WAperson[id=lesourd,affiliation=CSUDS]
          {Henri Lesourd}
\WAperson[id=bkb,affiliation=DFKIHB]{Bernd Krieg-Br{\"u}ckner}
\WAperson[id=amahnke,affiliation=DFKIHB]{Achim Mahnke}
\WAperson[id=afranke,affiliation=CSUDS]
          {Andreas Franke}
\WAperson[id=autexier,
           affiliation=DFKIHB]{Serge Autexier}
\WAperson[id=hutter,
           affiliation=DFKIHB]{Dieter Hutter}
\WAperson[id=schairer,
           affiliation=DFKISB]{Axel Schairer}
\WAperson[id=mossakowski,
           affiliation=DFKIHB]{Till Mossakowski}
\WAperson[id=maeder,
           affiliation=DFKIHB]{Christian Maeder}
\WAperson[id=luettich,
           affiliation=DFKIHB]{Klaus L\"uttich}
\WAperson[id=melis,
           affiliation=DFKISB]{Erica Melis}
\WAperson[id=ako,
           affiliation=DFKIHB]{Andrea Kohlhase}
\WAperson[id=goguadse,
           affiliation=DFKISB]{Giorgi Goguadse}
\WAperson[id=alberto,
           affiliation=DFKISB]{Alberto Gonzales-Palomo}
\WAperson[id=frischauf,
           affiliation=DFKISB]{Adrian Frischauf}
\WAperson[id=homik,
           affiliation=DFKISB]{Martin Homik}
\WAperson[id=libbrecht,
           affiliation=DFKISB]{Paul Libbrecht}
\WAperson[id=cullrich,
           affiliation=DFKISB]{Carsten Ullrich}

\usepackage[backref=true,hyperref=auto,style=alphabetic]{biblatex}
\addbibresource{preamble.bib}
\addbibresource{kwarcpubs.bib}
\addbibresource{extpubs.bib}
\addbibresource{kwarccrossrefs.bib}
\addbibresource{extcrossrefs.bib}

\makeindex 
\spectrue
\makeextrefs{spec} 
\inputrefs{primer}{../primer/main}
\extrefstyle{primer}{\cite[\protect{\theextref}]{Kohlhase:OMDoc1.6primer}}
\inputrefs{projects}{../projects/main}
\extrefstyle{projects}{\cite[\protect{\theextref}]{Kohlhase:OMDoc1.6projects}}

\begin{document}
\begin{DCmetadata}[titlepage]
  \DCMtitle{An Open Markup Format\\[1ex]
    for Mathematical Documents\\
    \omdoc [Version 1.6 (pre-2.0)]}
  \DCMcreators{miko}
  \DCMdate{\today}
  \DCMcopyrightnotice{2009}{Michael Kohlhase}

  \DCMlicensenotice{This work is licensed by the Creative Commons Share-Alike license
    \url{http://creativecommons.org/licenses/by-sa/2.5/}: the contents of this
    specification or fragments thereof may be copied and distributed freely, as long as
    they are attributed to the original author and source, derivative works (i.e. modified
    versions of the material) may be published as long as they are also licensed under the
    Creative Commons Share-Alike license.}

  \DCMabstract{The \omdoc (\explainomdocacronym) format is a content markup scheme for
    (collections of) mathematical documents including articles, textbooks, interactive
    books, and courses.  \omdoc also serves as the content language for agent
    communication of mathematical services on a mathematical software bus.

    This document is the specification of {\vomdoc{1.6}} of the \omdoc format, the first
    step towards {\omdocv2}. It defines the \omdoc language features and their
    meaning. The content of this part is normative for the \omdoc format; an \omdoc
    document is valid as an \omdoc document, iff it meets all the constraints imposed
    here. \omdoc applications will normally presuppose valid \omdoc documents and only
    exhibit the intended behavior on such.}
\end{DCmetadata}
\newpage
\setcounter{tocdepth}{2}\tableofcontents
\newpage
%%%%%%%%%%%%%%%%%%%%%%%%%%%%%%%%%%%%%%%%%%%%%%%%%%%%%%%%%%%%%%%%%%%%%%%%%
% This file is part of the LaTeX sources of the OMDoc 1.6 specification
% Copyright (c) 2006 Michael Kohlhase
% This work is licensed by the Creative Commons Share-Alike license
% see http://creativecommons.org/licenses/by-sa/2.5/ for details
\svnInfo $Id: preface.tex 8280 2009-03-29 01:52:36Z kohlhase $
\svnKeyword $HeadURL: https://svn.omdoc.org/repos/omdoc/trunk/doc/spec/preface.tex $
%%%%%%%%%%%%%%%%%%%%%%%%%%%%%%%%%%%%%%%%%%%%%%%%%%%%%%%%%%%%%%%%%%%%%%%%%

\begin{omgroup}[id=preface,display=flow]{Preface}
 The {\omdoc} (\explainomdocacronym) format is a content markup scheme for (collections of)
mathematical documents including articles, textbooks, interactive books, and courses.
{\omdoc} also serves as the content language for agent communication of mathematical
services on a mathematical software bus.

This {\ifbook part of the {\report}\else document\fi} is the specification of {\vomdoc{1.6}}
of the {\omdoc} format, the first step towards {\omdocv2}. It defines the {\omdoc}
language features and their meaning. The content of this part is normative for the
{\omdoc} format; an {\omdoc} document is valid as an {\omdoc} document, iff it meets all
the constraints imposed here. {\omdoc} applications will normally presuppose valid
{\omdoc} documents and only exhibit the intended behavior on such.
\end{omgroup}



%%% Local Variables: 
%%% mode: latex
%%% TeX-master: "main"
%%% End: 

% LocalWords:  mypart tocdepth athematical uments
 
\Inspec{toc} % all
% \begin{presonly}% maybe include this again later. 
% \Inspec{appendix}
% \end{presonly}
\printbibliography\newpage
{\small\printindex}

\end{document}

%%% Local Variables: 
%%% mode: latex
%%% TeX-master: t
%%% End: 

% LocalWords:  rnc DCmetadata spechead DCMtitle omdoc DCMcreators miko DCMdate titlepage
% LocalWords:  DCMcopyrightnotice DCMlicensenotice DCMabstract vomdoc omdocv2 behavior
% LocalWords:  explainomdocacronym newpage setcounter tocdepth tableofcontents
% LocalWords:  Inspec presonly printbibliography printindex
