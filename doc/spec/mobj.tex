%%%%%%%%%%%%%%%%%%%%%%%%%%%%%%%%%%%%%%%%%%%%%%%%%%%%%%%%%%%%%%%%%%%%%%%%%
% This file is part of the LaTeX sources of the OMDoc 1.6 specification
% Copyright (c) 2006 Michael Kohlhase
% This work is licensed by the Creative Commons Share-Alike license
% see http://creativecommons.org/licenses/by-sa/2.5/ for details
% The source original is at https://github.com/KWARC/OMDoc/doc/spec 
%%%%%%%%%%%%%%%%%%%%%%%%%%%%%%%%%%%%%%%%%%%%%%%%%%%%%%%%%%%%%%%%%%%%%%%%%

\begin{omgroup}[id=mobj,short=Mathematical Objects]
  {Mathematical Objects (Module {\MOBJmodule{spec}})}

  A distinguishing feature of mathematics is its ability to represent and manipulate ideas
  and objects in symbolic form as mathematical \indexalt{formulae}{formula}.  \omdoc uses
  the \openmath and \mathml formats to represent mathematical formulae and objects.
  Therefore, the {\openmath} standard~\cite{BusCapCar:2oms04} and the \mathml3
  recommendation~\cite{CarIon:MathML03} are part of this specification. \openmath and
  \mathml3 are well-aligned: the \cmathml sublanguage directly encodes \openmath
  objects. \mathml additionally has a sublanguage for expressing the layout of formulae:
  \pmathml, which can be mixed with \cmathml; therefore we prefer \mathml syntax for
  \omdoc and will use it throughout this specification. But we stress that \openmath
  syntax is supported in \omdoc as well.

  We will review {\openmath} objects in \sref{openmath} and {\cmathml} in \sref{cmml},
  and specify an \omdoc element for entering mathematical formulae (element
  \element{legacy}) in \sref{legacy}.  \ednote{discuss MathML3 and the relation between
    MathML and OpenMath and what that means for \omdoc}

\begin{omgroup}[id=mobj.core]{Content Representation of Mathematical Objects}

\begin{module}[id=mobj-core]
\begin{omtext}
We will now recapitulate the representational core of \openmath and \cmathml. Both
represent mathematical \inlinedef{objects as expressions made up from
\begin{compactenum}
\item \defis{symbol} which reference previously declared mathematical objects, these are
  identified by referencing a \trefii{content}{dictionary} (see \sref{cd.def}). 
\item \defis{application} of function or relation operators to a sequence of arguments 
\item \defiis{binding}{structure} that represent functional objects with the help of
  \defiis{bound}{variable}.
\item \defis{identifier} for objects that are known locally (usually bound variables).
\end{compactenum}}
 In {\myfigref{om-commutativity}} we have put the \openmath and strict \cmathml encodings
of the law of commutativity for the real numbers side by side to show the similarities and
differences. There is an obvious line-by-line similarity for the tree constructors and
token elements.  The main difference is the treatment of annotation, which we will
describe in `\sref{annotating}. 
\end{omtext}

\begin{definition}[id=cd.def]
  A \defii[CD]{content}{dictionary} (\defi{CD}) is a \indextoo{machine-readable} document
  that defines the meaning of mathematical concepts expressed by \openmath/\mathml
  symbols.
\end{definition}

The \openmath2 standard provides a minimal data model and \xml encoding for content
dictionaries. We will not review the \openmath content dictionary format there, since
\omdoc theories fill that role in the \omdoc universe -- see \sref{identifying} for
details. 

\begin{omtext}
The central idea is that \trefis{symbol} are identified by a triple: 
\begin{compactenum}
\item \inlinedef{the URI of the file containing the CD (called the
    \defiii{content}{dictionary}{base})},
\item the \inlinedef{the name of the CD, called the \defiii{content}{dictionary}{name}},
  and 
\item the \inlinedef{local name in the CD, called the \defii{symbol}{name}}.
\end{compactenum}
\end{omtext}
\end{module}

\begin{module}[id=OpenMath]
\begin{omgroup}[id=openmath]{The Representational Core of \openmath}
\openmath is a markup language for mathematical formulae that concentrates on the
meaning of formulae building on an extremely simple kernel (markup primitive for
syntactical forms of content formulae), and adds an extension mechanism for mathematical
concepts, the content dictionaries.


We will only review the \xml encoding of {\openmath} objects here, since it is most
relevant to the \omdoc format. All elements of the {\xml} encoding live in the
\twinalt{namespace}{OpenMath}{namespace} \url{http://www.openmath.org/OpenMath}, for which
we traditionally use the \twintoo{namespace}{prefix}
\atwinalt{\snippetin{om:}}{OpenMath}{namespace}{URI}.

\begin{presonly}
\begin{myfig}{om}{{\openmath} Objects in \omdoc}
\begin{scriptsize}
\begin{tabular}{|>{\tt}l|>{\tt}l|>{\tt}l|>{\tt}l|}\hline
{\rm Element}& \multicolumn{2}{l|}{Attributes\hspace*{2.25cm}} & Content  \\\hline
             & {\rm Required}  & {\rm Optional}     &           \\\hline\hline
 \element[ns-elt=om]{OMS}          & cd, name  & id, cdbase, class, style   &  EMPTY \\\hline
  \element[ns-elt=om]{OMV}          & name & id, class, style   &  EMPTY \\\hline
  \element[ns-elt=om]{OMA}          & & id, cdbase, class, style   & \llquote{OMel}* \\\hline
  \element[ns-elt=om]{OMBIND}    & & id, cdbase, class, style   & \llquote{OMel},OMBVAR,\llquote{OMel} \\\hline
  \element[ns-elt=om]{OMBVAR}    & & id, class, style   & (OMV | OMATTR)+ \\\hline
  \element[ns-elt=om]{OMFOREIGN} & & id, cdbase, class, style   & ANY \\\hline
  \element[ns-elt=om]{OMATTR}    & & id, cdbase, class, style   & \llquote{OMel}\\\hline
  \element[ns-elt=om]{OMATP}     & & id, cdbase, class, style   & (OMS, (\llquote{OMel}|OMFOREIGN))+ \\\hline
  \element[ns-elt=om]{OMI}         & & id, class, style   &  [0-9]* \\\hline 
  \element[ns-elt=om]{OMB}        & & id,  class, style   &  \#PCDATA \\\hline 
  \element[ns-elt=om]{OMF}        & & id, class, style, dec, hex &  \#PCDATA \\\hline 
  \element[ns-elt=om]{OME}        & & id, class, style   & \llquote{OMel}?\\\hline
  \element[ns-elt=om]{OMR}       & href &      & \llquote{OMel}?\\\hline
 \multicolumn{4}{|l|}{where {\llquote{OMel}} is {\tt{(OMS|OMV|OMI|OMB|OMSTR|OMF|OMA|OMBIND|OME|OMATTR)}}}\\\hline
\end{tabular}
\end{scriptsize}
\end{myfig}
\end{presonly}

\begin{definition}
  The \eldef[ns-elt=om]{OMA} element contains representations of the function and its
  argument in ``\twinalt{prefix-}{prefix}{notation}'' or ``{\twintoo{Polish}{notation}}'',
  i.e. the first child is the representation of the function and all the subsequent ones
  are representations of the arguments in order.
\end{definition}

\begin{definition}
  Objects and concepts that carry meaning independent of the local context (they are
  called \defis{symbol}) in {\openmath}) are represented as \eldef[ns-elt=om]{OMS}
  elements, where the value of the \attribute[ns-elt=om]{name}{OMS} attribute gives the
  name of the symbol.  The \attribute[ns-elt=om]{cd}{OMS} attribute specifies the relevant
  \trefii{content} {dictionary}, the optional \attributeshort{cdbase} on an
  \element[ns-elt=om]{OMS} element contains a {\indextoo{URI}} that can be used to
  disambiguate the content dictionary.  Alternatively, the {\attributeshort{cdbase}}
  attribute can be given on an {\openmath} element that is a parent to the
  \element[ns-elt=om]{OMS} in question: The \element[ns-elt=om]{OMS} inherits the
  {\attributeshort{cdbase}} of the nearest ancestor (inducing the usual {\xml} scoping
  rules for declarations).\footnote{Note that while the {\attributeshort{cdbase}}
    inheritance mechanism described here remains in effect for {\openmath} objects
    embedded in to the \omdoc format, it is augmented by one in \omdoc. As a consequence,
    {\openmath} objects in \omdoc documents will usually not contain
    {\attributeshort{cdbase}} attributes; see \sref{identifying} for a discussion.}
\end{definition}

\begin{definition}
  Variables are represented as \eldef[ns-elt=om]{OMV} element.  As variables do not carry
  a meaning independent of their local content, \element[ns-elt=om]{OMV} only carries a
  \attribute[ns-elt=om]{name}{OMV} attribute (see \sref{sem-var} for further
  discussion).
\end{definition}

\begin{example}
  The formula $\sin(x)$ would be modeled as an application of the $\sin$ function (which
  in turn is represented as an {\openmath} symbol) to a variable:
\begin{lstlisting}[label=sinx,language=OpenMath,numbers=none,index={OMA,OMV,OMS}]
<OMA xmlns="http://www.openmath.org/OpenMath"
     cdbase="http://www.openmath.org/cd">
  <OMS cd="transc1" name="sin"/>
  <OMV name="x"/>
</OMA>
\end{lstlisting}
  In our case, the function $\sin$ is represented as an \element[ns-elt=om]{OMS} element
  with name {\snippet{sin}} from the {\indextoo{content dictionary}}
  {\snippet{transc1}}. The \element[ns-elt=om]{OMS} inherits the
  {\attributeshort{cdbase}}-value \url{http://www.openmath.org/cd}, which shows that it
  comes from the {\openmath} standard collection of content dictionaries from the
  \element[ns-elt=om]{OMA} element above.  The variable $x$ is represented in an
  \element[ns-elt=om]{OMV} element with \attribute[ns-elt=om]{name}{OMV}-value
  {\snippet{x}}.
\end{example}

\begin{example}
  For the \element[ns-elt=om]{OMBIND} element consider the following representation of the
  formula $\allcdot{x}{\sin(x)\leq\pi}$.
\begin{lstlisting}[label=allxsinx,language=OpenMath,numbers=none,
   index={OMA,OMV,OMBIND,OMBVAR}]
<OMBIND xmlns="http://www.openmath.org/cd">
  <OMS cd="quant1" name="forall"/>
  <OMBVAR><OMV name="x"/></OMBVAR>
  <OMA>
    <OMS cd="arith1" name="leq"/>
    <OMA><OMS cd="transc1" name="sin"/><OMV name="x"/></OMA>
    <OMS cd="nums1" name="pi"/>
  </OMA>
</OMBIND>
\end{lstlisting}
\end{example}

\begin{definition}[id=ombind.def]
  The {\eldef[ns-elt=om]{OMBIND}} element has exactly three children, the first one is a
  ``{\twintoo{binding}{operator}}''\footnote{%\label {foot:binding-operator}
    The binding operator must be a symbol which either has the {\indextoo{role}}
    \attvalshort{binder}{role} assigned by the {\openmath} content dictionary
    (see~\cite{BusCapCar:2oms04} for details) or the symbol declaration in the \omdoc
    content dictionary must have the value \attval{binder}{role}{symbol} for the attribute
    \attribute{role}{symbol} (see \sref{symbol-dec}).} --- in this case the universal
  quantifier, the second one is a list of {\twintoo{bound}{variable}s} that must be
  encapsulated in an {\eldef[ns-elt=om]{OMBVAR}} element, and the third is the
  {\indextoo{body}} of the binding object, in which the bound variables can be used.
  {\openmath} uses the \element[ns-elt=om]{OMBIND} element to unambiguously specify the
  scope of bound variables in expressions: the bound variables in the
  \element[ns-elt=om]{OMBVAR} element can be used only inside the mother
  \element[ns-elt=om]{OMBIND} element, moreover they can be systematically
  \twinalt{renamed}{variable}{renaming} without changing the meaning of the binding
  expression. As a consequence, bound variables in the scope of an
  \element[ns-elt=om]{OMBIND} are distinct as {\openmath} objects from any variables
  outside it, even if they share a name.
\end{definition}
\end{omgroup}
\end{module}

\begin{module}[id=cMathML]
\begin{omgroup}[id=cmml]{Strict Content MathML}

  \cmathml is a content markup format that represents the abstract structure of formulae
  in trees of logical sub-expressions much like \openmath: the \mathml3
  recommendation~\cite{CarlisleEd:MathML3} identifies a sublanguage: strict \cmathml that
  is isomorphic to \openmath2. 
\begin{presonly}
\begin{myfig}{cmml}{{\cmathml} in \omdoc}
\begin{scriptsize}
\begin{tabular}{|l|l|p{2truecm}|p{4truecm}|}\hline
{\rm Element}& \multicolumn{2}{l|}{Attributes\hspace*{2.25cm}}  & Content  \\\hline
              & {\rm Required}  & {\rm Optional}     &            \\\hline\hline
 \element[ns-elt=m]{math}       & & 
 \attribute[ns-elt=m]{id}{math}, 
 \attribute[ns-elt=m]{xlink:href}{math}                   & \llquote{CMel}+\\\hline

 \element[ns-elt=m]{apply}  & &
 \attribute[ns-elt=m]{id}{apply}, 
   \attribute[ns-elt=m]{xlink:href}{apply}                   &
 \element[ns-elt=m]{bvar?},\llquote{CMel}*\\\hline

\ \element[ns-elt=m]{csymbol}    & 
 \attribute[ns-elt=m]{definitionURL}{csymbol}  & 
\attribute[ns-elt=m]{id}{csymbol}, 
\attribute[ns-elt=m]{xlink:href}{csymbol}    & 
EMPTY \\\hline

 \element[ns-elt=m]{ci}        & & 
\attribute[ns-elt=m]{id}{ci}, 
\attribute[ns-elt=m]{xlink:href}{ci}  &
 \#PCDATA \\\hline       

 \element[ns-elt=m]{cn}         & & 
\attribute[ns-elt=m]{id}{cn}, 
\attribute[ns-elt=m]{xlink:href}{cn}         &
 ([0-9]|,|.)(*|e([0-9]|,|.)*)?\\\hline       

 \element[ns-elt=m]{bvar}       & &
 \attribute[ns-elt=m]{id}{bvar}, 
 \attribute[ns-elt=m]{xlink:href}{bvar}                   &
 \element[ns-elt=m]{ci}|
\element[ns-elt=m]{semantics}\\\hline

 \element[ns-elt=m]{semantics}  & & 
\attribute[ns-elt=m]{id}{semantics}, 
\attribute[ns-elt=m]{xlink:href}{semantics}, 
\attribute[ns-elt=m]{definitionURL}{semantics} & 
\llquote{CMel},(\element[ns-elt=m]{annotation} | 
                        \element[ns-elt=m]{annotation-xml})*\\\hline

 \element[ns-elt=m]{annotation} & & 
\attribute[ns-elt=m]{definitionURL}{annotation}, \attribute[ns-elt=m]{encoding}{annotation}      &
 \#PCDATA \\\hline

 \element[ns-elt=m]{annotation-xml} & & 
\attribute[ns-elt=m]{definitionURL}{annotation-xml}, 
\attribute[ns-elt=m]{encoding}{annotation-xml}      &
 ANY \\\hline
 \multicolumn{4}{|l|}{where {\llquote{CMel}} is 
\element[ns-elt=m]{apply}|
     \element[ns-elt=m]{csymbol}|
\element[ns-elt=m]{ci}|
\element[ns-elt=m]{cn}|\element[ns-elt=m]{semantics}}\\\hline
\end{tabular}
\end{scriptsize}
\end{myfig}
\end{presonly}

\begin{definition}[id=math.def]
  The top-level element of \mathml is the {\eldef[ns-elt=m]{math}} element it contains an
  functional expression composed
  \begin{compactenum}
  \item \trefis{identifier} (element \eldef[ns-elt=m]{ci}) corresponding to
    {\indextoo{variable}s}. The content of the \element[ns-elt=m]{ci} element is a unicode
    string used as the name of the identifier.
  \item \trefis{symbol} (element \eldef[ns-elt=m]{csymbol}) for arbitrary symbols. The
    content of the \element[ns-elt=m]{csymbol} element is the name of the symbol, its
    meaning is dermined by the \attribute{csymbol}{cd} attribute that contains a content
    dictionary name.
  \item \trefis{application} (element\eldef[ns-elt=m]{apply})    
  \item \trefiis{binding}{structure} (element \eldef[ns-elt=m]{bind} with
    \eldef[ns-elt=m]{bvars}).
  \end{compactenum}
\end{definition}
\end{omgroup}
\end{module}

\setbox0=\hbox{\begin{minipage}{5.3cm}\def\baselinestretch{.975}
\begin{lstlisting}[label=omvsmom,language=OpenMath,frame=none,numbers=none,
    index={OMBIND,OMS,OMBVAR,OMV,OMATTR,OMATP}]
<OMBIND>                          
 <OMS cd="quant1" name="forall"/> 
 <OMBVAR>                         
  <OMATTR>                        
   <OMATP>                        
    <OMS cd="sts" name="type"/>   
    <OMS cd="setname1" name="R"/>  
   </OMATP>                       
   <OMV name="a"/>                
  </OMATTR> 

                       
  <OMATTR>                        
   <OMATP>                        
    <OMS cd="sts" name="type"/>   
    <OMS cd="setname1" name="R"/>  
   </OMATP>                       
   <OMV name="b"/>                
  </OMATTR>                        
 </OMBVAR>                        
  <OMA>                           
   <OMS cd="relation" name="eq"/> 
   <OMA>                          
    <OMS cd="arith1" name="plus"/>
    <OMV name="a"/>               
    <OMV name="b"/>               
   </OMA>                         
   <OMA>                          
    <OMS cd="arith1" name="plus"/>
    <OMV name="b"/>               
    <OMV name="a"/>               
   </OMA>                         
 </OMA>                           
</OMBIND>                         
\end{lstlisting}
\end{minipage}}
\setbox1=\hbox{\begin{minipage}{7.5cm}
 \begin{lstlisting}[label=omvsmm,language=MathML,frame=none,numbers=none,index={math,apply,forall,bvar,ci,eq,plus}]
<m:apply>  
 <m:forall/>
 <m:bvar>
  <m:semantics>
   <m:ci>a</m:ci>
   <m:annotation-xml 
     definitionURL="http://www.openmath.org/cd/sts#type">
    <m:csymbol cd="setname1">R</m:csymbol>
   </m:annotation-xml>
  </m:semantics>
 </m:bvar>
 <m:bvar>
  <m:semantics>
   <m:ci>a</m:ci>
   <m:annotation-xml 
     definitionURL="http://www.openmath.org/cd/sts#type">
    <m:csymbol cd="setname1">R</m:csymbol>
   </m:annotation-xml>
  </m:semantics>
 </m:bvar>
 <m:apply>
  <m:csymobl cd="relation1">eq</m:csymbol>
  <m:apply>
   <m:csymbol cd="arith1">plus</m:csymbol>
   <m:ci>a</m:ci>
   <m:ci>b</m:ci>
  </m:apply>
  <m:apply>
   <m:csymbol cd="arith1">plus</m:csymbol>
   <m:ci>b</m:ci>
   <m:ci>a</m:ci>
  </m:apply>
 </m:apply>
</m:apply>
\end{lstlisting}
\end{minipage}}
\begin{myfig}{om-commutativity}{{\openmath} vs. C-{\mathml} for Commutativity}
\begin{tabular}{cc}
  {\large\openmath}  &  {\large\mathml}\\
  \fbox{\box0} & \fbox{\box1}
\end{tabular}
\end{myfig}
\end{omgroup}

\begin{omgroup}[id=annotating]{Annotating Mathematical Objects}
  \openmath offers an element for annotating (parts of) formulae with external information
  (e.g. {\mathml} or {\LaTeX} presentation):

\begin{definition}[id=omattr.def]
  The {\eldef[ns-elt=om]{OMATTR}} element that pairs an {\openmath} object with an
  attribute-value list. To annotate an {\openmath} object, it is embedded as the second
  child in an \element[ns-elt=om]{OMATTR} element. The attribute-value list is specified
  by children of the preceding {\eldef[ns-elt=om]{OMATP}} (Attribute value Pair) element,
  which has an even number of children: children at odd positions must be
  \element[ns-elt=om]{OMS} (specifying the attribute, they are called \defis{key} or
  \defis{feature})\footnote{There are two kinds of keys in {\openmath} distinguished
    according to the \attribute{role}{symbol} value on their \element{symbol} declaration
    in the \twintoo{content}{dictionary}: \attval{attribution}{role}{symbol} specifies
    that this attribute value pair may be ignored by an application, so it should be used
    for information which does not change the meaning of the attributed {\openmath}
    object. The \attribute{role}{symbol} is used for keys that modify the meaning of the
    attributed {\openmath} object and thus cannot be ignored by an application.}, and
  children at even positions are the \defis{value} of the keys specified by their
  immediately preceding siblings. In the {\openmath} fragment in {\mylstref{omattr}} the
  expression $x+\pi$ is annotated with an alternative representation and a color.
\end{definition}

A special application of the \element[ns-elt=om]{OMATTR} element is associating
non-\openmath objects with \openmath objects.

\begin{definition}[id=omforeign.def]
  For this, {\openmath}2 allows to use an \eldef[ns-elt=om]{OMFOREIGN} element in the even
  positions of an \element[ns-elt=om]{OMATP}. This element can be used to hold arbitrary
  {\xml} content (in our example above SVG: Scalable Vector Graphics~\cite{W3C:svg02}),
  its required \attribute[ns-elt=om]{encoding}{OMFOREIGN} attribute specifies the format
  of the content.
\end{definition}

We recommend a \twintoo{MIME}{type}~\cite{FreBor:MIME96} (see \sref{pres-bound} for an
application).

\begin{example}[id=omattr.ex]
  Here we use the \element[ns-elt=om]{OMATTR} element to associate various other
  representationsn with an application.
\begin{lstlisting}[language=OpenMath,label=lst:omattr,mathescape,
                   caption={Associating Alternate Representations with an
                   {\openmath} Object},
                   numbers=none,index={OMATTR,OMATP}]
<OMATTR>
  <OMATP>
    <OMS cd="alt-rep" name="ascii"/>
    <OMSTR>(x+1)</OMSTR>
    <OMS cd="alt-rep" name="svg"/>
    <OMFOREIGN encoding="application/svg+xml">
      <svg xmlns='http://www.w3.org/2000/svg'>$\ldots$</svg>
    </OMFOREIGN>
    <OMS cd="pres" name="color"/>
    <OMS cd="pres" name="red"/>
  </OMATP>
  <OMA>
    <OMS cd="arith1" name="plus"/>
    <OMV name="x"/>
    <OMS cd="nums1" name="pi"/>
  </OMA>
</OMATTR>
\end{lstlisting}
\end{example}

In \cmathml, the same effect can be achieved by the \element[ns-elt=m]{semantics} element
whose first child is the annotated object and subsequent \element[ns-elt=m]{annotation}
and \element[ns-elt=m]{annotation-xml} and children carry the annotations.

\begin{definition}[id=semantics.def]
  The \eldef[ns-elt=m]{semantics} element provides a way to annotate {\cmathml} elements
  with arbitrary information. The first child of the \element[ns-elt=m]{semantics} element
  is annotated with the information in the {\eldef[ns-elt=m]{annotation-xml}} (for
  {\xml}-based information) and {\eldef[ns-elt=m]{annotation}} (for other information)
  elements that follow it. These elements carry
  \attribute[ns-elt=m]{definitionURL}{annotation} attributes that point to a
  ``definition'' of the kind of information provided by them. The optional
  \attribute[ns-elt=m]{encoding}{annotation} is a string that describes the format of the
  content.
\end{definition}

\begin{example}[id=semantics.ex]
  To express the content of \sref{omattr.ex} in \cmathml, we use the
  \element[ns-elt=m]{semantics}, \element[ns-elt=m]{annotation}, and
  \element[ns-elt=m]{annotationxml} elements.
\begin{lstlisting}[language=MathML,label=lst:semantics,mathescape,
                   caption={Associating Alternate Representations in \cmathml},
                   numbers=none,index={m:semantics,m:annotation-xml,m:annotation}]
<m:semantics>
  <m:apply>
    <m:csymbol cd="arith1">plus</m:csymbol>
    <m:ci>x</m:ci>
    <m:csymbol cd="nums1">pi</m:csymbol>
  </m:apply>
  <m:annotation definitionURL="http://omdoc.org/cds/alt-rep#ascii"> 
    (x+1)
  </m:annotation>
  <m:annotation-xml definitionURL="http://omdoc.org/cds/alt-rep#svg" 
     encoding="application/svg+xml">
   <svg xmlns='http://www.w3.org/2000/svg'>$\ldots$</svg>
  </m:annotation-xml>
  <m:annotation-xml definitionURL="http://omdoc.org/cds/pres#color" 
     encoding="application/openmath+xml"> 
   <OMS cd="pres" name="red"/>
  </m:annotation-xml>
</m:semantics>
\end{lstlisting}
\end{example}
\end{omgroup}


\begin{omgroup}[id=om.error]{Programming Extensions of for Mathematical Objects}

\begin{definition}[id=omi.def]
  For representing objects in {\twintoo{computer algebra}{system}s} {\openmath} also
  provides other basic data types: {\eldef[ns-elt=om]{OMI}} for {\indextoo{integer}s},
  {\eldef[ns-elt=om]{OMB}} for {\indextoo{byte array}s}, {\eldef[ns-elt=om]{OMSTR}} for
  {\indextoo{string}s}, and {\eldef[ns-elt=om]{OMF}} for floating point numbers. These do
  not play a large role in the context of \omdoc, so we refer the reader to the
  {\openmath} standard~\cite{BusCapCar:2oms04} for details.
\end{definition}

\begin{definition}
  \cmathml uses the \eldef[ns-elt=m]{cn} element for number expressions. The attribute
  \attribute[ns-elt=m]{type}{cn} can be used to specify the mathematical type of the
  number, e.g. {\snippet{complex}}, {\snippet{real}}, or {\snippet{integer}}. The content
  of the \element[ns-elt=m]{cn} element is interpreted as the value of the number
  expression.
\end{definition}

\begin{definition}[id=ome.def]
  The {\eldef[ns-elt=om]{OME}} element is used for {\atwintoo{in-place}{error}{markup}} in
  {\openmath} objects, it can be used almost everywhere in {\openmath} elements. It has
  two children; the first one is an {\twintoo{error}{operator}}\footnote{An error operator
    is like a {\twintoo{binding}{operator}}, only the symbol has role
    \attval{error}{role}{symbol}.}, i.e. an {\openmath} symbol that specifies the kind
  of error, and the second one is the faulty {\openmath} object fragment. Note that since
  the whole object must be a valid {\openmath} object, the second child must be a
  well-formed {\openmath} object fragment.
\end{definition}
As a consequence, the \element[ns-elt=om]{OME} element can only be used for
``{\twintoo{semantic}{error}s}'' like non-existing content dictionaries, out-of-bounds
errors, etc.  {\xml}-well-formedness and DTD-validity errors will have to be handled by
the {\xml} tools involved. In the following example, we have marked up two errors in a
faulty representation of $\sin(\pi)$.  The outer error flags an arity violation (the
function $\sin$ only allows one argument), and the inner one flags the typo in the
representation of the constant $\pi$ (we used the name {\snippet{po}} instead of
{\snippet{pi}}).

\begin{lstlisting}[label=ome,language=OpenMath,numbers=none,index={OME}]
<OME>
  <OMS cd="type-error" name="arity-violation"/>
  <OMA>
    <OMS cd="transc1" name="sin"/>
    <OME>
      <OMS cd="error" name="unexpected_symbol"/>
      <OMS cd="nums1" name="po"/>
    </OME>
    <OMV name="x"/>
  </OMA>
</OME>
\end{lstlisting}
  As we can see in this example, errors can be nested to encode multiple faults found by
  an {\openmath} application.

  \ednote{need to talk about the \element[ns-elt=m]{cerror} element}
\end{omgroup}





\begin{module}[id=omml-semvar]
\begin{omgroup}[id=sem-var,short=Semantics of Variables]{The Semantics of Variables in
  \openmath and \cmathml}
 
\begin{omtext}
A more subtle, but nonetheless crucial difference between {\openmath} and {\mathml} is the
handling of variables, symbols, their names, and equality conditions.  {\openmath} uses
the \attribute[ns-elt=om]{name}{OMV, om:OMS} attribute to identify a variable or symbol,
and delegates the presentation of its name to other methods such as style sheets. As a
consequence, the elements \element[ns-elt=om]{OMS} and \element[ns-elt=om]{OMV} are
empty, and we have to understand the value of the {\attribute[ns-elt=om]{name}{OMV,
    om:OMS}} attribute as a {\indextoo{pointer}} to a defining occurrence. In case of
symbols, this is the symbol declaration in the \twintoo{content}{dictionary} identified
in the \attribute[ns-elt=om]{cd}{OMS} attribute. A symbol {\snippet{<OMS
    cd="\llquote{$cd_1$}" name="\llquote{$name_1$}"/>}} is equal to {\snippet{<OMS
    cd="\llquote{$cd_2$}" name="\llquote{$name_2$}"/>}}, iff
{\llquote{$cd_1$}=\llquote{$cd_2$}} and {\llquote{$name_1$}=\llquote{$name_2$}} as {\xml}
simple names.  In case of variables this is more difficult: if the variable is
\twinalt{bound}{bound}{variable} by an \element[ns-elt=om]{OMBIND} element
{\inlinedef{We say that an \element[ns-elt=om]{OMBIND} element \defi{binds} an
    {\openmath} variable {\snippet{<OMV name="x"/>}}, iff this
    \element[ns-elt=om]{OMBIND} element is the nearest one, such that {\snippet{<OMV
        name="x"/>}} occurs in (second child of the \element[ns-elt=om]{OMATTR} element
    in) the \element[ns-elt=om]{OMBVAR} child (this is the
    \defii{defining}{occurrence} of {\snippet{<OMV name="x"/>}} here).}}, then we
interpret all the variables {\snippet{<OMV name="x"/>}} in the
\element[ns-elt=om]{OMBIND} element as equal and different from any variables
{\snippet{<OMV name="x"/>}} outside. In fact the {\openmath} standard states that bound
variables can be \twinalt{renamedo}{renaming}{variable} without changing the object
({\twinalt{$\alpha$-conversion}{alpha}{conversion}}). If {\snippet{<OMV name="x"/>}} is
not bound, then the scope of the variable cannot be reliably defined; so equality with
other occurrences of the variable {\snippet{<OMV name="x"/>}} becomes an ill-defined
problem.  We therefore discourage the use of unbound variables in \omdoc; they are very
simple to avoid by using symbols instead, introducing suitable theories if necessary (see
\sref{theories-contexts}).

{\mathml} goes a different route: the \element[ns-elt=m]{csymbol} and \element[ns-elt=m]{ci}
elements have content that is {\pmathml}, which is used for the presentation of the
variable or symbol name.\footnote{Note that surprisingly, the empty {\cmathml} elements
  are treated more in the {\openmath} spirit.}  While this gives us a much better handle
on presentation of objects with variables than {\openmath} (where we are basically forced
to make due with the ASCII\footnote{In the current {\openmath} standard, variable names
  are restricted to alphanumeric characters starting with a letter. Note that unlike with
  symbols, we cannot associate presentation information with variables via style sheets,
  since these are not globally unique (see \sref{pres-bound} for a discussion of the
  \omdoc solution to this problem).}  representation of the variable name), the question
of scope and equality becomes much more difficult: Are two variables (semantically) the
same, even if they have different colors, sizes, or font families? Again, for symbols the
situation is simpler, since the \attribute[ns-elt=m]{definitionURL}{csymbol} attribute on the
\element[ns-elt=m]{csymbol} element establishes a global identity criterion (two symbols are
equal, iff they have the same \attribute[ns-elt=m]{definitionURL}{csymbol} value (as URI
strings; see~\cite{BerFie:uri98}).) The second edition of the {\mathml} standard adopts
the same solution for bound variables: it recommends to annotate the \element[ns-elt=m]{bvar}
elements that declare the bound variable with an \attribute[ns-elt=m]{id}{bvar} attribute and
use the \attribute[ns-elt=m]{definitionURL}{ci} attribute on the {\twintoo{bound}{occurrence}s}
of the \element[ns-elt=m]{ci} element to point to those. The following example is taken
from~\cite{KohDev:bvm03}, which has more details.
\end{omtext}

\begin{lstlisting}[language=MathML,label=bvar-mathml,
     index={math,bvar,ci,definitionURL}]
<m:lambda>
  <m:bvar><m:ci xml:id="the-boundvar">complex presentation</m:ci></m:bvar>
  <m:apply>
    <m:plus/>
    <m:ci definitionURL="#the-boundvar">complex presentation</m:ci>
    <m:ci definitionURL="#the-boundvar">complex presentation</m:ci>
  </m:apply>
</m:lambda>  
\end{lstlisting}

For presentation in {\mathml}, this gives us the best of both approaches, the
\element[ns-elt=m]{ci} content can be used, and the {\indextoo{pointer}} gives a simple
semantic equivalence criterion. For presenting {\openmath} and {\cmathml} in other
formats \omdoc makes use of the infrastructure introduced in module
{\PRESmodule{spec}}; see \sref{pres-bound} for a discussion.
\end{omgroup}
\end{module}

\begin{module}[id=legacy]
\begin{omgroup}[id=legacy]{Legacy Representation for Migration}

  Sometimes, \omdoc is used as a migration format from {\indextoo{legacy}} texts (see
  {\extref{primer}{algebra}} for an example). In such documents it can be too much effort
  to convert all mathematical objects and formulae into {\openmath} or {\cmathml} form.

\begin{presonly}
\begin{myfig}{mobjtable}{Mathematical Objects in \omdoc}
\begin{scriptsize}
\begin{tabular}{|l|p{1.5truecm}|l|l|}\hline
Element & \multicolumn{2}{l|}{Attributes\hspace*{2.25cm}} & Content  \\\hline
             & Required  & Optional     &           \\\hline\hline
 \element{legacy}  & 
 \attribute{format}{legacy} & 
 \attribute[ns-attr=xml]{id}{legacy}, 
 \attribute{formalism}{legacy}  &  
\#PCDATA \\\hline
\end{tabular}
\end{scriptsize}
\end{myfig}
\end{presonly}

\begin{definition}[id=legacy.def]
  For this situation \omdoc provides the {\eldef{legacy}} element, which can contain
  arbitrary math markup\footnote{If the content is an {\xml}-based, format like Scalable
    Vector Graphics~\cite{W3C:svg02}, the {\indextoo{DTD}} must be augmented accordingly
    for validation.}. The \element{legacy} element can occur wherever an {\openmath}
  object or {\cmathml} expression can and has an optional
  \attribute[ns-attr=xml]{id}{legacy} attribute for identification. The content is
  described by a pair of attributes:
  \begin{itemize}
  \item \attribute{format}{legacy} (required) specifies the format of the content using
    URI reference. \omdoc does not restrict the possible values, possible values include
    \attval{TeX}{format}{legacy}, \attval{pmml}{format}{legacy},
    \attval{html}{format}{legacy}, and \attval{qmath}{format}{legacy}.
  \item \attribute{formalism}{legacy} is optional and describes the formalism (if
    applicable) the content is expressed in. Again, the value is unrestricted character
    data to allow a {\twintoo{URI}{reference}} to a definition of a formalism.
  \end{itemize}
\end{definition}

For instance in the following \element{legacy} element, the identity function is encoded
in the untyped $\lambda$-calculus, which is characterized by a reference to the relevant
Wikipedia article.

\begin{lstlisting}[index={legacy}]
<legacy format="TeX" formalism="http://en.wikipedia.org/wiki/Lambda_calculus">
  \lambda{x}{x}
</legacy>
\end{lstlisting}
\end{omgroup}
\end{module}
\end{omgroup}
%%% Local Variables: 
%%% mode: latex
%%% TeX-master: "main"
%%% End: 

% LocalWords:  pmml qmath mobjtable xref cd transc var quant arith geq OMB OMF oldpart sc
% LocalWords:  OMSTR OME po OMATP OMFOREIGN om OMR arcangle nodesep openmath emph omgroup
% LocalWords:  nat xlink href lst foo baz cmml ci cn csymbol definitionURL bvar omdoc
% LocalWords:  sts setname eq mathml mapsto truecm PCDATA dag sem alt ascii dec cmathml
% LocalWords:  pres omvscmml OMel dtd ref CMel BNF forall underdefined sec rep MOBJmodule
% LocalWords:  omattr svg xmlns boundvar mobj cdbase omcds sinx allxsinx ome ns ednote
% LocalWords:  omvsmom omvsmm elt mathescape attr nt OMV OMA OMBIND OMI sref sref myfig
% LocalWords:  OMBVAR omcd CDDefinition leq nums nd th hline hspace extref omtext adefii
%  LocalWords:  inlinedef indextoo twinalt twintoo atwinalt snippetin llquote omcore.def
%  LocalWords:  eldef defi attributeshort lstlisting allcdot attvalshort defis mylstref
%  LocalWords:  ldots atwintoo myfigref tikzpicture cdot attributeshortcomment omdocv
%  LocalWords:  lstset acyclicity cdots defiis pmathml hbox baselinestretch fbox fbox
%  LocalWords:  omml-types DevKoh stm03 Mylstsref realizations omml-semvar defii renamedo
%  LocalWords:  KohDev bvm03 PRESmodule itemize characterized
