%%%%%%%%%%%%%%%%%%%%%%%%%%%%%%%%%%%%%%%%%%%%%%%%%%%%%%%%%%%%%%%%%%%%%%%%%
% This file is part of the LaTeX sources of the OMDoc 1.6 specification
% Copyright (c) 2006 Michael Kohlhase
% This work is licensed by the Creative Commons Share-Alike license
% see http://creativecommons.org/licenses/by-sa/2.5/ for details
% The source original is at https://github.com/KWARC/OMDoc/doc/spec 
%%%%%%%%%%%%%%%%%%%%%%%%%%%%%%%%%%%%%%%%%%%%%%%%%%%%%%%%%%%%%%%%%%%%%%%%%

\begin{omgroup}[id=rnc]{The RelaxNG Schema for OMDoc}

We reprint the modularized {\relaxng} schema for \omdoc here. It is available at
\url{http://www.omdoc.org/rnc} and consists of separate files for the \omdoc modules,
which are loaded by the schema driver {\snippet{omdoc.rnc}} in this directory. We will use
the abbreviated syntax for {\relaxng} here, since the {\xml} syntax, \indexalt{document
  typedefinitions}{DTD} and even \twinalt{{\xml} schemata}{XML}{schema} can be generated
from it by standard tools.

The {\relaxng} schema consists of the grammar fragments for the modules (see
\sref{rnc.math} to \sref{rnc.quiz}).

\begin{omgroup}[id=rnc.drivers]{The Sub-Language Drivers}

The Schema comes in two parts: strict \omdoc

\lstinputlisting[language=RNC,nolol]{../../schema/rnc/strict/omdoc-strict.rnc}
 
and pragmatic \omdoc

\lstinputlisting[language=RNC,nolol]{../../schema/rnc/omdoc.rnc}
\end{omgroup}

\begin{omgroup}[id=rnc.math]{Module {\MOBJmodule{RNC}}: Mathematical Objects and Text}

The RNC module {\MOBJmodule{RNC}} includes the representations for mathematical objects
and defines the \element{legacy} element (see \sref{mobj} for a discussion). It
includes the standard {\relaxng} schema for {\openmath} (we have reprinted it in
{\myappsecref{rnc.openmath}}) adding the \omdoc identifier and {\css}
\twinalt{attributes}{CSS}{attribute} to all elements. If also includes a schema for
{\mathml} (see {\myappsecref{rnc.mathml}}).

\lstinputlisting[language=RNC,index={legacy},nolol]{../../schema/rnc/strict/omdocmobj.rnc}
\end{omgroup}

\begin{omgroup}[id=rnc.mtxt]{Module {\MTXTmodule{RNC}}: Mathematical Text}
  The RNC module {\MTXTmodule{RNC}} provides infrastructure for mathematical vernacular
  (see \sref{mtext} for a discussion).
  \lstinputlisting[language=RNC,index={omtext},nolol]
  {../../schema/rnc/strict/mtxt-strict.rnc}

  And now the pragmatic language
  \lstinputlisting[language=RNC,index={omtext},nolol]
                          {../../schema/rnc/pragmatic/omdocmtxt.rnc}
\end{omgroup}

\begin{omgroup}[id=rnc.meta]{Module {\DOCmodule{META}}: Metadata}

  For the treatment of metadata we include a generic version of the Dublin Core vocabulary
  for bibliographic metadata (see the schema at \sref{rnc.dc}), and extend it with MARC
  relator roles (see {\srefs{dc-elements}{dc-roles}} for a discussion and
  \sref{rnc.marc} for the schema) and a content-oriented version of Creative Commons
  License specifications (see \sref{rnc.cc} for the schema).

\lstinputlisting[language=RNC,index={metadata,link,resource},nolol]{../../schema/rnc/strict/meta-strict.rnc}
\end{omgroup}

\begin{omgroup}[id=rnc.dc]{Dublin Core Metadata}
\lstinputlisting[language=RNC,nolol,index={}]{../../schema/rnc/pragmatic/omdocdc.rnc}
\lstinputlisting[language=RNC,nolol,
index={contributor,creator,title,subject,description,publisher,type,
  format,source,language,relation,rights,date,identifier}]{../../schema/rnc/pragmatic/dublincore.rnc}
\end{omgroup}

\begin{omgroup}[id=rnc.marc]{MARC Relators for Bibliographic Roles}
\lstinputlisting[language=RNC,nolol]{../../schema/rnc/pragmatic/MARCRelators.rnc}
\end{omgroup}

\begin{omgroup}[id=rnc.cc]{Creative Commons Licenses}
\lstinputlisting[language=RNC,nolol,index={license,permissions,prohibitions,requirements}]
  {../../schema/rnc/pragmatic/omdoccc.rnc}
\lstinputlisting[language=RNC,nolol,index={license,permissions,prohibitions,requirements}]
  {../../schema/rnc/pragmatic/creativecommons.rnc}
\end{omgroup}

\begin{omgroup}[id=rnc.doc]{Module {\DOCmodule{RNC}}: Document Infrastructure}
  The RNC module {\DOCmodule{RNC}} specifies the document infrastructure of \omdoc
  documents (see \sref{omdoc-infrastructure} for a discussion).
  \lstinputlisting[language=RNC, index={omdoc},nolol]{../../schema/rnc/strict/doc-strict.rnc}
  \lstinputlisting[language=RNC, index={},nolol]{../../schema/rnc/pragmatic/omdocdoc.rnc}
\end{omgroup}

\begin{omgroup}[id=rnc.st]{Module {\STmodule{RNC}}: Mathematical Statements}

The RNC module {\STmodule{RNC}} deals with mathematical statements like assertions and
examples in \omdoc and provides an infrastructure for mathematical theories as contexts,
for the \omdoc elements that fix the meaning for symbols, see \sref{statements}
for a discussion.
\lstinputlisting[language=RNC,nolol,
index={theory,constant,definition,type,include}] {../../schema/rnc/strict/st-strict.rnc}
\lstinputlisting[language=RNC,nolol,
index={assertion,alternative,example,imports}] {../../schema/rnc/pragmatic/omdocst.rnc}
\end{omgroup}

\begin{omgroup}[id=rnc.adt]{Module {\ADTmodule{RNC}}: Abstract Data Types}
The RNC module {\ADTmodule{RNC}} specifies the grammar for abstract data types in 
\omdoc, see \sref{adt} for a discussion.
\lstinputlisting[language=RNC,nolol,
index={adt,sortdef,insort,constructor,recognizer,argument,destructor}]
{../../schema/rnc/pragmatic/omdocadt.rnc}
\end{omgroup}

\begin{omgroup}[id=rnc.proof]{Module {\PFmodule{RNC}}: Proofs and Proof objects}
The RNC module {\PFmodule{RNC}} deals with mathematical argumentations and proofs in
\omdoc, see \sref{proofs} for a discussion.
\lstinputlisting[language=RNC,nolol,
index={proof,proofobject,derive,hypothesis,method,premise}]
{../../schema/rnc/pragmatic/omdocpf.rnc}
\end{omgroup}

\begin{omgroup}[id=rnc.ext]{Module {\EXTmodule{RNC}}: Applets and non-XML data}
The RNC module {\EXTmodule{RNC}} provides an infrastructure for applets, program code, and
non-{\xml} data like images or measurements (see \sref{ext} for a discussion).
\lstinputlisting[language=RNC,nolol,
index={omlet,private,code,input,output,effect,data}] {../../schema/rnc/pragmatic/omdocext.rnc}
\end{omgroup}

\begin{omgroup}[id=rnc.pres]{Module {\PRESmodule{RNC}}: Adding Presentation Information}
The RNC module {\PRESmodule{RNC}} provides a sub-language for defining notations for
mathematical symbols and for styling \omdoc elements (see \sref{pres} for a
discussion).
\lstinputlisting[language=RNC,nolol]
{../../schema/rnc/strict/notation-strict.rnc}

\lstinputlisting[language=RNC,nolol]
{../../schema/rnc/pragmatic/notation-mmt.rnc}
\end{omgroup}

\begin{omgroup}[id=rnc.quiz]{Module {\QUIZmodule{RNC}}: Infrastructure for Assessments}
The RNC module {\QUIZmodule{RNC}} provides a basic infrastructure for various kinds of
exercises (see \sref{quiz} for a discussion).
 \lstinputlisting[language=RNC,nolol,index={exercise,hint,solution,mc,choice,answer}]
  {../../schema/rnc/pragmatic/omdocquiz.rnc}
\end{omgroup}
\end{omgroup}

%%% Local Variables: 
%%% mode: latex
%%% TeX-master: "main"
%%% End: 

% LocalWords:  rnc MOBJ lstlistings lstomdoc fullflexible OMB OMSTR OMF OME RT
% LocalWords:  OMATP ol ul li tr td mtxt doc dc adt sortdef insort mc nolol
% LocalWords:  proofobject cth ext omlet pres omstyle xslt omdoc mobj openmath
% LocalWords:  mathml omtext OMS OMV OMI OMA OMBIND OMBVAR OMATTR
% LocalWords:  rt mobj metadata destructor recurse mobj  mobj
