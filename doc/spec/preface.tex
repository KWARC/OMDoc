%%%%%%%%%%%%%%%%%%%%%%%%%%%%%%%%%%%%%%%%%%%%%%%%%%%%%%%%%%%%%%%%%%%%%%%%%
% This file is part of the LaTeX sources of the OMDoc 1.6 specification
% Copyright (c) 2015 Michael Kohlhase
% This work is licensed by the Creative Commons Share-Alike license
% see http://creativecommons.org/licenses/by-sa/2.5/ for details
% The source original is at https://github.com/KWARC/OMDoc/doc/spec 
%%%%%%%%%%%%%%%%%%%%%%%%%%%%%%%%%%%%%%%%%%%%%%%%%%%%%%%%%%%%%%%%%%%%%%%%%

\begin{omgroup}[id=preface,display=flow]{Preface}
 The \omdoc (\explainomdocacronym) format is a content markup scheme for (collections of)
mathematical documents including articles, textbooks, interactive books, and courses.
\omdoc also serves as the content language for agent communication of mathematical
services on a mathematical software bus.

This {\ifbook part of the {\report}\else document\fi} is the specification of {\vomdoc{1.6}}
of the \omdoc format, the first step towards {\omdocv2}. It defines the \omdoc
language features and their meaning. The content of this part is normative for the
\omdoc format; an \omdoc document is valid as an \omdoc document, iff it meets all
the constraints imposed here. \omdoc applications will normally presuppose valid
\omdoc documents and only exhibit the intended behavior on such.
\end{omgroup}



%%% Local Variables: 
%%% mode: latex
%%% TeX-master: "main"
%%% End: 

% LocalWords:  mypart tocdepth athematical uments
