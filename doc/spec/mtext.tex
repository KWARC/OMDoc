%%%%%%%%%%%%%%%%%%%%%%%%%%%%%%%%%%%%%%%%%%%%%%%%%%%%%%%%%%%%%%%%%%%%%%%%%
% This file is part of the LaTeX sources of the OMDoc 1.6 specification
% Copyright (c) 2006 Michael Kohlhase
% This work is licensed by the Creative Commons Share-Alike license
% see http://creativecommons.org/licenses/by-sa/2.5/ for details
% The source original is at https://github.com/KWARC/OMDoc/doc/spec 
%%%%%%%%%%%%%%%%%%%%%%%%%%%%%%%%%%%%%%%%%%%%%%%%%%%%%%%%%%%%%%%%%%%%%%%%%

\begin{module}[id=mtext]
\begin{omgroup}[id=mtext,short=Mathematical Text]
  {Mathematical Text (Module {\MTXTmodule{spec}})}

\begin{omtext}
  The everyday mathematical language used in textbooks, conversations, and written onto
  blackboards all over the world consists of a {\indextoo{rigorous}}, slightly stylized
  version of {\twintoo{natural}{language}} interspersed with mathematical formulae,
  \inlinedef{that is sometimes called {\twindef{mathematical}{vernacular}}\footnote{The
      term ``mathematical vernacular'' was first introduced by Nicolaas Govert de Bruijn
      in the 1970s (see~\cite{DeBruijn:tmv94} for a discussion). It derives from the word
      ``vernacular'' used in the {\twintoo{Catholic}{church}} to distinguish the language
      used by {\indextoo{laymen}} from the official {\indextoo{Latin}}.}.}
\end{omtext}

\begin{presonly}
\begin{myfig}{qtcfmp}{The {\omdoc} Elements for Specifying Mathematical Properties}
\begin{scriptsize}
\begin{tabular}{|>{\tt}l|>{\tt}l|>{\tt}p{2.8truecm}|c|>{\tt}p{4truecm}|}\hline
{\rm Element}& \multicolumn{2}{l|}{Attributes\hspace*{2.25cm}} & M & Content  \\\hline
             & {\rm Required}  & {\rm Optional}                & D &           \\\hline\hline
 omtext  &  & xml:id, type, for, class, style, verbalizes    & +  & h:p* \\\hline
 h:p           & &xml:id, style, class, index, verbalizes & + & \llquote{math vernacular} \\\hline
 h:span  &  &  type, for, relation, verbalizes
                                                     & + & \llquote{math vernacular}\\\hline
 term      & name & cd, cdbase, role, xml:id, class, style & -- & \llquote{math vernacular}\\\hline
\end{tabular}
\end{scriptsize}
\end{myfig}
\end{presonly}

\begin{omgroup}[id=mathvernacular]{Mathematical Vernacular}

  {\omdoc} models mathematical vernacular as parsed text interspersed with
  content-carrying elements. Most prominently, the {\openmath} objects, {\cmathml}
  expressions, and {\element{legacy}} elements are used for mathematical objects, see
  {\sref{mobj}}. The text structure is marked up with the inline fragment of {\xhtml}
  1.0~\cite{W3C:xhtml2000}.

  In {\myfigref{modstable}} we have given an overview over the ones described in this
  book. The last two modules in {\myfigref{modstable}} are optional (see
  {\sref{sub-languages}}).  Other (external or future) {\omdoc} modules can introduce
  further elements; natural extensions come when {\omdoc} is applied to areas outside
  mathematics, for instance {\twintoo{computer science}{vernacular}} needs to talk about
  {\twintoo{code}{fragment}s} (see {\sref{private}} and~\cite{Kohlhase:codemlspec}),
  {\twintoo{chemistry}{vernacular}} about chemical formulae (e.g. represented in Chemical
  Markup Language~\cite{CML:web}).

\begin{myfig}{modstable}{{\omdoc} Modules Contributing to Mathematical Vernacular}
\begin{small}
  \begin{tabular}{|l|p{4cm}|p{3.5cm}|l|}\hline
    Module & Elements & Comment & see\\\hline\hline
    XHTML & {\element[ns-elt=h]{p}} and inline Elements & extended by \MTXTmodule{spec} &\cite{W3C:xhtml2000}\\\hline
   \MOBJmodule{spec} &  {\element[ns-elt=om]{OM*}}, {\element[ns-elt=m]{*}}, {\element{legacy}}
    & mathematical Objects 
    & \sref{mobj}\\\hline
    \MTXTmodule{spec}&  {\element[ns-elt=h]{span}}, {\element{term}}, {\element{note}}, {\element{idx}}, {\element{citation}}  
    & phrase-level markup
    & below \\\hline                  
    \DOCmodule{spec}  & {\element{ref}}, {\element{ignore}}
    & document structure
    & \sref{omdoc-infrastructure}\\\hline                  
   \EXTmodule{spec}  & {\element{omlet}} & for applets, images, \ldots 
    & \sref{eldef.omlet}\\\hline
  \end{tabular}
\end{small}
\end{myfig}

  As we have explicated above, all mathematical documents state properties of mathematical
  objects --- informally in mathematical vernacular or formally (as logical formulae), or
  both. {\omdoc} uses the {\element{omtext}} element to mark up text passages that form
  conceptual units, e.g. paragraphs, statements, or remarks.

\begin{definition}[id=omtext.def]
  {\element{omtext}} elements have an optional {\attribute[ns-attr=xml]{id}{omtext}}
  attribute, so that they can be {\indextoo{cross-reference}d}, the intended purpose of
  the text fragment in the larger document context can be described by the optional
  attribute {\attribute{type}{omtext}}.
\end{definition}
\end{omgroup}

\begin{omgroup}[id=omtext]{Rhetoric/Mathematical Roles of Text Fragments}

\begin{oldpart}{The rhethorical relations will be completely reworked taking the SALT
    ontology into account. For the moment we liberalize the RNC schema to allow
    {\texttt{xsd:anyURI}} here } This can take e.g. the values
  {\attval{abstract}{type}{omtext}}, {\attval{introduction}{type}{omtext}},
  {\attval{conclusion}{type}{omtext}}, {\attval{comment}{type}{omtext}},
  {\attval{thesis}{type}{omtext}}, {\attval{antithesis}{type}{omtext}},
  {\attval{elaboration}{type}{omtext}}, {\attval{motivation}{type}{omtext}},
  {\attval{evidence}{type}{omtext}}, {\attval{transition}{type}{omtext}} with the obvious
  meanings. In the last five cases {\element{omtext}} also has the extra attribute
  {\attribute{for}{omtext}}, and in the last one, also an attribute
  {\attribute{from}{omtext}}, since these are in reference to other {\omdoc} elements.
\end{oldpart}
The content of an {\element{omtext}} element is {\twintoo{mathematical}{vernacular}}
contained in a sequence of {\element[ns-elt=h]{p}} elements. This can be preceded by a
{\element{metadata}} element that can be used to specify authorship, give the passage a
title, etc. (see {\sref{dc-elements}}).

We have used the {\attribute{type}{omtext}} attribute on {\element{omtext}} to classify
text fragments by their {\twintoo{rhetoric}{role}}. This is adequate for much of the
generic text that makes up the {\indextoo{narrative}} and explanatory text in a
mathematical textbook. But many text fragments in mathematical documents directly state
properties of mathematical objects (we will call them
{\twintoo{mathematical}{statement}s}; see {\sref{statements}} for a more elaborated markup
infrastructure). These are usually classified as {\indextoo{definition}s},
{\indextoo{axiom}s}, etc.  Moreover, they are of a form that can (in principle) be
formalized up to the level of logical formula; in fact, mathematical vernacular is seen by
mathematicians as a more convenient form of communication for mathematical statements that
can ultimately be translated into a foundational logical system like axiomatic set
theory~\cite{Bernays:ast91}.  For such text fragments, {\omdoc} reserves the following
values for the {\attribute{type}{omtext}} attribute:
\begin{description}
\item[{\attval{axiom}{type}{omtext}}] (fixes or restricts the meaning of certain
  symbols or concepts.) An axiom is a piece of mathematical knowledge that cannot
  be derived from anything else we know.
\item[{\attval{definition}{type}{omtext}}] (introduces new concepts or symbols.) A
  definition is an axiom that introduces a new symbol or construct, without restricting
  the meaning of others.
\item[{\attval{example}{type}{omtext}}] (for or against a mathematical property).
\item[{\attval{proof}{type}{omtext}}] (a proof), i.e. a rigorous --- but maybe informal
  --- argument that a mathematical statement holds.
\item[{\attval{hypothesis}{type}{omtext}}] (a local assumption in a proof that will be
  discharged later) for text fragments that come from (parts of) proofs.
\item[{\attval{derive}{type}{omtext}}] (a step in a proof), we will specify the exact
  meanings of this and the two above in {\sref{proofs}} and present more structured
  counterparts.
\end{description} 
For the first four values, {\element{omtext}} also provides the attribute 
{\attribute{for}{omtext}}, as they point to other mathematical aspects such as symbols, 
assertions, definitions, axioms or alternatives.

Finally, {\omdoc} also reserves the values {\attval{theorem}{type}{omtext}},
{\attval{proposition}{type}{omtext}}, {\attval{lemma}{type}{omtext}},
{\attval{corollary}{type}{omtext}}, {\attval{postulate}{type}{omtext}},
{\attval{conjecture}{type}{omtext}}, {\attval{false-conjecture}{type}{omtext}},
{\attval{formula}{type}{omtext}}, {\attval{obligation}{type}{omtext}},
{\attval{assumption}{type}{omtext}}, {\attval{rule}{type}{omtext}} and
{\attval{assertion}{type}{omtext}} for statements that assert properties of mathematical
objects (see {\myfigref{assertion-types}} in {\sref{assertions}} for explanations). Note
that the differences between these values are largely pragmatic or proof-theoretic
(conjectures become theorems once there is a proof).  Further types of text can be
specified by providing a URI that points to a description of the text type (much like the
{\attribute[ns-elt=m]{definitionURL}{csymbol}} attribute on the
{\element[ns-elt=m]{csymbol}} elements in {\cmathml}).

Of course, the {\attribute{type}{omtext}} only allows a rough classification of the
{\twintoo{mathematical}{statement}s} at the text level, and does not make the underlying
{\twintoo{content}{structure}} explicit or reveals their contribution and interaction with
{\twintoo{mathematical}{context}}.  For that purpose {\omdoc} supplies a set of
specialized elements, which we will discuss in {\sref{statements}}.  Thus
{\element{omtext}} elements will be used to give informal accounts of mathematical
statements that are better and more fully annotated by the infrastructure introduced in
{\sref{statements}}. However, in narrative documents, we often want to be informal, while
maintaining a link to the formal element. For this purpose {\omdoc} provides the optional
{\attribute{verbalizes}{omtext}} attribute on the {\element{omtext}} element. Its value is
a whitespace-separated list of URI references to formal representations (see
{\sref{inline-statements}} for further discussion).
\end{omgroup}

\begin{omgroup}[id=phrases]{Phrase-Level Markup of Mathematical Vernacular}

\begin{omtext}
  To make the sentence-internal structure of mathematical vernacular more explicit,
  {\omdoc} provides an infrastructure to mark up natural language phrases in
  sentences. Linguistically, \inlinedef{a {\defin{phrase}} is a group of words that
    functions as a single unit in the syntax of a sentence}. Examples include ``noun
  phrases, verb phrases, or prepositional phrases''. In {\omdoc} we use the
  {\element[ns-elt=h]{span}} element from {\xhtml} a general wrapper for sentence-level
  phrases that allows to mark their specific properties with special attributes and a
  {\element{metadata}} child. The {\element{term}} element is naturally restricted to
  phrases by construction.
\end{omtext}

\begin{definition}[id=span.def]
  The {\eldef[ns-elt=h]{span}} element has the optional attribute
  {\attribute[ns-attr=xml,ns-elt=h]{id}{span}} for referencing the text fragment and the {\css}
  attributes {\attribute[ns-elt=h]{style}{span}} and {\attribute[ns-elt=h]{class}{span}}
  to associate presentation information with it (see the discussion in
  {\srefs{common-attribs}{omstyle}}).
\end{definition}

The semantics of the {\element[ns-elt=h]{span}} element is defined by mapping to the SALT
Rhetorical Ontology~\cite{Groza:SALT07} i.e. for example we define a
{\attval[ns-elt=h]{nucleus}{type}{span}} phrase to be an instance of
\url{http://salt.semanticauthoring.org/onto/rhetorical-ontology#nucleus}.  The
{\attribute[ns-elt=h]{type}{span}} attribute serves a linguistic purpose. A
{\element[ns-elt=h]{span}} denoting a part of a sentence that plays an important role in
the understanding of the entire text or is simply basic information essential to the
author's purpose takes the value of a {\attval[ns-elt=h]{nucleus}{type}{span}}. A
{\element[ns-elt=h]{span}} that plays a secondary role in the text, i.e. that serves
primarily to further explain or support the {\attval[ns-elt=h]{nucleus}{type}{span}} with
additional information takes the value of a
{\attval[ns-elt=h]{satellite}{type}{span}}. The main difference between these two concepts
is that a {\attval[ns-elt=h]{nucleus}{type}{span}} can be comprehended in a context of a
text by iself, while on the other hand a {\attval[ns-elt=h]{satellite}{type}{span}} is
incomprehensible without its corresponding {\attval[ns-elt=h]{nucleus}{type}{span}}
phrase. In order to further clarify and annotate this dependence, if a
{\element[ns-elt=h]{span}} element has a value {\attval[ns-elt=h]{satellite}{type}{span}}
for the {\attribute[ns-elt=h]{type}{span}} attribute, it also has the optional attributes
{\attribute[ns-elt=h]{for}{span}} and {\attribute[ns-elt=h]{relation}{span}}, which are
explained below.

  The {\attribute[ns-elt=h]{relation}{span}} attribute gives the type of dependency relation 
  connecting the {\attval[ns-elt=h]{satellite}{type}{span}} phrase with its corresponding 
  {\attval[ns-elt=h]{nucleus}{type}{span}} phrase. It can take one of the following values: 
  {\attval[ns-elt=h]{antithesis}{relation}{span}}, {\attval[ns-elt=h]{circumstance}{relation}{span}}, 
  {\attval[ns-elt=h]{concession}{relation}{span}}, {\attval[ns-elt=h]{condition}{relation}{span}}, 
  {\attval[ns-elt=h]{evidence}{relation}{span}}, {\attval[ns-elt=h]{means}{relation}{span}}, 
  {\attval[ns-elt=h]{preparation}{relation}{span}}, {\attval[ns-elt=h]{purpose}{relation}{span}}, 
  {\attval[ns-elt=h]{cause}{relation}{span}}, {\attval[ns-elt=h]{consequence}{relation}{span}}, 
  {\attval[ns-elt=h]{elaboration}{relation}{span}}, {\attval[ns-elt=h]{restatement}{relation}{span}} and 
  {\attval[ns-elt=h]{solutionhood}{relation}{span}}. We go through each of these terms 
  separately to further clarify their role and meaning.

\begin{description}
\item[{\attval[ns-elt=h]{antithesis}{relation}{span}}] is a relation where the author has a
  positive regard for the {\attval[ns-elt=h]{nucleus}{type}{span}}. The
  {\attval[ns-elt=h]{nucleus}{type}{span}} and the {\attval[ns-elt=h]{satellite}{type}{span}} are in
  contrast i.e. both can not be true, and the intention of the author is to increase the
  reader's positive regard towards the {\attval[ns-elt=h]{nucleus}{type}{span}}.
\item[{\attval[ns-elt=h]{circumstance}{relation}{span}}] is a relation where the situation
  presented in the {\attval[ns-elt=h]{satellite}{type}{span}} is unrealized. It simply provides a
  framework in the subject matter within which the reader is to interpret the
  {\attval[ns-elt=h]{nucleus}{type}{span}}.
\item[{\attval[ns-elt=h]{concession}{relation}{span}}] is a relation where the author yet again
  wants to increase the reader's positive regard for the
  {\attval[ns-elt=h]{nucleus}{type}{span}}. This time, by acknowledging a potential or apparent
  incompatibility between the {\attval[ns-elt=h]{nucleus}{type}{span}} and the
  {\attval[ns-elt=h]{satellite}{type}{span}}.
\item[{\attval[ns-elt=h]{condition}{relation}{span}}] is a relation in which the
  {\attval[ns-elt=h]{satellite}{type}{span}} represents a hypothetical future, i.e. unrealized
  situation and the realization of the statement given in the
  {\attval[ns-elt=h]{nucleus}{type}{span}} phrase depends on the realization of the situation
  described in the {\attval[ns-elt=h]{satellite}{type}{span}} phrase.
\item[{\attval[ns-elt=h]{evidence}{relation}{span}}] is a relation where the author wants to
  increase the reader's belief in the {\attval[ns-elt=h]{nucleus}{type}{span}} by providing the
  {\attval[ns-elt=h]{satellite}{type}{span}}, which is something that the reader believes in or
  will find credible.
\item[{\attval[ns-elt=h]{means}{relation}{span}}] is a relation where the
  {\attval[ns-elt=h]{satellite}{type}{span}} represents a method or an instrument which tends to
  make the realization of the situation presented in the {\attval[ns-elt=h]{nucleus}{type}{span}}
  phrase more likely.  For instance: {\presbf{The Gaussian algorithm solves a linear
      system of equations}}, by first reducing the given system to a triangular or echelon
  form using elementary row operations and then using a back substitution to find the
  solution.
\item[{\attval[ns-elt=h]{preparation}{relation}{span}}] is a relation where the
  {\attval[ns-elt=h]{satellite}{type}{span}} precedes the {\attval[ns-elt=h]{nucleus}{type}{span}} in the
  text, and tends to make the reader more ready, interested or oriented for reading what
  is to be stated in the {\attval[ns-elt=h]{nucleus}{type}{span}} phrase.
\item[{\attval[ns-elt=h]{purpose}{relation}{span}}] is a relation where the author wants the
  reader to recognize that the activity described in the {\attval[ns-elt=h]{nucleus}{type}{span}}
  phrase is initiated in order to realize what is described in the
  {\attval[ns-elt=h]{satellite}{type}{span}} phrase.
\item[{\attval[ns-elt=h]{cause}{relation}{span}}] is a relation where the author wants the reader
  to recognize that the {\attval[ns-elt=h]{satellite}{type}{span}} is the cause for the action
  described in the {\attval[ns-elt=h]{nucleus}{type}{span}} phrase.
\item[{\attval[ns-elt=h]{consequence}{relation}{span}}] is a relation where the author wants the
  reader to recognize that the action described in the {\attval[ns-elt=h]{nucleus}{type}{span}} is
  to have a result or consequences, as described in the {\attval[ns-elt=h]{satellite}{type}{span}}
  phrase.
\item[{\attval[ns-elt=h]{elaboration}{relation}{span}}] is a relation where the
  {\attval[ns-elt=h]{satellite}{type}{span}} phrase provides additional detail for the
  {\attval[ns-elt=h]{nucleus}{type}{span}}.  For instance: {\presbf{In elementary number theory,
      integers are studied without the use of techniques from other mathematical fields.}}
  Questions of divisibility, factorization into prime numbers, investigation of perfect
  numbers, use of the Euclidean algorithm to compute the GCD and congruences belong here.
\item[{\attval[ns-elt=h]{restatement}{relation}{span}}] is a relation where the
  {\attval[ns-elt=h]{satellite}{type}{span}} simply restates what is said in the
  {\attval[ns-elt=h]{nucleus}{type}{span}} phrase. However the {\attval[ns-elt=h]{nucleus}{type}{span}} is
  more central to the authors purposes than the {\attval[ns-elt=h]{satellite}{type}{span}} is.
  For instance: The somewhat older term arithmetic is also used to refer to number theory,
  but is no longer as popular as it once was\ldots \ldots {\presbf{Number theory used to
      be called "the higher arithmetic", but this is dropping out of use.}}
\item[{\attval[ns-elt=h]{solutionhood}{relation}{span}}] is a relation in which the
  {\attval[ns-elt=h]{nucleus}{type}{span}} phrase represents a solution to the problem(s)
  presented in the {\attval[ns-elt=h]{satellite}{type}{span}} phrase.
\end{description}

\begin{lstlisting}[label=lst:phrase-type-attribute, caption= Phrases and their attribute 
usage, index={h:span}]
<omtext>
  <h:span id="sat1.2" type="satellite" relation="concession" for="#nuc1.1">Although it 
  still shows up in the names of mathematical fields such as arithmetic functions, 
  arithmetic of eliptic curves, fundamental theorem of arithmetic
  </h:span>
  <h:span id="nuc1.1" type="nucleus"> the word arithmetic is no longer as popular as it once was
  </h:span>
</omtext>
\end{lstlisting}

  The {\attribute[ns-elt=h]{for}{span}} attribute, available when a {\element[ns-elt=h]{span}} is denoted to be a 
  {\attval[ns-elt=h]{satellite}{type}{span}}, is there to link the {\element[ns-elt=h]{span}} to its 
  corresponding {\attval[ns-elt=h]{nucleus}{type}{span}} phrase i.e. serves only for referential 
  purposes, holding the value of the URI of the {\attval[ns-elt=h]{nucleus}{type}{span}} phrase. 
  Thus having the phrases uniquely identified by the {\attribute[ns-attr=xml,ns-elt=h]{id}{span}} 
  attribute is highly encouraged due to its great relevance for weaving the semantics of 
  a text at this granularity level.

  Furthermore, the {\element[ns-elt=h]{span}} element allows the attribute
  {\attribute[ns-elt=h]{index}{span}} for
  {\atwintoo{parallel}{multilingual}{markup}}: Recall that sibling
  {\element{omtext}} elements form {\twintoo{multilingual}{group}s} of text fragments.  We
  can use the {\element[ns-elt=h]{span}} element to make the correspondence relation on
  text fragments more fine-grained: {\element[ns-elt=h]{span}} elements in sibling
  {\element{omtext}s} that have the same {\attribute[ns-elt=h]{index}{span}} value are
  considered to be equivalent.  Of course, the value of an
  {\attribute[ns-elt=h]{index}{span}} has to be unique in the dominating {\element{omtext}}
  element (but not beyond). Thus the {\attribute[ns-elt=h]{index}{span}} attributes
  simplify manipulation of {\indextoo{multilingual}} texts, see
  {\mylstref{parallel-multiling}} for an example at the discourse level.

  Finally, the {\element[ns-elt=h]{span}} element can carry a {\attribute[ns-elt=h]{verbalizes}{span}}
  attribute whose value is a whitespace-separated list of URI references that act as
  pointers to other {\omdoc} elements. This has two applications: 
  \begin{example}[display=flow] the first is another kind of parallel markup where we can
    state that a phrase corresponds to (and thus ``verbalizes'') a part of formula in a
    sibling {\element{FMP}} element.\ednote{MK: this needs to be done differently, rework
      this example}

\begin{lstlisting}[label=lst:parallel-formal-informal,mathescape,
  caption=Parallel Markup between Formal and Informal,
  index={h:span,h:p,FMP}]
<h:p>
  If <h:span verbalizes="#isaG">$\langle G,\circ\rangle$ is a group</h:span>, then of course
     <h:span verbalizes="#isaM">it is a monoid</h:span> by construction.
</h:p>
<FMP>
  <OMA><OMS cd="logic1" name="implies"/>
    <OMA id="isaG"><OMS cd="algebra" name="group"/>
      <OMA id="GG"><OMS cd="set" name="pair">
        <OMV name="G"/><OMV name="op"/>
      </OMA>
    </OMA>
    <OMA xml:id="isaM"><OMS cd="algebra" name="monoid"/>
      <OMR href="GG"/>
    </OMA>
  </OMA>
</FMP>
\end{lstlisting}
\end{example}
Another important application of the {\attribute[ns-elt=h]{verbalizes}{span}} is the case of
inline mathematical statements, which we will discuss in {\sref{inline-statements}}.

\end{omgroup}


\begin{omgroup}[id=terms]{Technical Terms}

\begin{omtext}
  In {\omdoc} we can give the notion of a \inlinedef{{\twindef{technical}{term}} a very
    precise meaning: it is a {\indextoo{span}} representing a {\indextoo{concept}} for
    which a {\indextoo{declaration}} exists in a {\twintoo{content}{dictionary}} (see
    {\sref{symbol-dec}})}. In this respect it is the natural language equivalent for an
  {\openmath} symbol or a {\cmathml} token\footnote{and is subject to the same visibility
    and scoping conditions as those; see {\sref{theories-contexts}} for details}. Let us
  consider an example: We can equivalently say ``$0\in\NN$'' and ``the number zero is a
  natural number''. The first rendering in a formula, we would cast as the following
  {\openmath} object:
\begin{lstlisting}[language=OpenMath,numbers=none]
<OMA><OMS cd="set1" name="in"/>
  <OMS cd="nat" name="zero"/>
  <OMS cd="nat" name="Nats"/>
</OMA>
\end{lstlisting}
with the effect that the components of the formula are disambiguated by pointing to the
respective content dictionaries. Moreover, this information can be used by added-value
services e.g. to cross-link the symbol presentations in the formula to their definition
(see {\extref{processing}{transform-xsl}}), or to detect logical dependencies. To allow
this for mathematical vernacular as well, we provide the {\element{term}} element: in our
example we might use the following markup.
\begin{lstlisting}[language=OpenMath,numbers=none,mathescape]
$\ldots$<term  cd="nat" name="zero">the number zero</term> is an 
<term cd="nat" name="Nats">natural number</term>$\ldots$
\end{lstlisting}
\end{omtext}

\begin{definition}[id=term.def]
  The {\eldef{term}} element has one required attribute: {\attribute{name}{term}} and two
  optional ones: {\attribute{cd}{term}} and {\attribute{cdbase}{term}}.  Together they
  determine the meaning of the phrase just like they do for {\element[ns-elt=om]{OMS}}
  elements (see the discussion in {\sref{openmath}} and {\sref{identifying}}). The
  {\element{term}} element also allows the attribute {\attribute[ns-attr=xml]{id}{term}}
  for identification of the phrase occurrence, the {\css}
  \twinalt{attributes}{CSS}{attribute} for styling and the optional
  {\attribute{role}{term}} attribute that allows to specify the role the respective phrase
  plays. We reserve the value {\attval{definiens}{role}{term}} for the defining occurrence
  of a phrase in a definition.  This will in general mark the exact point to point to when
  presenting other occurrences of the same\footnote{We understand this to mean with the
    same {\attribute{cd}{term}} and {\attribute{name}{term}} attributes.} phrase. Other
  attribute values for the {\attribute{role}{term}} are possible, {\omdoc} does not fix
  them at the current time.  Consider for instance the following text fragment from
  {\myfigref{bourbaki}} in {\extref{primer}{algebra}}.
\end{definition}

\begin{sblockquote}
  {\presbf{Definition 1.}} {\presem{Let $E$ be a set. A mapping of $E\times E$ is called a
      {\presbf{law of composition}} on $E$. The value $f(x,y)$ of $f$ for an ordered pair
      $(x,y)\in E\times E$ is called the {\presbf{composition of}} $x$ and $y$ under this
      law.  A set with a law of composition is called a magma.}}
\end{sblockquote}
Here the first boldface term is the definiens for a ``law of composition'', the second one
for the result of applying this to two arguments. It seems that this is not a totally
different concept that is defined here, but is derived systematically from the concept of
a ``law of composition'' defined before. Pending a thorough linguistic investigation we
will mark up such occurrences with {\attvalveryshort{definiens-applied}}, for instance in

\begin{lstlisting}[mathescape,caption={Marking up the Technical Terms},label=lst:terms]
Let $E$ be a set. A mapping of $E\times E$ is called a 
<term cd="magmas" name="law_of_comp" role="definiendum">law of composition</term> on $E$. 
The value $f(x,y)$ of $f$ for an ordered pair $(x,y)\in E\times E$ is called the 
<term  cd="magmas"name="law_of_comp" role="definiendum-applied">composition of</term>
$x$ and $y$ under this law.
\end{lstlisting}
There are probably more such systematic correlations; we leave their categorization and
modeling in {\omdoc} to the future.
\end{omgroup}

\begin{omgroup}[id=rt]{Index and Bibliography}


  \ednote{introduce idx and citation, and also describe makeindex and bibliography in
    doc!}

\begin{presonly}
\begin{myfig}{rt}{Rich Text Format {\omdoc}}
  \begin{scriptsize}
\begin{tabular}{|>{\tt}l|>{\tt}l|>{\tt}l|c|>{\tt}l|}\hline
{\rm Element}& \multicolumn{2}{l|}{Attributes\hspace*{2.25cm}} &MD& Content  \\\hline
 idx         & &(xml:id|xref)                           & -- & idt?, ide+ \\\hline
 ide         & &index, sort-by, see, seealso, links    & -- & idp* \\\hline
 idt         & &style, class                            & --&  \llquote{math vernacular} \\\hline
 idp         & &sort-by, see, seealso, links            & --&  \llquote{math vernacular} \\\hline
 note        & &type, xml:id, style, class, index, verbalizes & + & \llquote{math  vernacular} \\\hline
 citation & ref & & text\\\hline
\end{tabular}
\end{scriptsize}
\end{myfig}
\end{presonly}

\begin{definition}[id=idx.def,title=Index Markup]
  The {\eldef{idx}} element is used for index markup in {\omdoc}. It contains an optional
  {\eldef{idt}} element that contains the {\twintoo{index}{text}}, i.e. the phrase that is
  indexed. The remaining content of the index element specifies what is entered into
  various indexes. For every index this phrase is registered to there is one {\eldef{ide}}
  element ({\twintoo{index}{entry}}); the respective entry is specified by name in its
  optional {\attribute{index}{ide}} attribute. The {\element{ide}} element contains a
  sequence of {\twintoo{index}{phrase}s} given in {\eldef{idp}} elements. The content of
  an {\element{idp}} element is regular mathematical text. Since index entries are usually
  sorted, (and mathematical text is difficult to sort), they carry an attribute
  {\attribute{sort-by}{idp}} whose value (a sequence of Unicode characters) can be sorted
  lexically~\cite{Unicode:collation}. Moreover, each {\element{idp}} and {\element{ide}}
  element carries the attributes {\attribute{see}{idp,ide}},
  {\attribute{seealso}{idp,ide}}, and {\attribute{links}{idp,ide}}, that allow to specify
  extra information on these. The values of the first ones are references to
  {\element{idx}} elements, while the value of the {\attribute{links}{idp}} attribute is a
  whitespace-separated list of (external) URI references.  The formatting of the
  {\twintoo{index}{text}} is governed by the attributes {\attributeshort{style}} and
  {\attributeshort{class}} on the {\element{idt}} element. The {\element{idx}} element can
  carry either an {\attribute[ns-attr=xml]{id}{idx}} attribute (if this is the defining
  occurrence of the index text) or an {\attribute{xref}{idx}} attribute. In the latter
  case, all the {\element{ide}} elements from the defining {\element{idx}} (the one that
  has the {\attribute[ns-attr=xml]{id}{idx}} attribute) are imported into the referring
  {\element{idx}} element (the one that has the {\attribute{xref}{idx}} attribute).
\end{definition}


\begin{lstlisting}[label=lst:richtext,
   caption={An Example of Rich Text Structure},
   index={h:p,h:ul,h:li,h:span,h:a}]
<omtext>
  <h:p style="color:red" xml:id="p1">All <idx><idt>animals are dangerous</idt>
    <idp>dangerous</idp><idp  seealso="creature">animal</idp></idx>!
    (which is a highly <em>unfounded</em> statement). 
    In reality only some animals are, for instance:</h:p>
  <h:ul id="l1">
    <h:li>sharks (they bite) and </h:li>
    <h:li>bees (they sting).</h:li>
  </h:ul>
  <h:p>If we measure danger by the number of deaths, we obtain</h:p>
  <table xmlns="http://www.w3.org/1999/xhtml">
    <tr>             <th>Culprits</th> <th>Deaths</th> <th>Action</th></tr>
    <tr>             <td>sharks</td>   <td>312</td>    <td>bite</td></tr>
    <tr xml:id="bn"> <td>bees</td>     <td>23</td>     <td>sting</td></tr>
    <tr>             <td>cars</td>     <td>7500</td>   <td>crash</td></tr>
  </table>
  <h:p>So, if we do the numbers <note xml:id="n1" type="ednote">check the
  numbers again</note> we see that animals are dangerous, but they are 
  less so than cars but much more photogenic as we can see
  <h:a href="http://www.yellowpress.com/killerbee.jpg">here</h:a>.</h:p>

  <note type="footnote">From the International Journal of Bee-keeping; numbers only
  available for 2002.</note> 
</omtext>
\end{lstlisting}


\begin{definition}[id=note.def,title=Notes]
  The {\eldef{note}} element is the closest approximation to a {\indextoo{footnote}} or
  {\indextoo{endnote}}, where the kind of note is determined by the
  {\attribute{type}{note}} attribute. {\omdoc} supplies {\attval{footnote}{type}{note}} as
  a default value, but does not restrict the range of values. Its {\attribute{for}{note}}
  attribute allows it to be attached to other {\omdoc} elements externally where it is not
  allowed by the {\omdoc} document type. In our example, we have attached a footnote by
  reference to a table row, which does not allow {\element{note}} children.
\end{definition}

\begin{oldpart}{do we want to support parallel path markup in the future? It has not been
    used that much. Also, there is another place where this is explained.}
  All elements in the {\RTmodule{spec}} module carry an optional
  {\attributeshortcomment{xml:id}{in module RT}} attribute for identification and an
  {\attribute{index}{in module RT}} attribute for
  {\atwintoo{parallel}{multilingual}{markup}} (e.g. {\sref{phrases}} for an explanation
  and {\mylstref{parallel-multiling}} for a {\indextoo{translation}} example).

\begin{lstlisting}[label=lst:parallel-multiling,caption={Multilingual Parallel Markup},index={omtext,docalt,h:ul,h:li,h:p}]
<docalt xml:id="animals.overview">
  <omtext>
    <h:p index="intro">Consider the following animals:</h:p>
    <h:ul index="animals">
      <h:li index="first">a tiger,</h:li>
      <h:li index="second">a dog.</h:li>
    </h:ul>
  </omtext>
  <omtext xml:lang="de">
    <h:p index="intro">Betrachte die folgenden Tiere:</h:p>
    <h:ul index="animals">
      <h:li index="first">Ein Tiger</h:li>
      <h:li index="second">Ein Hund</h:li>
    </h:ul>
  </omtext>
</docalt>
\end{lstlisting}
\end{oldpart}
\end{omgroup}
\end{omgroup}
\end{module}

%%% Local Variables: 
%%% mode: latex
%%% TeX-master: "main"
%%% End: 

% LocalWords:  RT ids multiling lst ul li testtext lang Hier testen wir ein ol
% LocalWords:  mehrsprachigen parallelen erster Strichelstrich etwas anderes tr
% LocalWords:  td href xlink rt mtxt Req dtd mobj richtext ednote bn modstable
% LocalWords:  Betrachte folgenden Tiere und Hund ns elt idx attr dl di dt dd
% LocalWords:  Nicolaas Govert ref omlet alsoin alsoincmp lang en fr nl lst
% LocalWords:  escapechar multiling mathescape trl aut func im cd eq fns xref
% LocalWords:  Sei eine Menge ist Funktion da und Une unaire est une fonction
% LocalWords:  avec fol hol ple qtcfmp binop mtxt de Bruijn truecm FMP xml
% LocalWords:  omtext mtext DTD ol ul mobj attribs omstyle une une
% LocalWords:  OMA OMS nat Nats xsl openmath bourbaki magmas comp OMV OMR href
% LocalWords:  un op defini tion jecture definitionURL csymbol def omdoc sump
% LocalWords:  idt idp seealso multi metadata une et ide th une restructurred
% LocalWords:  CoP Soit une isaG isaM une GG une dass une gc dec une
