%%%%%%%%%%%%%%%%%%%%%%%%%%%%%%%%%%%%%%%%%%%%%%%%%%%%%%%%%%%%%%%%%%%%%%%%%
% This file is part of the LaTeX sources of the OMDoc 1.6 specification
% Copyright (c) 2006 Michael Kohlhase
% This work is licensed by the Creative Commons Share-Alike license
% see http://creativecommons.org/licenses/by-sa/2.5/ for details
% The source original is at https://github.com/KWARC/OMDoc/doc/spec 
%%%%%%%%%%%%%%%%%%%%%%%%%%%%%%%%%%%%%%%%%%%%%%%%%%%%%%%%%%%%%%%%%%%%%%%%%

\begin{omgroup}{Derived Statements}

\begin{module}[id=derived-defs]
\begin{omgroup}[short=Derived Definitions]{Derived Definition Forms}

\begin{oldpart}{fit into the picture here}
\begin{definition}[display=flow,id=well-defined.def]
  We say that a definiendum is \defi{well-defined}, iff the corresponding definiens
  uniquely determines it;
\end{definition}
adding such definitions to a theory always results in a conservative extension.

\begin{myfig}{math-def}{Some Common Definitions}\small
 \begin{tabular}{|p{2.2cm}|p{7cm}|l|}\hline
   Definiens & Definiendum & Type \\\hline\hline
   The number 1 & $1\deq s(0)$ (1 is the successor of 0) & simple\\\hline
   The exponential function $e^\cdot$ 
         &  The {\textbf{exponential function}} $e^\cdot$ is the solution to the
   differential equation $\partial f=f$ [where $f(0)=1$]. & implicit\\\hline
   The addition function + & {\textbf{Addition}} on the natural numbers is defined by the equations
   $x+0=x$ and $x+s(y)=s(x+y)$. & recursive\\\hline
 \end{tabular}
\end{myfig}

\begin{omtext}
  Definitions can have many forms, they can be
\begin{itemize}
\item equations where the left hand side is the defined symbol and the right hand side is
  a term that does not contain it, as in our discussion above or the first case in
  {\myfigref{math-def}}. \inlinedef{We call such definitions
    \adefii{simple}{simple}{definition}.}
\item general statements that uniquely determine the meaning of the objects or concepts in
  question, as in the second definition in {\myfigref{math-def}}. \inlinedef{We call such
    definitions \adefii{implicit}{implicit}{definition}; the
    {\twintoo{Peano}{axioms}} are another example of this category.}
  
  Note that this kind of definitions requires a proof of unique existence to ensure
  {\indextoo{well-defined}ness}.  Incidentally, if we leave out the part in square
  brackets in the second definition in {\myfigref{math-def}}, the differential equation
  only characterizes the exponential function up to additive real constants. In this case,
  the ``definition'' only restricts the meaning of the exponential function to a set of
  possible values.  \inlinedef{We call such a set of axioms a
    \adefii{loose}{loose}{definition} definition.}
\item given as a set of equations, as in the third case of {\myfigref{math-def}}, even
  though this is strictly a special case of an implicit definition: it is a sub-case,
  where well-definedness can be shown by giving an argument why the systematic
  applications of these equations \indexalt{terminates}{termination}, e.g.  by exhibiting
  an {\indextoo{ordering}} that makes the left hand sides strictly smaller than the
  right-hand sides. \inlinedef{We call such a definition
    \adefii{inductive}{definition}{inductive}.}
\end{itemize}
\end{omtext}

\begin{presonly}
  \begin{myfig}{alternative-theory}{Theory-Constitutive Elements in \omdoc}
    \begin{scriptsize}
      \begin{tabular}{|>{\snippet}l|>{\tt}l|>{\tt}p{3.6truecm}|c|>{\tt}p{3.2truecm}|}\hline
        {\rm Element}& \multicolumn{2}{l|}{Attributes\hspace*{2.25cm}} & M & Content  \\\hline
                     & {\rm Required}  & {\rm Optional}                & C &          \\\hline\hline
          definition & for & xml:id, type, style, class, uniqueness, existence 
                                                                       & + & h:p* \llquote{mobj}  \\\hline
          definition & for & xml:id, type, style, class, consistency, exhaustivity & +  
                           & h:p*, requation+, measure?, ordering?  \\\hline
          requation  &           & xml:id, style, class    & -- & \llquote{mobj},\llquote{mobj} \\\hline
          measure    &           & xml:id, style, class    & -- & \llquote{mobj} \\\hline
          ordering   &           & xml:id, style, class    & -- & \llquote{mobj} \\\hline
 \multicolumn{5}{|l|}{where \llquote{mobj} is {\tt{(\mobjabbr)}}}\\\hline
\end{tabular}
\end{scriptsize}
\end{myfig}
\end{presonly}

In {\myfigref{math-def}} we have seen that there are many ways to fix the meaning of a
symbol, therefore \omdoc \element{definition} elements are more complex than
\element{axiom}s.  In particular, the \element{definition} element supports several kinds
of definition mechanisms with specialized content models specified in the
\attribute{type}{definition} attribute (cf. the discussion at the end of
\sref{statements-constitutive}):

\begin{module}[id=implicit-defs]
\begin{omgroup}{Implicit Definitions}

This kind of definition is often (more accurately) called ``{\emph{\indextoo{definition by
      description}}}'', since the {\indextoo{definiendum}} is described so accurately,
that there is exactly one object satisfying the description. The ``description'' of the
defined symbol is given as a multi-system \element{FMP} group whose content uniquely
determines the value of the symbols that are specified in the
\attribute{for}{definition} attribute of the \element{definition} element with
\attribute{type}{definition} \attribute{implicit}{type}{definition}.  The necessary
statement of unique existence can be specified in the \attribute{existence}{definition}
and \attribute{uniqueness}{definition} attribute, whose values are
{\twintoo{URI}{reference}s} to to assertional statements (see
\sref{assertional-statements}) that represent the respective properties.  We give
an example of an implicit definition in {\mylstref{exp-def}}.

\begin{lstlisting}[label=lst:exp-def,mathescape,
  caption={An Implicit Definition of the Exponential Function},
  index={definition,assertion,type,uniqueness,existence}]
<definition xml:id="exp-def" for="#exp" type="implicit" 
            uniqueness="#exp-unique" existence="#exp-exists">
  <FMP>$exp\sp{\prime}=exp\wedge exp(0)=1$</FMP>
</definition>
<assertion xml:id="exp-unique">
  <h:p>
    There is at most one differentiable function that solves the 
    differential equation in definition <ref type="cite" xref="#exp-def"/>.
  </h:p>
</assertion>
<assertion xml:id="exp-exists">
  <h:p>
    The differential equation in <ref type="cite" xref="#exp-def"/> is solvable.
  </h:p>
</assertion>
\end{lstlisting}
\end{omgroup}
\end{module}

\begin{module}[id=inductive-defs]
\begin{omgroup}{Inductive Definitions}

\begin{omtext}
  This is a variant of the \attval{implicit}{type}{definition} case above. It defines a
  {\twintoo{recursive}{function}} by a set of {\twintoo{recursive}{equation}s}
  \inlinedef{in {\eldef{requation}} elements whose left and right hand sides are specified
    by the two children}. The first one is called \inlinedef{the \defi{pattern}}, and
  the second one \inlinedef{the \defi{value}}. The intended meaning of the defined
  symbol is, that the value (with the variables suitably substituted) can be substituted
  for a formula that matches the pattern element. In this case, the \element{definition}
  element carries a \attribute{type}{definition} with value
  \attval{inductive}{type}{definition} and the optional attributes
  \attribute{exhaustivity}{definition} and \attribute{consistency}{definition}, which
  point to \element{assertion}s stating that the cases spanned by the patterns are
  exhaustive (i.e. all cases are considered), or that the values are consistent (where the
  cases overlap, the values are equal).
\end{omtext}

\Mylstref{recursive} gives an example of a a recursive definition of the addition on the
natural numbers.
\begin{lstlisting}[label=lst:recursive,mathescape,
  caption={A recursive definition of addition},
  index={definition,requation}]
<definition xml:id="plus.def" for="#plus" type="inductive" 
            consistency="#s-not-0" exhaustivity="#s-or-0">
  <metadata><dc:subject>addition</dc:subject></metadata>
  <h:p>Addition is defined by recursion on the second argument.</h:p>
  <requation>$x+0\leadsto x$</requation>
  <requation>$x+s(y)\leadsto s(x+y)$</requation>
</definition>
\end{lstlisting} 
To guarantee termination of the recursive instantiation (necessary to ensure
well-definedness), it is possible to specify a {\indextoo{measure function}} and
well-founded {\indextoo{ordering}} by the optional \element{measure} and
\element{ordering} elements which contain mathematical objects. The elements contain
mathematical objects.

\begin{definition}[id=measure.def]
  The content of the {\eldef{measure}} element specifies a measure function, i.e. a
  function from argument tuples for the function defined in the parent
  \element{definition} element to a space with an ordering relation which is specified
  in the {\eldef{ordering}} element. This element also carries an optional attribute
  \attribute{terminating}{measure} that points to an \element{assertion} element that
  states that this ordering relation is a terminating partial ordering.
\end{definition}

\begin{definition}[id=pattern.def] 
  \adefii{Pattern definitions}{pattern}{definition} are a special degenerate cases
  of the recursive definition. A function is defined by a set of \element{requation}
  elements, but the defined function does not occur in the second children.
\end{definition}
This form of definition is often used instead of \attval{simple}{type}{definition} in
logical languages that do not have a function constructor. It allows to define a function
by its behavior on patterns of arguments. Since termination is trivial in this case, no
\element{measure} and {\snippet{ordering}} elements appear in the body of a
\element{definition} element whose \attribute{type}{definition} has value
\attval{pattern}{type}{definition}.
\end{omgroup}
\end{module}
\end{oldpart}
\end{omgroup}
\end{module}

\begin{module}[id=adt]
\begin{omgroup}[id=adt,short=Abstract Data Types]
                           {Abstract Data Types (Module {\ADTmodule{spec}})}

\begin{module}[id=adt]
 Most specification languages for mathematical theories support definition mechanisms for
sets that are inductively generated by a set of constructors and
{\twintoo{recursive}{function}s} on these under the heading of {\twintoo{abstract}{data
    type}s}. Prominent examples of abstract data types are natural numbers, lists, trees,
etc. The module {\ADTmodule{spec}} presented in this chapter extends \omdoc by a concise
syntax for abstract data types that follows the model used in the {\casl} (Common Abstract
Specification Language~\cite{CoFI:2004:CASL-RM}) standard.

\begin{omtext}
  Conceptually, an abstract data type declares a collection of symbols and axioms that can
  be used to construct certain mathematical objects and to group them into sets. For
  instance, the {\twintoo{Peano}{axioms}} (see {\myfigref{peano}}) introduce the symbols
  $0$ (the number {\indextoo{zero}}), $s$ (the {\twintoo{successor}{function}}), and $\NN$
  (the set of {\twintoo{natural}{number}s}) and fix their meaning by five axioms. These
  state that the set $\NN$ contains exactly those objects that can be constructed from $0$
  and $s$ alone (\inlinedef{these symbols are called \defiis{constructor}{symbol} and the
    representations \defii{constructor}{term}s}). Optionally, an abstract data type can
  also declare \inlinedef{\defiis{selector}{symbol}, for (partial) inverses of the
    constructors}. In the case of natural numbers the {\twintoo{predecessor}{function}} is
  a selector for $s$: it ``selects'' the argument $n$, from which a (non-zero) number
  $s(n)$ has been constructed.
\end{omtext}

\begin{definition}[display=flow,id=free.def]
  Following {\casl} we will call sets of objects that can be represented as constructor
  terms \defis{sort}. A sort is called\defi{free}, iff there are no identities between
  constructor terms, i.e.  two objects represented by different constructor terms can
  never be equal.
\end{definition}

\begin{omtext}
  The sort $\NN$ of natural numbers is a free sort. An example of a sort that is not free
  is the theory of finite sets given by the constructors $\emptyset$ and the
  {\twintoo{set}{insertion}} function $\iota$ , since the set $\{a\}$ can be obtained by
  inserting $a$ into the empty set an arbitrary (positive) number of times; so
  e.g. $\iota(a,\emptyset)=\iota(a,\iota(a,\emptyset))$. \inlinedef{This kind of sort is
    called \defi{generated}, since it only contains elements that are expressible in
    the constructors}.
\end{omtext}

\begin{definition}[display=flow,id=generated.def]
  An abstract data type is called \defi{loose}, if it contains elements besides the
  ones generated by the constructors. We consider free sorts more \defi{strict} than
  generated ones, which in turn are more strict than loose ones.
\end{definition}

\begin{presonly}
\begin{myfig}{adtheory}{Abstract data types in \omdoc}
\begin{scriptsize}
\begin{tabular}{|>{\tt}l|>{\tt}l|>{\tt}p{3.3truecm}|c|>{\tt}p{3truecm}|}\hline
{\rm Element}& \multicolumn{2}{l|}{Attributes\hspace*{2.25cm}} & M & Content  \\\hline
             & {\rm Req.}  & {\rm Optional}     & D &           \\\hline\hline
 adt         &          & xml:id, class, style, parameters  & +  & sortdef+\\\hline
 sortdef     & name     & type, role, scope, class, style & + & (constructor | insort)*, recognizer? \\\hline
 constructor & name     & type, scope, class, style & +  & argument*\\\hline
 argument    &          &              & +  & type, selector?\\\hline
 insort      & for      &              & -- & \\\hline
 selector    & name     & type, scope, role, total, class, style
                                       & + & EMPTY\\\hline
 recognizer  & name     & type, scope, role, class, style & + & \\\hline
\end{tabular}
\end{scriptsize}
\end{myfig}
\end{presonly}

\begin{definition}[id=adt.def]
  In \omdoc, we use the {\eldef{adt}} element to specify abstract data types possibly
  consisting of multiple sorts.  It is a {\indextoo{theory-constitutive}} statement and
  can only occur as a child of a \element{theory} element (see
  \sref{statements-constitutive} for a discussion). An \element{adt} element contains
  one or more \element{sortdef} elements that define the {\indextoo{sort}s} and specify
  their members and it can carry a \attribute{parameters}{adt} attribute that contains a
  whitespace-separated list of parameter variable names. If these are present, they
  declare type variables that can be used in the specification of the new sort and
  constructor symbols see {\extref{projects}{verifun}} for an example.
\end{definition}

We will use an augmented representation of the abstract data type of natural numbers as a
running example for introduction of the functionality added by the {\ADTmodule{spec}}
module; {\mylstref{nat-adt}} contains the listing of the \omdoc encoding. In this
example, we introduce a second sort $\bbP$ for {\atwintoo{positive}{natural}{number}s} to
make it more interesting and to pin down the {\indextoo{type}} of the
{\twintoo{predecessor}{function}}.

\begin{definition}[id=sordef.def]
  A {\eldef{sortdef}} element is a highly condensed piece of syntax that declares a
  {\twintoo{sort}{symbol}} together with the {\twintoo{constructor}{symbol}s} and their
  {\twintoo{selector}{symbol}s} of the corresponding sort. It has a required
  \attribute{name}{sortdef} attribute that specifies the symbol name, an optional
  \attribute{type}{adt} attribute that can have the values \attribute{free}{type}{adt},
  \attval{generated}{type}{adt}, and \attval{loose}{type}{adt} with the meaning
  discussed above. A \element{sortdef} element contains a set of {\eldef{constructor}}
  and {\eldef{insort}} elements.  The latter are empty elements which refer to a sort
  declared elsewhere in a \element{sortdef} with their \attribute{for}{insort}
  attribute: An \element{insort} element with
  {\snippet{for="}}\llquote{URI}{\snippet{\#\llquote{name}"}} specifies that all the
  constructors of the sort {\snippet{\llquote{name}}} are also constructors for the one
  defined in the parent \element{sortdef}. Furthermore, the type of a sort given by a
  \element{sortdef} element can only be as strict as the types of any sorts included by
  its \element{insort} children.
\end{definition}

{\Mylstref{nat-adt}} introduces the {\twintoo{sort}{symbol}s} {\snippet{pos-nats}}
(positive natural numbers) and {\snippet{nats}} (natural numbers) , the symbol names are
given by the required \attribute{name}{constructor} attribute. Since a constructor is in
general an $n$-ary function, a \element{constructor} element contains $n$
\element{argument} children that specify the argument sorts of this function along with
possible selector functions.

\begin{definition}[id=argument.def]
  The argument sort is given as the first child of the {\eldef{argument}} element: a
  \element{type} element as described in \sref{type-axioms}.
\end{definition}
Note that $n$ may be 0 and thus the constructor element may not have \element{argument}
children (see for instance the \element{constructor} for {\snippet{zero}} in
{\mylstref{nat-adt}}). The first \element{sortdef} element there introduces the
constructor symbol {\snippet{succ@Nat}} for the {\twintoo{successor}{function}}. This
function has one argument, which is a natural number (i.e. a member of the sort
{\snippet{nats}}).

Sometimes it is convenient to specify the inverses of a constructors that are
functions. For this \omdoc offers the possibility to add an empty \element{selector}
element as the second child of an \element{argument} child of a
\element{constructor}.

\begin{definition}[id=selector.def]
  The {\eldef{selector}} element has a required attribute \attribute{name}{selector}
  specifies the symbol name, the optional \attribute{total}{selector} attribute of the
  \element{selector} element specifies whether the function represented by this symbol
  is \twinalt{total}{total}{function} (value \attval{yes}{total}{selector}) or
  \twinalt{partial}{partial}{function} (value \attval{no}{total}{selector}).  In
  {\mylstref{nat-adt}} the \element{selector} element in the first \element{sortdef}
  introduces a {\twintoo{selector}{symbol}} for the {\twintoo{successor}{function}}
  {\snippet{succ}}. As {\snippet{succ}} is a function from {\snippet{nats}} to
  {\snippet{pos-nats}}, {\snippet{pred}} is a total function from {\snippet{pos-nats}} to
  {\snippet{nats}}.
\end{definition}

\begin{definition}[id=recognizer.def]
  Finally, a \element{sortdef} element can contain a {\eldef{recognizer}} child that
  specifies a symbol for a {\indextoo{predicate}} that is true, iff its argument is of the
  respective sort. The name of the predicate symbol is specified in the required
  \attribute{name}{recognizer} attribute.
\end{definition} 

\begin{omtext} 
  \Mylstref{nat-adt} introduces such a \inlinedef{\defii{recognizer}{predicate}} as
  the last child of the \element{sortdef} element for the sort {\snippet{pos-nats}}.
\end{omtext}

Note that the \element{sortdef}, \element{constructor}, \element{selector},
and \element{recognizer} elements define symbols of the name specified by their
\attribute{name}{symbol} element in the theory that contains the \element{adt}
element. To govern the visibility, they carry the attribute
\attribute{scope}{symbol} (with values \attribute{global}{scope}{symbol} and
\attval{local}{scope}{symbol}) and the attribute \attribute{role}{symbol}
(with values \attval{type}{role}{symbol}, \attval{sort}{role}{symbol},
\attval{object}{role}{symbol}).
\begin{lstlisting}[label=lst:nat-adt,
  caption={The natural numbers using \element{adt} in \omdoc},
  index={adt,sortdef,constructor,argument,selector,recognizer,insort}]
<theory xml:id="Nat">
  <adt xml:id="nat-adt">
    <metadata>
      <dc:title>Natural Numbers as an Abstract Data Type.</dc:title>
      <dc:description>The Peano axiomatization of natural numbers.</dc:description>
    </metadata>

    <sortdef name="pos-nats" type="free">
      <metadata>
        <dc:description>The set of positive natural numbers.</dc:description>
      </metadata>
      <constructor name="succ">
        <metadata><dc:description>The successor function.</dc:description></metadata>
        <argument>
          <type><OMS cd='Nat' name="nats"/></type>
          <selector name="pred" total="yes">
            <metadata><dc:description>The predecessor function.</dc:description></metadata>
          </selector>
        </argument>
      </constructor>
      <recognizer name="positive">
        <metadata>
          <dc:description>
            The recognizer predicate for positive natural numbers.
          </dc:description>
        </metadata>
      </recognizer>
    </sortdef>

    <sortdef name="nats"  type="free">
      <metadata><dc:description>The set of natural numbers</dc:description></metadata>
      <constructor name="zero">
        <metadata><dc:description>The number zero.</dc:description></metadata>
      </constructor>
      <insort for="#pos-nats"/>
    </sortdef>
  </adt>
</theory>
\end{lstlisting}
To summarize {\mylstref{nat-adt}}: The abstract data type {\snippet{nat-adt}} is free and
defines two sorts {\snippet{pos-nats}} and {\snippet{nats}} for the (positive) natural
numbers. The positive numbers ({\snippet{pos-nats}}) are generated by the successor
function (which is a constructor) on the natural numbers (all positive natural numbers are
successors). On {\snippet{pos-nats}}, the inverse {\snippet{pred}} of {\snippet{succ}} is
{\indextoo{total}}.  The set {\snippet{nats}} of all natural numbers is defined to be the
union of {\snippet{pos-nats}} and the constructor {\snippet{zero}}.  Note that this
definition implies the five well-known Peano Axioms: the first two specify the
constructors, the third and fourth exclude identities between constructor terms, while the
induction axiom states that {\snippet{nats}} is generated by {\snippet{zero}} and
{\snippet{succ}}.  The document that contains the {\snippet{nat-adt}} could also contain
the symbols and axioms defined implicitly in the \element{adt} element explicitly as
\element{symbol} and \element{axiom} elements for reference.  These would then carry
the \attribute{generated-from}{axiom} attribute with value
{\snippet{nat-adt}}.
\end{module}
\end{omgroup}
\end{module}

\begin{omgroup}[id=adt-strict]{Strict Translations}

  We will now give the a formal\ednote{do we really want to call it ``formal''?} semantics
  of the {\STmodule{spec}} elements in terms of strict \omdoc (see
  \sref{strict}).\ednote{what do we do if there is both FMP and CMPs in an
    axiom?}\ednote{what do we do if there is more than one symbol per
    definition?}\ednote{what do we do for non-simple definitions}

Implicit definitions are licensed by a {\twintoo{description}{operator}} in the
meta-theory. In a theory  {\llquote{t}}, whose meta-theory {\llquote{m}} contains a
description operator {\llquote{that}} we have\ednote{point to a theory with description
  operator or similar functionality. Maybe give a special theory $D$ and allow all
  meta-theories {\llquote{t}} that have a view from $D$.}\ednote{do we want a description
  operator that takes the existence and uniqueness proofs as arguments? Can we point to
  \element{proof} elements somehow? What do we really do with the attributes?}
\begin{center}\lstset{frame=none,numbers=none,lineskip=-.7ex,aboveskip=-.5em,belowskip=-1em,mathescape}
 \begin{tabular}{|p{6.5cm}|p{7.5cm}|}\hline
    pragmatic & strict\\\hline
\begin{lstlisting}
<symbol name="$\llquote{n}$">
  <type system="$\llquote{s}$">$\llquote{t}$</type>
</symbol>
<definition type="implicit" 
            xml:id="$\llquote{i}$" for="$\llquote{f}$"
            uniqueness="$\llquote{u}$" existence="$\llquote{e}$">
  <FMP>$\llquote{\ensuremath{\Phi[n]}}$</FMP>
</definition>
\end{lstlisting}
&
\begin{lstlisting}
<object name="$\llquote{n}$" xml:id="$\llquote{i}$">
  <type system="$\llquote{s}$">$\llquote{t}$</type>
  <definition>
    <OMBIND>
      <OMS cd="$\llquote{m}$" name="$\llquote{that}$"/>
      <OMBVAR><OMV name="$\llquote{x}$"/></OMBVAR>
      $\psom{\Phi[x]}$
    </OMBIND>
 </definition>
</object>
\end{lstlisting}
\\\hline 
\end{tabular}
\end{center}
where $\Phi[x]$ is obtained from $\Phi[n]$ by replacing all {\snippet{<OMS
    cd="\llquote{t}'' name="\llquote{n}"/>}} by {\snippet{<OMV name="\llquote{n}''/>}}.

Similarly inductive definitions are licensed by a recursion operator {\llquote{rec}}:
\begin{center}\lstset{frame=none,numbers=none,lineskip=-.7ex,aboveskip=-.5em,belowskip=-1em}
  \begin{tabular}{|p{6.5cm}|p{7.5cm}|}\hline
    pragmatic & strict\\\hline
{
\begin{lstlisting}[numbers=none,frame=none,mathescape]
<symbol name="$\llquote{n}$">
  <type system="$\llquote{s}$">$\llquote{t}$</type>
</symbol>
<definition type="inductive" 
            xml:id="$\llquote{i}$" for="$\llquote{f}$"
            consistency="$\llquote{c}$" exhaustivity="$\llquote{e}$">
  <requation>$\llquote{\ensuremath{\Phi_1[n]}}\llquote{\ensuremath{\Psi_1[n]}}$</requation>
  $\ldots$
  <requation>$\llquote{\ensuremath{\Phi_m[n]}}\llquote{\ensuremath{\Psi_m[n]}}$</requation>
</definition>
\end{lstlisting}
}&{
\begin{lstlisting}[numbers=none,frame=none,mathescape]
<object name="$\llquote{n}$" xml:id="$\llquote{i}$">
  <type system="$\llquote{s}$">$\llquote{t}$</type>
  <definition>
    <OMBIND>
      <OMS cd="$\llquote{m}$" name="$\llquote{rec}$"/>
      <OMBVAR><OMV name="$\llquote{x}$"/></OMBVAR>
      $\psom{\Phi_1[x]}\;\psom{\Psi_2[x]}\;\ldots\;
       \psom{\Phi_m[x]}\;\psom{\Psi_m[x]}$
    </OMBIND>
  </definition>
</object>
\end{lstlisting}
}\\\hline 
\end{tabular}
\end{center}
\ednote{what is the correct form of the recursion operator?}\ednote{similar question with
  the attributes here.}
\end{omgroup}
\end{omgroup}

%%% Local Variables: 
%%% mode: latex
%%% TeX-master: "main"
%%% End: 

% LocalWords:  adt suc pred emptyset adtheory sortdef nat insort prediate lst
% LocalWords:  pos nats succ inductively ary omdoc peano Req dc cd metadata
% LocalWords: verifun
