%%%%%%%%%%%%%%%%%%%%%%%%%%%%%%%%%%%%%%%%%%%%%%%%%%%%%%%%%%%%%%%%%%%%%%%%%
% This file is part of the LaTeX sources of the OMDoc 1.6 book
% Copyright (c) 2006 Michael Kohlhase
% This work is licensed by the Creative Commons Share-Alike license
% see http://creativecommons.org/licenses/by-sa/2.5/ for details
% The source original is at https://github.com/KWARC/OMDoc/doc/book
%%%%%%%%%%%%%%%%%%%%%%%%%%%%%%%%%%%%%%%%%%%%%%%%%%%%%%%%%%%%%%%%%%%%%%%%%

\begin{omgroup}[display=flow]{Acknowledgments}

  \markboth{Acknowledgments}{Acknowledgments}\thispagestyle{empty} Of course the {\omdoc}
  format has not been developed by one person alone. The original proposal was taken up by
  several research groups, most notably the {\OMEGA} group at Saarland University, the
  {\maya} and {\activemath} projects at the German Research Center of Artificial
  Intelligence (DFKI), the {\sc{MoWGLI}} EU Project, the RIACA group at the Technical
  University of Eindhoven, and the {\sc{CourseCapsules}} project at Carnegie Mellon
  University.  They discussed the initial proposals, represented their materials in
  {\omdoc} and in the process refined the format with numerous suggestions and
  discussions.

  The author specifically would like to thank Serge Autexier, Bernd Krieg-Br\"uckner, Olga
  Caprotti, David Carlisle, Claudio Sacerdoti Coen, Arjeh Cohen, Armin Fiedler, Andreas
  Franke, George Goguadze, Alberto Gonz\'alez Palomo, Dieter Hutter, Andrea Kohlhase,
  Christoph Lange, Paul Libbrecht, Erica Melis, Till Mossakowski, Normen M\"uller,
  Immanuel Normann, Martijn Oostdijk, Martin Pollet, Julian Richardson, Manfred Riem, and
  Michel Vollebregt for their input, discussions, and feedback from implementations and
  applications.

  Special thanks are due to Alan Bundy and J\"org Siekmann. The first triggered the work
  on {\omdoc}, has lent valuable insight over the years, and has graciously consented to
  write the foreword to this book. J\"org continually supported the {\omdoc} idea with his
  abundant and unwavering enthusiasm. In fact the very aim of the {\omdoc} format:
  openness, cooperation, and philosophic adequateness came from the spirit in his {\OMEGA}
  group, which the author has had the privilege to belong to for more than 10 years.

  The work presented in this {\report} was supported by the ``Deutsche
  For\-schungs\-gemeinschaft'' in the special research action ``Resource-adaptive
  cognitive processes'' (SFB 378), and a three-year Heisenberg Stipend to the author.
  Carnegie Mellon University, SRI International, and the International University Bremen
  have supported the author while working on revisions for versions 1.1 and 1.2.
\end{omgroup}

%%% Local Variables: 
%%% mode: LaTeX
%%% TeX-master: "main"
%%% End: 

% LocalWords:  MoWGLI RIACA CourseCapsules Autexier Bernd Krieg uckner Caprotti
% LocalWords:  Carlisle Sacerdoti Coen Arjeh Fiedler Franke Goguadze Gonz alez
% LocalWords:  Palomo Hutter Lange Libbrecht Melis Mossakowski Normen uller SFB
% LocalWords:  Normann Martijn Oostdijk Pollet Riem Vollebregt Bundy Joerg SRI
% LocalWords:  Siekmann Deutsche schungs gemeinschaft Br stex
