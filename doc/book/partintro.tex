%%%%%%%%%%%%%%%%%%%%%%%%%%%%%%%%%%%%%%%%%%%%%%%%%%%%%%%%%%%%%%%%%%%%%%%%%
% This file is part of the LaTeX sources of the OMDoc 1.6 book
% Copyright (c) 2006 Michael Kohlhase
% This work is licensed by the Creative Commons Share-Alike license
% see http://creativecommons.org/licenses/by-sa/2.5/ for details
\svnInfo $Id: partintro.tex 8429 2009-07-19 11:09:13Z kohlhase $
\svnKeyword $HeadURL: https://svn.omdoc.org/repos/omdoc/trunk/doc/book/partintro.tex $
%%%%%%%%%%%%%%%%%%%%%%%%%%%%%%%%%%%%%%%%%%%%%%%%%%%%%%%%%%%%%%%%%%%%%%%%%

\mockchapter
  In this part of the book we will look at the problem of marking up mathematical
  knowledge and mathematical documents in general, situate the {\omdoc} format, and
  compare it to other formats like {\openmath} and {\mathml}.

  The {\omdoc} format is an open markup language for mathematical documents and the
  \twinalt{knowledge}{mathematical}{knowledge} encapsulated in them. The representation in
  {\omdoc} makes the document content unambiguous and their context transparent.

  {\omdoc} approaches this goal by embedding control codes into mathematical documents
  that identify the document structure, the meaning of text fragments, and their relation
  to other mathematical knowledge in a process called {\emph{\twintoo{document}{markup}}}.
  Document markup is a communication form that has existed for many years. Until the
  computerization of the printing industry, markup was primarily done by a copy editor
  writing instructions on a manuscript for a typesetter to follow. Over a period of time,
  a standard set of symbols was developed and used by copy editors to communicate with
  typesetters on the intended appearance of documents. As computers became widely
  available, authors began using word processing software to write and edit their
  documents.  Each word processing program had its own method of markup to store and
  recall documents.

  Ultimately, the goal of all markup is to help the recipient of the document better cope
  with the content by providing additional information e.g. by visual cues or explicit
  structuring elements. Mathematical texts are usually very carefully designed to give
  them a structure that supports understanding of the complex nature of the objects
  discussed and the argumentations about them.  Such documents are usually structured
  according to the argument made and enhanced by specialized notation (mathematical
  formulae) for the particular objects.\footnote{Of course this holds not only for texts
    in pure mathematics, but for any argumentative text, including texts from the sciences
    and engineering disciplines.  We will use the adjective ``mathematical'' in an
    inclusive way to make this distinction on text form, not strictly on the scientific
    labeling.}  In contrast, the structure of texts like novels or poems normally obey
  different (e.g. aesthetic) constraints.

  In mathematical discourses, conventions about document form, numbering, typography,
  formula structure, choice of glyphs for concepts, etc. and the corresponding markup
  codes have evolved over a long scientific history and by now carry a lot of the
  information needed to understand a particular text. But since they pre-date the computer
  age, they were developed for the consumption by humans (mathematicians) and mainly with
  ``{\indextoo{ink-on-paper}}'' representations (books, journals, letters) in mind, which
  turns out to be too limited in many ways.

  In the age of {\twintoo{Internet}{publication}} and
  {\atwintoo{mathematical}{software}{system}s}, the universal accessibility of the
  documents breaks an assumption implicit in the design of traditional
  {\twintoo{mathematical}{document}s}: namely that the reader will come from the same
  \twinalt{(scientific) background}{scientific}{background} as the author and will
  directly understand the {\indextoo{notation}s} and {\twintoo{structural}{convention}s}
  used by the author.  We can also rely less and less on the premise that mathematical
  documents are primarily for human consumption as mathematical software systems are more
  and more embedded into the process of doing mathematics. This, together with the fact
  that mathematical documents are primarily produced and stored on computers, places a
  much heavier burden on the markup format, since it has to make all of this implicit
  information explicit in the communication.

  In the next two chapters we will set the stage for the {\omdoc} approach. We will first
  discuss general issues in markup formats (see {\sref{markup-types}}), existing
  solutions (see {\sref{markup:www}}), and the current
  {\xml}-based framework for {\twintoo{markup}{language}s} on the web
  (see {\sref{xml}}).  Then we will elaborate the special
  requirements for marking up the content of mathematics (see {\sref{math-markup}}).


%%% Local Variables: 
%%% mode: latex
%%% TeX-master: "main"
%%% End: 

% LocalWords:  www
